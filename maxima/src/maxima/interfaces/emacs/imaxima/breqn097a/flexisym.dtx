% \iffalse meta-comment
%
% Copyright (C) 1997-2003 by Michael J. Downes
% Copyright (C) 2007 by Morten Hoegholm <mh.ctan@gmail.com>
%
% This work may be distributed and/or modified under the
% conditions of the LaTeX Project Public License, either
% version 1.3 of this license or (at your option) any later
% version. The latest version of this license is in
%    http://www.latex-project.org/lppl.txt
% and version 1.3 or later is part of all distributions of
% LaTeX version 2005/12/01 or later.
%
% This work has the LPPL maintenance status "maintained".
%
% This Current Maintainer of this work is Morten Hoegholm.
%
% This work consists of the main source file flexisym.dtx
% and the derived files
%    flexisym.sty, flexisym.pdf, flexisym.ins, flexisym.drv,
%    cmbase.sym, msabm.sym.
%
% Distribution:
%    CTAN:macros/latex/contrib/mh/flexisym.dtx
%    CTAN:macros/latex/contrib/mh/flexisym.pdf
%
% Unpacking:
%    (a) If flexisym.ins is present:
%           tex flexisym.ins
%    (b) Without flexisym.ins:
%           tex flexisym.dtx
%    (c) If you insist on using LaTeX
%           latex \let\install=y% \iffalse meta-comment
%
% Copyright (C) 1997-2003 by Michael J. Downes
% Copyright (C) 2007 by Morten Hoegholm <mh.ctan@gmail.com>
%
% This work may be distributed and/or modified under the
% conditions of the LaTeX Project Public License, either
% version 1.3 of this license or (at your option) any later
% version. The latest version of this license is in
%    http://www.latex-project.org/lppl.txt
% and version 1.3 or later is part of all distributions of
% LaTeX version 2005/12/01 or later.
%
% This work has the LPPL maintenance status "maintained".
%
% This Current Maintainer of this work is Morten Hoegholm.
%
% This work consists of the main source file flexisym.dtx
% and the derived files
%    flexisym.sty, flexisym.pdf, flexisym.ins, flexisym.drv,
%    cmbase.sym, msabm.sym.
%
% Distribution:
%    CTAN:macros/latex/contrib/mh/flexisym.dtx
%    CTAN:macros/latex/contrib/mh/flexisym.pdf
%
% Unpacking:
%    (a) If flexisym.ins is present:
%           tex flexisym.ins
%    (b) Without flexisym.ins:
%           tex flexisym.dtx
%    (c) If you insist on using LaTeX
%           latex \let\install=y% \iffalse meta-comment
%
% Copyright (C) 1997-2003 by Michael J. Downes
% Copyright (C) 2007 by Morten Hoegholm <mh.ctan@gmail.com>
%
% This work may be distributed and/or modified under the
% conditions of the LaTeX Project Public License, either
% version 1.3 of this license or (at your option) any later
% version. The latest version of this license is in
%    http://www.latex-project.org/lppl.txt
% and version 1.3 or later is part of all distributions of
% LaTeX version 2005/12/01 or later.
%
% This work has the LPPL maintenance status "maintained".
%
% This Current Maintainer of this work is Morten Hoegholm.
%
% This work consists of the main source file flexisym.dtx
% and the derived files
%    flexisym.sty, flexisym.pdf, flexisym.ins, flexisym.drv,
%    cmbase.sym, msabm.sym.
%
% Distribution:
%    CTAN:macros/latex/contrib/mh/flexisym.dtx
%    CTAN:macros/latex/contrib/mh/flexisym.pdf
%
% Unpacking:
%    (a) If flexisym.ins is present:
%           tex flexisym.ins
%    (b) Without flexisym.ins:
%           tex flexisym.dtx
%    (c) If you insist on using LaTeX
%           latex \let\install=y% \iffalse meta-comment
%
% Copyright (C) 1997-2003 by Michael J. Downes
% Copyright (C) 2007 by Morten Hoegholm <mh.ctan@gmail.com>
%
% This work may be distributed and/or modified under the
% conditions of the LaTeX Project Public License, either
% version 1.3 of this license or (at your option) any later
% version. The latest version of this license is in
%    http://www.latex-project.org/lppl.txt
% and version 1.3 or later is part of all distributions of
% LaTeX version 2005/12/01 or later.
%
% This work has the LPPL maintenance status "maintained".
%
% This Current Maintainer of this work is Morten Hoegholm.
%
% This work consists of the main source file flexisym.dtx
% and the derived files
%    flexisym.sty, flexisym.pdf, flexisym.ins, flexisym.drv,
%    cmbase.sym, msabm.sym.
%
% Distribution:
%    CTAN:macros/latex/contrib/mh/flexisym.dtx
%    CTAN:macros/latex/contrib/mh/flexisym.pdf
%
% Unpacking:
%    (a) If flexisym.ins is present:
%           tex flexisym.ins
%    (b) Without flexisym.ins:
%           tex flexisym.dtx
%    (c) If you insist on using LaTeX
%           latex \let\install=y\input{flexisym.dtx}
%        (quote the arguments according to the demands of your shell)
%
% Documentation:
%    (a) If flexisym.drv is present:
%           latex flexisym.drv
%    (b) Without flexisym.drv:
%           latex flexisym.dtx; ...
%    The class ltxdoc loads the configuration file ltxdoc.cfg
%    if available. Here you can specify further options, e.g.
%    use A4 as paper format:
%       \PassOptionsToClass{a4paper}{article}
%
%    Programm calls to get the documentation (example):
%       pdflatex flexisym.dtx
%       makeindex -s gind.ist flexisym.idx
%       pdflatex flexisym.dtx
%       makeindex -s gind.ist flexisym.idx
%       pdflatex flexisym.dtx
%
% Installation:
%    TDS:tex/latex/mh/flexisym.sty
%    TDS:tex/latex/mh/cmbase.sym
%    TDS:tex/latex/mh/mathpazo.sym
%    TDS:tex/latex/mh/mathptmx.sym
%    TDS:tex/latex/mh/msabm.sym
%    TDS:doc/latex/mh/flexisym.pdf
%    TDS:source/latex/mh/flexisym.dtx
%
%<*ignore>
\begingroup
  \def\x{LaTeX2e}
\expandafter\endgroup
\ifcase 0\ifx\install y1\fi\expandafter
         \ifx\csname processbatchFile\endcsname\relax\else1\fi
         \ifx\fmtname\x\else 1\fi\relax
\else\csname fi\endcsname
%</ignore>
%<*install>
\input docstrip.tex
\Msg{************************************************************************}
\Msg{* Installation}
\Msg{* Package: flexisym 2007/12/19 v0.96 Flexisym (MH)}
\Msg{************************************************************************}

\keepsilent
\askforoverwritefalse

\preamble

This is a generated file.

Copyright (C) 1997-2003 by Michael J. Downes
Copyright (C) 2007 by Morten Hoegholm <mh.ctan@gmail.com>

This work may be distributed and/or modified under the
conditions of the LaTeX Project Public License, either
version 1.3 of this license or (at your option) any later
version. The latest version of this license is in
   http://www.latex-project.org/lppl.txt
and version 1.3 or later is part of all distributions of
LaTeX version 2005/12/01 or later.

This work has the LPPL maintenance status "maintained".

This Current Maintainer of this work is Morten Hoegholm.

This work consists of the main source file flexisym.dtx
and the derived files
   flexisym.sty, flexisym.pdf, flexisym.ins, flexisym.drv,
   cmbase.sym, mathpazo.sym, mathptmx.sym, msabm.sym.

\endpreamble

\generate{%
  \file{flexisym.ins}{\from{flexisym.dtx}{install}}%
  \file{flexisym.drv}{\from{flexisym.dtx}{driver}}%
  \usedir{tex/latex/mh}%
  \file{flexisym.sty}{\from{flexisym.dtx}{package}}%
  \file{cmbase.sym}{\from{flexisym.dtx}{cmbase}}%
  \file{mathpazo.sym}{\from{flexisym.dtx}{mathpazo}}%
  \file{mathptmx.sym}{\from{flexisym.dtx}{mathptmx}}%
  \file{msabm.sym}{\from{flexisym.dtx}{msabm}}%
}

\obeyspaces
\Msg{************************************************************************}
\Msg{*}
\Msg{* To finish the installation you have to move the following}
\Msg{* files into a directory searched by TeX:}
\Msg{*}
\Msg{*     flexisym.sty, cmbase.sym, mathpazo.sym, mathptmx.sym, msabm.sym}
\Msg{*}
\Msg{* To produce the documentation run the file `flexisym.drv'}
\Msg{* through LaTeX.}
\Msg{*}
\Msg{* Happy TeXing!}
\Msg{*}
\Msg{************************************************************************}

\endbatchfile
%</install>
%<*ignore>
\fi
%</ignore>
%<*driver>
\NeedsTeXFormat{LaTeX2e}
\ProvidesFile{flexisym.drv}%
  [2007/12/19 v0.96 flexisym (MH)]
\documentclass{ltxdoc}
\CodelineIndex
\EnableCrossrefs
\setcounter{IndexColumns}{2}
\providecommand*\pkg[1]{\textsf{#1}}
\providecommand*\cls[1]{\textsf{#1}}
\providecommand*\opt[1]{\texttt{#1}}
\providecommand*\env[1]{\texttt{#1}}
\providecommand*\fn[1]{\texttt{#1}}
\makeatletter
\providecommand{\AmS}{{\protect\AmSfont
  A\kern-.1667em\lower.5ex\hbox{M}\kern-.125emS}}
\providecommand{\AmSfont}{%
  \usefont{OMS}{cmsy}{\if\expandafter\@car\f@series\@nil bb\else m\fi}{n}}
\makeatother
\newenvironment{aside}{\begin{quote}\bfseries}{\end{quote}}
\begin{document}
  \DocInput{flexisym.dtx}
\end{document}
%</driver>
% \fi
%
% \title{The \textsf{flexisym} package}
% \date{2007/12/19 v0.96}
% \author{Morten H\o gholm \\\texttt{mh.ctan@gmail.com}}
%
% \maketitle
%
% \part*{User's guide}
%
% For now, the user's guide is in breqn.
%
% \StopEventually{}
% \part*{Implementation}
% 
% \section{flexisym}
%
%    \begin{macrocode}
%<*package>
\ProvidesPackage{flexisym}[2007/12/19 v0.96]
\let\@xp\expandafter \let\@nx\noexpand
\edef\do{%
  \@nx\AtEndOfPackage{%
    \catcode\number`\"=\number\catcode`\"
    \relax
  }%
}
\do \let\do\relax
\catcode`\"=12
\let\@sym\@gobble
\DeclareOption{robust}{%
  \def\@sym#1{%
    \ifx\protect\@typeset@protect \else\protect#1\@xp\@gobblefour\fi
  }%
}
\def\mg@bin{2}% binary operators
\def\mg@rel{2}% relations
%%\def\mg@nre{B}% negated relations
\def\mg@del{3}% delimiters
%%\def\mg@arr{B}% arrows
\def\mg@acc{0}% accents
\def\mg@cop{3}% cumulative operators (sum, int)
\def\mg@latin{1}% (Latin) letters
\def\mg@greek{1}% (lowercase) Greek
\def\mg@Greek{0}% (capital) Greek
%%\def\mg@bflatin{4}% bold upright Latin letters ?
%%\def\mg@Bbb{B}% blackboard bold
\def\mg@cal{2}% script/calligraphic
%%\def\mg@frak{5}% Fraktur letters
\def\mg@digit{0}% decimal digits % 1 = oldstyle, 0 = capital
\expandafter\let\csname MathChar \endcsname\mathchar
\expandafter\let\csname Delimiter \endcsname\delimiter
\expandafter\let\csname Radical \endcsname\radical
\newcommand{\MathChar}{}
\edef\MathChar{\csname MathChar \endcsname\noexpand\string}
\newcommand{\Delimiter}{}
\edef\Delimiter{\csname Delimiter \endcsname\noexpand\string}
\newcommand{\Radical}{}
\edef\Radical{\csname Radical \endcsname\noexpand\string}
\let\sumlimits\displaylimits
\let\intlimits\nolimits
\let\namelimits\displaylimits
\edef\m@Ord#1#2#3{\csname MathChar \endcsname"0#1#2#3 }
\edef\m@Var#1#2#3{\csname MathChar \endcsname"7#1#2#3 }
\edef\m@Bin#1#2#3{\csname MathChar \endcsname"2#1#2#3 }
\edef\m@Rel#1#2#3{\csname MathChar \endcsname"3#1#2#3 }
\edef\m@Pun#1#2#3{\csname MathChar \endcsname"6#1#2#3 }
\edef\m@COs#1#2#3{\csname MathChar \endcsname"1#1#2#3 \sumlimits}
\edef\m@COi#1#2#3{\csname MathChar \endcsname"1#1#2#3 \intlimits}
\def\delim@a#1#2#3#4{\ifx\relax#1#2#3#4\else#1\fi #2#3#4}
\def\delim@b#1#2#3#4{\ifx\relax#1#2#3#4\else#1\fi }
\def\@tempa{%
  \@nx\@xp\@nx\delim@a\@nx\csname sd@##1##2##3\@nx\endcsname ##1##2##3 }
\edef\m@DeL#1#2#3{\csname Delimiter \endcsname"4\@tempa}
\edef\m@DeR#1#2#3{\csname Delimiter \endcsname"5\@tempa}
\edef\m@DeB#1#2#3{\csname Delimiter \endcsname"0\@tempa}
\edef\m@DeA#1#2#3{\csname Delimiter \endcsname"3\@tempa}
\edef\m@Rad#1#2#3{\csname Radical \endcsname"\@tempa}
\def\do#1#2{\@xp\def\csname sd@#1\endcsname{#2}}
\do{300}{028}
\do{301}{029}
\do{302}{05B}
\do{303}{05D}
\do{304}{262}
\do{305}{263}
\do{306}{264}
\do{307}{265}
\do{308}{266}
\do{309}{267}
\do{30A}{268}
\do{30B}{269}
\do{30C}{26A}
\do{30D}{26B}
\do{30E}{13D}
\do{30F}{26E}
\do{340}{37A}
\do{341}{37B}
\do{33A}{33A}
\do{33B}{33B}
\do{33E}{33E}
\do{33C}{26A}
\do{33D}{26B}
\do{378}{222}
\do{379}{223}
\do{33F}{26C}
\do{37E}{22A}
\do{37F}{22B}
\do{377}{26D}
\do{30F}{26E}
\def\m@Acc#1#2#3#4{\mathaccent"#1#2#3{#4}}
\def\@symAcc{\@sym}
\let\@symtype\@firstofone
\def\@symOrd#1#2{\@symtype\mathord{\OrdSymbol{#2}}}
\def\@symVar{\@symOrd}
\def\@symBin#1#2{\@symtype\mathbin{\OrdSymbol{#2}}}
\def\@symRel#1#2{\@symtype\mathrel{\OrdSymbol{#2}}}
\def\@symPun#1#2{\@symtype\mathpunct{\OrdSymbol{#2}}}
\def\@symCOi#1#2{\@symtype{\mathop{\OrdSymbol{#2}}\intlimits}}
\def\@symCOs#1#2{\@symtype{\mathop{\OrdSymbol{#2}}\sumlimits}}
\def\@symOpe#1#2{\@symtype\mathopen{\OrdSymbol{#2}}}
\def\@symClo#1#2{\@symtype\mathclose{\OrdSymbol{#2}}}
\def\@symDeL#1#2{\@symtype\mathopen{\OrdSymbol{#2}}}
\def\@symDeR#1#2{\@symtype\mathclose{\OrdSymbol{#2}}}
\def\@symDeB#1#2{\@symtype\mathord{\OrdSymbol{#2}}}
\def\@symInn#1#2{\@symtype\mathinner{\OrdSymbol{#2}}}
\def\@xnce#1{\@xp\@nx\csname#1\endcsname}
\let\sym@global\global
\def\DeclareFlexSymbol#1#2#3#4{%
  \begingroup
  \edef\@tempb{\@nx\@sym\@nx#1\@xnce{m@#2}\@xnce{mg@#3}#4}%
  \ifcat\@nx#1\relax
    \sym@global\let#1\@tempb
  \else
    \sym@global\mathcode`#1="8000\relax
    \lccode`\~=`#1\relax
    \lowercase{\sym@global\let~\@tempb}%
  \fi
  \endgroup
}
\def\DeclareFlexCompoundSymbol#1#2#3{%
  \@xp\DeclareRobustCommand\@xp#1\@xp{\csname @sym#2\endcsname#1{#3}}%
  \sym@global\let#1#1\relax
}
\DeclareRobustCommand\textchar{\text@char\textfont}
\DeclareRobustCommand\scriptchar{\text@char\scriptfont}%
\def\text@char@a{\?\endgroup}%
\def\text@char@sym#1#2#3{%
  \begingroup
    \let\@sym\relax % defense against infinite loops
    \the\text@script@char#3%
    \afterassignment\text@char@a
    \chardef\?="%
}
\def\text@char#1#2{\begingroup\check@mathfonts
  \let\text@script@char#1\let\@sym\text@char@sym
  \let\@symtype\@secondoftwo \let\OrdSymbol\@firstofone
  \let\ifmmode\iftrue \everymath{$\@gobble}%$
  \def\mkern{\muskip\z@}\let\mskip\mkern
  \ifcat\relax\noexpand#2#2%
  \else
    \lccode`\~=\expandafter`\string#2\relax
    \lowercase{~}%
  \fi
  \endgroup
}
\providecommand\textprime{}
\DeclareRobustCommand\textprime{\leavevmode
  \raise.8ex\hbox{\text@char\scriptfont\prime}%
}
\@ifundefined{resetMathstrut@}{}{%
  \def\resetMathstrut@{%
    \setbox\z@\hbox{\textchar\vert}%
    \ht\Mathstrutbox@\ht\z@ \dp\Mathstrutbox@\dp\z@
  }%
}
\@ifundefined{rightarrowfill@}{}{%
  \def\rightarrowfill@#1{\m@th\setboxz@h{$#1\relbar$}\ht\z@\z@
    $#1\copy\z@\mkern-6mu\cleaders
    \hbox{$#1\mkern-2mu\box\z@\mkern-2mu$}\hfill
    \mkern-6mu\OrdSymbol{\rightarrow}$}
  \def\leftarrowfill@#1{\m@th\setboxz@h{$#1\relbar$}\ht\z@\z@
    $#1\OrdSymbol{\leftarrow}\mkern-6mu\cleaders
    \hbox{$#1\mkern-2mu\copy\z@\mkern-2mu$}\hfill
    \mkern-6mu\box\z@$}
  \def\leftrightarrowfill@#1{\m@th\setboxz@h{$#1\relbar$}\ht\z@\z@
    $#1\OrdSymbol{\leftarrow}\mkern-6mu\cleaders
    \hbox{$#1\mkern-2mu\box\z@\mkern-2mu$}\hfill
    \mkern-6mu\OrdSymbol{\rightarrow}$}
}
\def\binrel@sym#1#2#3#4#5{%
  \xdef\binrel@@##1{%
    \ifx\m@Ord#2\@nx\@symOrd
    \else\ifx\m@Var#2\@nx\@symVar
    \else\ifx\m@COs#2\@nx\@symCOs
    \else\ifx\m@COi#2\@nx\@symCOi
    \else\ifx\m@Bin#2\@nx\@symBin
    \else\ifx\m@Rel#2\@nx\@symRel
    \else\ifx\m@Pun#2\@nx\@symPun
    \else\@nx\@symErr \fi\fi\fi\fi\fi\fi\fi
  ?{\@nx\OrdSymbol{##1}}}%
}
\def\binrel@a{%
  \def\@symOrd##1##2{\gdef\binrel@@####1{\@symOrd##1{\OrdSymbol{####1}}}}%
  \def\@symVar##1##2{\gdef\binrel@@####1{\@symVar##1{\OrdSymbol{####1}}}}%
  \def\@symCOs##1##2{\gdef\binrel@@####1{\@symCOs##1{\OrdSymbol{####1}}}}%
  \def\@symCOi##1##2{\gdef\binrel@@####1{\@symCOi##1{\OrdSymbol{####1}}}}%
  \def\@symBin##1##2{\gdef\binrel@@####1{\@symBin##1{\OrdSymbol{####1}}}}%
  \def\@symRel##1##2{\gdef\binrel@@####1{\@symRel##1{\OrdSymbol{####1}}}}%
  \def\@symPun##1##2{\gdef\binrel@@####1{\@symPun##1{\OrdSymbol{####1}}}}%
}
\def\binrel@#1{%
  \setbox\z@\hbox{$%
    \let\mathchoice\@gobblethree
    \let\@sym\binrel@sym \binrel@a
    #1$}%
}
\def\@symextension{sym}
\newcommand\usesymbols[1]{%
  \@for\@tempb:=#1\do{%
    \@xp\@onefilewithoptions\@xp{\@tempb}[][]\@symextension
  }%
}
\newcommand\ProvidesSymbols[1]{\ProvidesFile{#1.sym}}
\DeclareRobustCommand{\not}[1]{\@symRel\not{\OrdSymbol{\notRel#1}}}
\DeclareRobustCommand{\OrdSymbol}[1]{%
  \begingroup\mathchars@reset#1\endgroup
}
\def\mathchars@reset{\let\@sym\@sym@ord \let\@symtype\@symtype@ord
  \let\OrdSymbol\relax}
\def\@symtype@ord#1#{}% a strange sort of \@gobble
\def\@sym@ord#1#2{\@xp\@sym@ord@a\string#2\@nil}%
\begingroup
\lccode`\.=`\@ \lowercase{\endgroup
\def\@sym@ord@a#1.}#2#3\@nil#4#5#6{%
  \csname MathChar \endcsname"0%
    \if D#2\@xp\delim@b\csname sd@#4#5#6\endcsname#4#5#6
    \else #4#5#6
    \fi
}
%    \end{macrocode}
%
%
% Before declaring any math characters active, we have to take care of
% a small problem with \pkg{amsmath} v2.x, if it is loaded before
% \pkg{flexisym}. \cs{std@minus} and \cs{std@equal} are defined as
% \begin{verbatim}
% \mathchardef\std@minus\mathcode`\-\relax
% \mathchardef\std@equal\mathcode`\=\relax
% \end{verbatim}
% in \fn{amsmath.sty} and again \cs{AtBeginDocument}. The
% latter is because
% \begin{quote}
%   In case some alternative math fonts are loaded
%   later. [\fn{amsmath.dtx}]
% \end{quote}
% The problem arises because \pkg{flexisym} sets the mathcode of all
% symbols to $32768$ which is illegal for a \cs{mathchardef}. 
%
% We have to remove the assignments from the \cs{AtBeginDocument} hook
% as they will cause an error there.
%    \begin{macrocode}
\@ifpackageloaded{amsmath}{%
  \begingroup
%    \end{macrocode}
% Split the contents of \cs{@begindocumenthook} by reading what we
% search for as a delimited argument and ensure these two assignments
% do not take place. It is questionable if anything reasonable can be
% done to them. In the case of a package such as \pkg{mathpazo} which defines
% \begin{verbatim}
%\DeclareMathSymbol{=}{\mathrel}{upright}{"3D}
% \end{verbatim}
% the \cs{Relbar} will look wrong if we don't use the correct
% symbol. The way to solve this is define additional \fn{.sym} files
% which contain the definition of \cs{relbar} and \cs{Relbar}
% needed. We need those additional files anyway for things like
% \cs{joinord}.
%    \begin{macrocode}
  \long\def\next#1\mathchardef\std@minus\mathcode`\-\relax
                  \mathchardef\std@equal\mathcode`\=\relax#2\flexi@stop{%
    \toks@{#1#2}%
    \xdef\@begindocumenthook{\the\toks@}%
  }%
  \expandafter\next\@begindocumenthook\flexi@stop
  \endgroup
}{}
%    \end{macrocode}
%
% There is problem when using \cs{DeclareMathOperator} as the
% operators defined call a command \cs{newmcodes@} which relies on the
% mathcode of \texttt{-} being less than 32768. We delay the
% definition \cs{AtBeginDocument} in case \pkg{amssymb} hasn't been
% loaded yet.
%    \begin{macrocode}
\AtBeginDocument{%
\def\newmcodes@{%
  \mathcode `\'39\mathcode `\*42\mathcode `\."613A
  \ifnum\mathcode`\-=45
  \else
%    \end{macrocode}
% The extra check. Don't do anything if \texttt{-} is math active.
%    \begin{macrocode}
    \ifnum\mathcode`\-=32768
    \else 
      \mathchardef \std@minus \mathcode `\-\relax
    \fi
  \fi
  \mathcode `\-45 \mathcode `\/47\mathcode `\:"603A\relax 
}%
}
%    \end{macrocode}
%
% And we then continue with the options.
%    \begin{macrocode}
\DeclareOption{mathstyleoff}{\PassOptionsToPackage{mathactivechars}{mathstyle}}
\DeclareOption{cmbase}{\usesymbols{cmbase}}
\DeclareOption{mathpazo}{\usesymbols{mathpazo}}
\DeclareOption{mathptmx}{\usesymbols{mathptmx}}
\ExecuteOptions{cmbase}
\ProcessOptions\relax
\renewcommand{\lnot}{\neg}
\renewcommand{\land}{\wedge}
\renewcommand{\lor}{\vee}
\renewcommand{\le}{\leq}
\renewcommand{\ge}{\geq}
\renewcommand{\ne}{\neq}
\renewcommand{\owns}{\ni}
\renewcommand{\gets}{\leftarrow}
\renewcommand{\to}{\rightarrow}
\renewcommand{\|}{\Vert}
\RequirePackage{mathstyle}
%</package>\endinput
%    \end{macrocode}
%
% \section{cmbase, mathpazo, mathptmx}
%
%
% For each math font package we define a corresponding symbol file
% with extension \fn{sym}. The Computer Modern base is called
% \opt{cmbase} and \opt{mathpazo} and \opt{mathptmx} corresponds to
% the packages. The definitions are almost identical as they mostly
% concern the positions in the math font encodings. Look for
% differences in \cs{joinord}, \cs{relbar} and \cs{Relbar}. If you
% inspect the source code, you'll see that the support for
% \pkg{mathptmx} didn't require any work but I thought it better to
% create a \fn{sym} file to maintain a uniform interface.
%
% \begin{aside}
% Open question on \verb"!" and \verb"?": maybe they
% should have type `Pun' instead of `DeR'.    Need to
% search for uses in math in AMS archives.    Or, maybe add a special
% `Clo' type for them: non-extensible closing delimiter.   
% \end{aside}
% 
% 
% 
% Default mathgroup setup.   
%    \begin{macrocode}
%<*cmbase|mathpazo|mathptmx>
%<cmbase>\ProvidesSymbols{cmbase}[2007/12/19 v0.92]
%<mathpazo>\ProvidesSymbols{mathpazo}[2007/12/19 v0.2]
%<mathptmx>\ProvidesSymbols{mathptmx}[2007/12/19 v0.2]
\@xp\xdef\csname mg@OT1\endcsname{\hexnumber@\symoperators}
\@xp\xdef\csname mg@OML\endcsname{\hexnumber@\symletters}
\@xp\xdef\csname mg@OMS\endcsname{\hexnumber@\symsymbols}
\@xp\xdef\csname mg@OMX\endcsname{\hexnumber@\symlargesymbols}
\gdef\mg@bin{\mg@OMS}
\gdef\mg@del{\mg@OMX}
\xdef\mg@digit{\@xp\@nx\csname mg@OT1\endcsname}
\gdef\mg@latin{\mg@OML}
\global\let\mg@Latin\mg@latin
\global\let\mg@greek\mg@latin
\global\let\mg@Greek\mg@digit
\global\let\mg@rel\mg@bin
\global\let\mg@ord\mg@bin
\global\let\mg@cop\mg@del
%    \end{macrocode}
% 
% 
% Symbols from the 128-character \fn{cmr} encoding.   
% Paren and square bracket delimiters from this encoding are covered
% by the definitions in the \fn{cmex} section, however.   
%    \begin{macrocode}
\DeclareFlexSymbol{!}     {Pun}{OT1}{21}
\DeclareFlexSymbol{+}     {Bin}{OT1}{2B}
\DeclareFlexSymbol{:}     {Rel}{OT1}{3A}
\DeclareFlexSymbol{\colon}{Pun}{OT1}{3A}
\DeclareFlexSymbol{;}     {Pun}{OT1}{3B}
\DeclareFlexSymbol{=}     {Rel}{OT1}{3D}
\DeclareFlexSymbol{?}     {Pun}{OT1}{3F}
%    \end{macrocode}
% \AmS\TeX, and therefore the \pkg{amsmath} package, make the
% uppercase Greek letters class 0 (nonvariable) instead of 7
% (variable), to eliminate the glaring inconsistency with lowercase
% Greek.    (In plain \TeX , \verb"{\bf\Delta}" works, while
% \verb"{\bf\delta}" doesn't.   ) Let us try to make them both
% variable (fonts permitting) instead of nonvariable.   
%    \begin{macrocode}
\DeclareFlexSymbol{\Gamma}  {Var}{Greek}{00}
\DeclareFlexSymbol{\Delta}  {Var}{Greek}{01}
\DeclareFlexSymbol{\Theta}  {Var}{Greek}{02}
\DeclareFlexSymbol{\Lambda} {Var}{Greek}{03}
\DeclareFlexSymbol{\Xi}     {Var}{Greek}{04}
\DeclareFlexSymbol{\Pi}     {Var}{Greek}{05}
\DeclareFlexSymbol{\Sigma}  {Var}{Greek}{06}
\DeclareFlexSymbol{\Upsilon}{Var}{Greek}{07}
\DeclareFlexSymbol{\Phi}    {Var}{Greek}{08}
\DeclareFlexSymbol{\Psi}    {Var}{Greek}{09}
\DeclareFlexSymbol{\Omega}  {Var}{Greek}{0A}
%    \end{macrocode}
% Decimal digits.   
%    \begin{macrocode}
\DeclareFlexSymbol{0}{Var}{digit}{30}
\DeclareFlexSymbol{1}{Var}{digit}{31}
\DeclareFlexSymbol{2}{Var}{digit}{32}
\DeclareFlexSymbol{3}{Var}{digit}{33}
\DeclareFlexSymbol{4}{Var}{digit}{34}
\DeclareFlexSymbol{5}{Var}{digit}{35}
\DeclareFlexSymbol{6}{Var}{digit}{36}
\DeclareFlexSymbol{7}{Var}{digit}{37}
\DeclareFlexSymbol{8}{Var}{digit}{38}
\DeclareFlexSymbol{9}{Var}{digit}{39}
%    \end{macrocode}
% Symbols from the 128-character \fn{cmmi} encoding.   
%    \begin{macrocode}
\DeclareFlexSymbol{,}{Pun}{OML}{3B}
\DeclareFlexSymbol{.}{Ord}{OML}{3A}
\DeclareFlexSymbol{/}{Ord}{OML}{3D}
\DeclareFlexSymbol{<}{Rel}{OML}{3C}
\DeclareFlexSymbol{>}{Rel}{OML}{3E}
%    \end{macrocode}
% To do: make the Var property of lc Greek work properly.   
%    \begin{macrocode}
\DeclareFlexSymbol{\alpha}{Var}{greek}{0B}
\DeclareFlexSymbol{\beta}{Var}{greek}{0C}
\DeclareFlexSymbol{\gamma}{Var}{greek}{0D}
\DeclareFlexSymbol{\delta}{Var}{greek}{0E}
\DeclareFlexSymbol{\epsilon}{Var}{greek}{0F}
\DeclareFlexSymbol{\zeta}{Var}{greek}{10}
\DeclareFlexSymbol{\eta}{Var}{greek}{11}
\DeclareFlexSymbol{\theta}{Var}{greek}{12}
\DeclareFlexSymbol{\iota}{Var}{greek}{13}
\DeclareFlexSymbol{\kappa}{Var}{greek}{14}
\DeclareFlexSymbol{\lambda}{Var}{greek}{15}
\DeclareFlexSymbol{\mu}{Var}{greek}{16}
\DeclareFlexSymbol{\nu}{Var}{greek}{17}
\DeclareFlexSymbol{\xi}{Var}{greek}{18}
\DeclareFlexSymbol{\pi}{Var}{greek}{19}
\DeclareFlexSymbol{\rho}{Var}{greek}{1A}
\DeclareFlexSymbol{\sigma}{Var}{greek}{1B}
\DeclareFlexSymbol{\tau}{Var}{greek}{1C}
\DeclareFlexSymbol{\upsilon}{Var}{greek}{1D}
\DeclareFlexSymbol{\phi}{Var}{greek}{1E}
\DeclareFlexSymbol{\chi}{Var}{greek}{1F}
\DeclareFlexSymbol{\psi}{Var}{greek}{20}
\DeclareFlexSymbol{\omega}{Var}{greek}{21}
\DeclareFlexSymbol{\varepsilon}{Var}{greek}{22}
\DeclareFlexSymbol{\vartheta}{Var}{greek}{23}
\DeclareFlexSymbol{\varpi}{Var}{greek}{24}
\DeclareFlexSymbol{\varrho}{Var}{greek}{25}
\DeclareFlexSymbol{\varsigma}{Var}{greek}{26}
\DeclareFlexSymbol{\varphi}{Var}{greek}{27}
%    \end{macrocode}
% Note that in plain \TeX\  \cs{imath} and \cs{jmath} are
% not variable-font.    But if a \verb"j" changes font to, let's
% say, sans serif or calligraphic, a dotless \verb"j" in the same
% context should change font in the same way.   
%    \begin{macrocode}
\DeclareFlexSymbol{\imath}{Var}{OML}{7B}
\DeclareFlexSymbol{\jmath}{Var}{OML}{7C}
\DeclareFlexSymbol{\ell}{Ord}{OML}{60}
\DeclareFlexSymbol{\wp}{Ord}{OML}{7D}
\DeclareFlexSymbol{\partial}{Ord}{OML}{40}
\DeclareFlexSymbol{\flat}{Ord}{OML}{5B}
\DeclareFlexSymbol{\natural}{Ord}{OML}{5C}
\DeclareFlexSymbol{\sharp}{Ord}{OML}{5D}
\DeclareFlexSymbol{\triangleleft}{Bin}{OML}{2F}
\DeclareFlexSymbol{\triangleright}{Bin}{OML}{2E}
\DeclareFlexSymbol{\star}{Bin}{OML}{3F}
\DeclareFlexSymbol{\smile}{Rel}{OML}{5E}
\DeclareFlexSymbol{\frown}{Rel}{OML}{5F}
\DeclareFlexSymbol{\leftharpoonup}{Rel}{OML}{28}
\DeclareFlexSymbol{\leftharpoondown}{Rel}{OML}{29}
\DeclareFlexSymbol{\rightharpoonup}{Rel}{OML}{2A}
\DeclareFlexSymbol{\rightharpoondown}{Rel}{OML}{2B}
\DeclareFlexSymbol{a}{Var}{latin}{61}
\DeclareFlexSymbol{b}{Var}{latin}{62}
\DeclareFlexSymbol{c}{Var}{latin}{63}
\DeclareFlexSymbol{d}{Var}{latin}{64}
\DeclareFlexSymbol{e}{Var}{latin}{65}
\DeclareFlexSymbol{f}{Var}{latin}{66}
\DeclareFlexSymbol{g}{Var}{latin}{67}
\DeclareFlexSymbol{h}{Var}{latin}{68}
\DeclareFlexSymbol{i}{Var}{latin}{69}
\DeclareFlexSymbol{j}{Var}{latin}{6A}
\DeclareFlexSymbol{k}{Var}{latin}{6B}
\DeclareFlexSymbol{l}{Var}{latin}{6C}
\DeclareFlexSymbol{m}{Var}{latin}{6D}
\DeclareFlexSymbol{n}{Var}{latin}{6E}
\DeclareFlexSymbol{o}{Var}{latin}{6F}
\DeclareFlexSymbol{p}{Var}{latin}{70}
\DeclareFlexSymbol{q}{Var}{latin}{71}
\DeclareFlexSymbol{r}{Var}{latin}{72}
\DeclareFlexSymbol{s}{Var}{latin}{73}
\DeclareFlexSymbol{t}{Var}{latin}{74}
\DeclareFlexSymbol{u}{Var}{latin}{75}
\DeclareFlexSymbol{v}{Var}{latin}{76}
\DeclareFlexSymbol{w}{Var}{latin}{77}
\DeclareFlexSymbol{x}{Var}{latin}{78}
\DeclareFlexSymbol{y}{Var}{latin}{79}
\DeclareFlexSymbol{z}{Var}{latin}{7A}
\DeclareFlexSymbol{A}{Var}{Latin}{41}
\DeclareFlexSymbol{B}{Var}{Latin}{42}
\DeclareFlexSymbol{C}{Var}{Latin}{43}
\DeclareFlexSymbol{D}{Var}{Latin}{44}
\DeclareFlexSymbol{E}{Var}{Latin}{45}
\DeclareFlexSymbol{F}{Var}{Latin}{46}
\DeclareFlexSymbol{G}{Var}{Latin}{47}
\DeclareFlexSymbol{H}{Var}{Latin}{48}
\DeclareFlexSymbol{I}{Var}{Latin}{49}
\DeclareFlexSymbol{J}{Var}{Latin}{4A}
\DeclareFlexSymbol{K}{Var}{Latin}{4B}
\DeclareFlexSymbol{L}{Var}{Latin}{4C}
\DeclareFlexSymbol{M}{Var}{Latin}{4D}
\DeclareFlexSymbol{N}{Var}{Latin}{4E}
\DeclareFlexSymbol{O}{Var}{Latin}{4F}
\DeclareFlexSymbol{P}{Var}{Latin}{50}
\DeclareFlexSymbol{Q}{Var}{Latin}{51}
\DeclareFlexSymbol{R}{Var}{Latin}{52}
\DeclareFlexSymbol{S}{Var}{Latin}{53}
\DeclareFlexSymbol{T}{Var}{Latin}{54}
\DeclareFlexSymbol{U}{Var}{Latin}{55}
\DeclareFlexSymbol{V}{Var}{Latin}{56}
\DeclareFlexSymbol{W}{Var}{Latin}{57}
\DeclareFlexSymbol{X}{Var}{Latin}{58}
\DeclareFlexSymbol{Y}{Var}{Latin}{59}
\DeclareFlexSymbol{Z}{Var}{Latin}{5A}
%    \end{macrocode}
% The \cs{ldotPun} glyph is used in constructing the
% \cs{ldots} symbol.    It is just a period with a different math
% symbol class.    \cs{lhookRel} and \cs{rhookRel} are used
% in a similar way for building hooked arrow symbols.   
%    \begin{macrocode}
\DeclareFlexSymbol{\ldotPun}{Pun}{OML}{3A}
\def\ldotp{\ldotPun}
\DeclareFlexSymbol{\lhookRel}{Rel}{OML}{2C}
\DeclareFlexSymbol{\rhookRel}{Rel}{OML}{2D}
%    \end{macrocode}
% Symbols from the 128-character \fn{cmsy} encoding.   
%    \begin{macrocode}
\DeclareFlexSymbol{*}{Bin}{bin}{03} % \ast
\DeclareFlexSymbol{-}{Bin}{bin}{00}
\DeclareFlexSymbol{|}{Ord}{OMS}{6A}
\DeclareFlexSymbol{\aleph}{Ord}{ord}{40}
\DeclareFlexSymbol{\Re}{Ord}{ord}{3C}
\DeclareFlexSymbol{\Im}{Ord}{ord}{3D}
\DeclareFlexSymbol{\infty}{Ord}{ord}{31}
\DeclareFlexSymbol{\prime}{Ord}{ord}{30}
\DeclareFlexSymbol{\emptyset}{Ord}{ord}{3B}
\DeclareFlexSymbol{\nabla}{Ord}{ord}{72}
\DeclareFlexSymbol{\top}{Ord}{ord}{3E}
\DeclareFlexSymbol{\bot}{Ord}{ord}{3F}
\DeclareFlexSymbol{\triangle}{Ord}{ord}{34}
\DeclareFlexSymbol{\forall}{Ord}{ord}{38}
\DeclareFlexSymbol{\exists}{Ord}{ord}{39}
\DeclareFlexSymbol{\neg}{Ord}{ord}{3A}
\DeclareFlexSymbol{\clubsuit}{Ord}{ord}{7C}
\DeclareFlexSymbol{\diamondsuit}{Ord}{ord}{7D}
\DeclareFlexSymbol{\heartsuit}{Ord}{ord}{7E}
\DeclareFlexSymbol{\spadesuit}{Ord}{ord}{7F}
\DeclareFlexSymbol{\smallint}{COs}{OMS}{73}
%    \end{macrocode}
% Binary operators.   
%    \begin{macrocode}
\DeclareFlexSymbol{\bigtriangleup}{Bin}{bin}{34}
\DeclareFlexSymbol{\bigtriangledown}{Bin}{bin}{35}
\DeclareFlexSymbol{\wedge}{Bin}{bin}{5E}
\DeclareFlexSymbol{\vee}{Bin}{bin}{5F}
\DeclareFlexSymbol{\cap}{Bin}{bin}{5C}
\DeclareFlexSymbol{\cup}{Bin}{bin}{5B}
\DeclareFlexSymbol{\ddagger}{Bin}{bin}{7A}
\DeclareFlexSymbol{\dagger}{Bin}{bin}{79}
\DeclareFlexSymbol{\sqcap}{Bin}{bin}{75}
\DeclareFlexSymbol{\sqcup}{Bin}{bin}{74}
\DeclareFlexSymbol{\uplus}{Bin}{bin}{5D}
\DeclareFlexSymbol{\amalg}{Bin}{bin}{71}
\DeclareFlexSymbol{\diamond}{Bin}{bin}{05}
\DeclareFlexSymbol{\bullet}{Bin}{bin}{0F}
\DeclareFlexSymbol{\wr}{Bin}{bin}{6F}
\DeclareFlexSymbol{\div}{Bin}{bin}{04}
\DeclareFlexSymbol{\odot}{Bin}{bin}{0C}
\DeclareFlexSymbol{\oslash}{Bin}{bin}{0B}
\DeclareFlexSymbol{\otimes}{Bin}{bin}{0A}
\DeclareFlexSymbol{\ominus}{Bin}{bin}{09}
\DeclareFlexSymbol{\oplus}{Bin}{bin}{08}
\DeclareFlexSymbol{\mp}{Bin}{bin}{07}
\DeclareFlexSymbol{\pm}{Bin}{bin}{06}
\DeclareFlexSymbol{\circ}{Bin}{bin}{0E}
\DeclareFlexSymbol{\bigcirc}{Bin}{bin}{0D}
\DeclareFlexSymbol{\setminus}{Bin}{bin}{6E}
\DeclareFlexSymbol{\cdot}{Bin}{bin}{01}
\DeclareFlexSymbol{\ast}{Bin}{bin}{03}
\DeclareFlexSymbol{\times}{Bin}{bin}{02}
%    \end{macrocode}
% Relation symbols.   
%    \begin{macrocode}
\DeclareFlexSymbol{\propto}{Rel}{rel}{2F}
\DeclareFlexSymbol{\sqsubseteq}{Rel}{rel}{76}
\DeclareFlexSymbol{\sqsupseteq}{Rel}{rel}{77}
\DeclareFlexSymbol{\parallel}{Rel}{rel}{6B}
\DeclareFlexSymbol{\mid}{Rel}{rel}{6A}
\DeclareFlexSymbol{\dashv}{Rel}{rel}{61}
\DeclareFlexSymbol{\vdash}{Rel}{rel}{60}
\DeclareFlexSymbol{\nearrow}{Rel}{rel}{25}
\DeclareFlexSymbol{\searrow}{Rel}{rel}{26}
\DeclareFlexSymbol{\nwarrow}{Rel}{rel}{2D}
\DeclareFlexSymbol{\swarrow}{Rel}{rel}{2E}
\DeclareFlexSymbol{\Leftrightarrow}{Rel}{rel}{2C}
\DeclareFlexSymbol{\Leftarrow}{Rel}{rel}{28}
\DeclareFlexSymbol{\Rightarrow}{Rel}{rel}{29}
\DeclareFlexSymbol{\leq}{Rel}{rel}{14}
\DeclareFlexSymbol{\geq}{Rel}{rel}{15}
\DeclareFlexSymbol{\succ}{Rel}{rel}{1F}
\DeclareFlexSymbol{\prec}{Rel}{rel}{1E}
\DeclareFlexSymbol{\approx}{Rel}{rel}{19}
\DeclareFlexSymbol{\succeq}{Rel}{rel}{17}
\DeclareFlexSymbol{\preceq}{Rel}{rel}{16}
\DeclareFlexSymbol{\supset}{Rel}{rel}{1B}
\DeclareFlexSymbol{\subset}{Rel}{rel}{1A}
\DeclareFlexSymbol{\supseteq}{Rel}{rel}{13}
\DeclareFlexSymbol{\subseteq}{Rel}{rel}{12}
\DeclareFlexSymbol{\in}{Rel}{rel}{32}
\DeclareFlexSymbol{\ni}{Rel}{rel}{33}
\DeclareFlexSymbol{\gg}{Rel}{rel}{1D}
\DeclareFlexSymbol{\ll}{Rel}{rel}{1C}
\DeclareFlexSymbol{\leftrightarrow}{Rel}{rel}{24}
\DeclareFlexSymbol{\leftarrow}{Rel}{rel}{20}
\DeclareFlexSymbol{\rightarrow}{Rel}{rel}{21}
\DeclareFlexSymbol{\sim}{Rel}{rel}{18}
\DeclareFlexSymbol{\simeq}{Rel}{rel}{27}
\DeclareFlexSymbol{\perp}{Rel}{rel}{3F}
\DeclareFlexSymbol{\equiv}{Rel}{rel}{11}
\DeclareFlexSymbol{\asymp}{Rel}{rel}{10}
%    \end{macrocode}
% The \cs{notRel} glyph is a special zero-width glyph intended only
% for use in constructing negated symbols.    \cs{mapstoRel} and
% \cs{cdotPun} have similar but more restricted applications.   
%    \begin{macrocode}
\DeclareFlexSymbol{\notRel}{Rel}{rel}{36}
\DeclareFlexSymbol{\mapstoOrd}{Ord}{OMS}{37}
\DeclareFlexSymbol{\cdotOrd}{Ord}{OMS}{01}
\def\cdotp{\mathpunct{\cdotOrd}}
%    \end{macrocode}
% Symbols from the 128-character \fn{cmex} encoding.   
% \verb"COs" stands for `cumulative operator
% (sum-like)'.   
% \verb"COi" stands for `cumulative operator
% (integral-like)'.    These typically differ only in the
% default placement of limits.    \verb"cop" stands for
% `cumulative operator math group'.   
%    \begin{macrocode}
\DeclareFlexSymbol{\coprod}{COs}{cop}{60}
\DeclareFlexSymbol{\bigvee}{COs}{cop}{57}
\DeclareFlexSymbol{\bigwedge}{COs}{cop}{56}
\DeclareFlexSymbol{\biguplus}{COs}{cop}{55}
\DeclareFlexSymbol{\bigcap}{COs}{cop}{54}
\DeclareFlexSymbol{\bigcup}{COs}{cop}{53}
\DeclareFlexSymbol{\int}{COi}{cop}{52}
\DeclareFlexSymbol{\prod}{COs}{cop}{51}
\DeclareFlexSymbol{\sum}{COs}{cop}{50}
\DeclareFlexSymbol{\bigotimes}{COs}{cop}{4E}
\DeclareFlexSymbol{\bigoplus}{COs}{cop}{4C}
\DeclareFlexSymbol{\bigodot}{COs}{cop}{4A}
\DeclareFlexSymbol{\oint}{COi}{cop}{48}
\DeclareFlexSymbol{\bigsqcup}{COs}{cop}{46}
%    \end{macrocode}
% Delimiter symbols.   
% \verb"DeL" stands for `delimiter (left)'.   
% \verb"DeR" stands for `delimiter (right)'.   
% \verb"DeB" stands for `delimiter (bidirectional)'.   
% The principal encoding point for an extensible delimiter is the
% first link in the list of linked sizes as specified in the font metric
% information.   
% For a math encoding such as OT1/OML/OMS/OMX where not all sizes of a
% given delimiter reside in a given font, the extra encoding point for the
% smallest delimiter must be supplied by defining
% \begin{verbatim}
% \sd@GXX
% \end{verbatim}
% where G is the mathgroup and XX is the hexadecimal glyph position.   
%    \begin{macrocode}
\DeclareFlexSymbol{\rangle}{DeR}{del}{0B}
\DeclareFlexSymbol{\langle}{DeL}{del}{0A}
\DeclareFlexSymbol{\rbrace}{DeR}{del}{09}
\DeclareFlexSymbol{\lbrace}{DeL}{del}{08}
\DeclareFlexSymbol{\rceil}{DeR}{del}{07}
\DeclareFlexSymbol{\lceil}{DeL}{del}{06}
\DeclareFlexSymbol{\rfloor}{DeR}{del}{05}
\DeclareFlexSymbol{\lfloor}{DeL}{del}{04}
\DeclareFlexSymbol{(}{DeL}{del}{00}
\DeclareFlexSymbol{)}{DeR}{del}{01}
\DeclareFlexSymbol{[}{DeL}{del}{02}
\DeclareFlexSymbol{]}{DeR}{del}{03}
\DeclareFlexSymbol{\lVert}{DeL}{del}{0D}
\DeclareFlexSymbol{\rVert}{DeR}{del}{0D}
\DeclareFlexSymbol{\lvert}{DeL}{del}{0C}
\DeclareFlexSymbol{\rvert}{DeR}{del}{0C}
\DeclareFlexSymbol{\Vert}{DeB}{del}{0D}
\DeclareFlexSymbol{\vert}{DeB}{del}{0C}
%    \end{macrocode}
% Maybe make the vert bars mathord instead of delimiter, to discourage
% poor usage.   
%    \begin{macrocode}
\DeclareFlexSymbol{|}{DeB}{del}{0C}
\DeclareFlexSymbol{/}{DeB}{del}{0E}
%    \end{macrocode}
% 
% 
% These wacky delimiters need to be supported I guess for
% compabitility reasons.   
% The DeA delimiter type is a special case used only for these
% arrows.   
%    \begin{macrocode}
\DeclareFlexSymbol{\lmoustache}{DeL}{del}{40}
\DeclareFlexSymbol{\rmoustache}{DeR}{del}{41}
\DeclareFlexSymbol{\lgroup}{DeL}{del}{3A}
\DeclareFlexSymbol{\rgroup}{DeR}{del}{3B}
\DeclareFlexSymbol{\bracevert}{DeB}{del}{3E}
\DeclareFlexSymbol{\arrowvert}{DeB}{del}{3C}
\DeclareFlexSymbol{\Arrowvert}{DeB}{del}{3D}
\DeclareFlexSymbol{\uparrow}{DeA}{del}{78}
\DeclareFlexSymbol{\downarrow}{DeA}{del}{79}
\DeclareFlexSymbol{\updownarrow}{DeA}{del}{3F}
\DeclareFlexSymbol{\Uparrow}{DeA}{del}{7E}
\DeclareFlexSymbol{\Downarrow}{DeA}{del}{7F}
\DeclareFlexSymbol{\Updownarrow}{DeA}{del}{77}
\DeclareFlexSymbol{\backslash}{DeB}{del}{0F}
%    \end{macrocode}
% 
% 
% 
% 
% \section{Some compound symbols}
% The following symbols are not robust in standard \LaTeX\ 
% because they use \verb"#" or \cs{mathpalette} (which is not
% robust and contains a \verb"#" in its expansion): \cs{angle},
% \cs{cong}, \cs{notin}, \cs{rightleftharpoons}.   
% 
% In this definition of \cs{hbar}, the symbol is cobbled together
% from a math italic h and the cmr overbar accent glyph.   
%    \begin{macrocode}
\DeclareFlexSymbol{\hbarOrd}{Ord}{OT1}{16}
\DeclareFlexCompoundSymbol{\hbar}{Ord}{\hbarOrd\mkern-9mu h}
%    \end{macrocode}
% For \cs{surd}, the interior symbol gets math class 1
% (cumulative operator) to make the glyph vertically centered on the
% math axis, but the desired horizontal spacing is the spacing for a
% mathord.    (Couldn't it just be class mathopen, though?   )
%    \begin{macrocode}
\DeclareFlexSymbol{\surdOrd}{Ord}{OMS}{70}
\DeclareFlexCompoundSymbol{\surd}{Ord}{\mathop{\surdOrd}}
%    \end{macrocode}
% As shown in this definition of \cs{angle}, rule dimens are not
% allowed to use math-units, unfortunately.   
%    \begin{macrocode}
\DeclareFlexCompoundSymbol{\angle}{Ord}{%
  \vbox{\ialign{%
      $\m@th\scriptstyle##$\crcr
      \notRel\mathrel{\mkern14mu}\crcr
      \noalign{\nointerlineskip}%
      \mkern2.5mu\leaders\hrule \@height.34pt\hfill\mkern2.5mu\crcr
  }}%
}
%    \end{macrocode}
% The \cs{not} function, which is defined in the \pkg{flexisym}
% package, requires a suitably defined \cs{notRel} symbol.   
%    \begin{macrocode}
\DeclareFlexCompoundSymbol{\neq}{Rel}{\not{=}}
%    \end{macrocode}
% .   
%    \begin{macrocode}
\DeclareFlexCompoundSymbol{\mapsto}{Rel}{\mapstoOrd\rightarrow}
%    \end{macrocode}
% The \cs{@vereq} function ends by centering the whole
% construction on the math axis, unlike \cs{buildrel} where the base
% symbol remains at its normal altitude.    Furthermore,
% \cs{@vereq} leaves the math style of the top symbol as given
% instead of downsizing to scriptstyle.   
%    \begin{macrocode}
\DeclareFlexCompoundSymbol{\cong}{Rel}{\mathpalette\@vereq\sim}
%    \end{macrocode}
% The \cs{m@th} in the \fn{fontmath.ltx} definition of
% \cs{notin} is superfluous unless \cs{c@ncel} doesn't include
% it (which was perhaps true in an older version of
% \fn{plain.tex}?).   
%    \begin{macrocode}
\providecommand*\joinord{}
%<cmbase|mathptmx>\renewcommand*\joinord{\mkern-3mu }
%<mathpazo>\renewcommand*\joinord{\mkern-3.45mu }
\DeclareFlexCompoundSymbol{\notin}{Rel}{\mathpalette\c@ncel\in}
\DeclareFlexCompoundSymbol{\rightleftharpoons}{Rel}{\mathpalette\rlh@{}}
\DeclareFlexCompoundSymbol{\doteq}{Rel}{\buildrel\textstyle.\over=}
\DeclareFlexCompoundSymbol{\hookrightarrow}{Rel}{\lhookRel\joinord\rightarrow}
\DeclareFlexCompoundSymbol{\hookleftarrow}{Rel}{\leftarrow\joinord\rhookRel}
\DeclareFlexCompoundSymbol{\bowtie}{Rel}{\triangleright\joinord\triangleleft}
\DeclareFlexCompoundSymbol{\models}{Rel}{\vert\joinord=}
\DeclareFlexCompoundSymbol{\Longrightarrow}{Rel}{\Relbar\joinord\Rightarrow}
\DeclareFlexCompoundSymbol{\longrightarrow}{Rel}{\relbar\joinord\rightarrow}
\DeclareFlexCompoundSymbol{\Longleftarrow}{Rel}{\Leftarrow\joinord\Relbar}
\DeclareFlexCompoundSymbol{\longleftarrow}{Rel}{\leftarrow\joinord\relbar}
\DeclareFlexCompoundSymbol{\longmapsto}{Rel}{\mapstochar\longrightarrow}
\DeclareFlexCompoundSymbol{\longleftrightarrow}{Rel}{\leftarrow\joinord\rightarrow}
\DeclareFlexCompoundSymbol{\Longleftrightarrow}{Rel}{\Leftarrow\joinord\Rightarrow}
%    \end{macrocode}
% Here is what you get from the old definition of \cs{iff}.   
% \begin{verbatim}
% \glue 2.77771 plus 2.77771
% \glue(\thickmuskip) 2.77771 plus 2.77771
% \OMS/cmsy/m/n/10 (
% \hbox(0.0+0.0)x-1.66663
% .\kern -1.66663
% \OMS/cmsy/m/n/10 )
% \penalty 500
% \glue 2.77771 plus 2.77771
% \glue(\thickmuskip) 2.77771 plus 2.77771
% \end{verbatim}
% Looks like it could be simplified slightly.    But it's not so
% easy as it looks to do it without screwing up the line breaking
% possibilities.   
%    \begin{macrocode}
\renewcommand*\iff{%
  \mskip\thickmuskip\Longleftrightarrow\mskip\thickmuskip
}
%    \end{macrocode}
% Some dotly symbols.   
%    \begin{macrocode}
\DeclareFlexCompoundSymbol{\cdots}{Inn}{\cdotp\cdotp\cdotp}%
\DeclareFlexCompoundSymbol{\vdots}{Ord}{%
  \vbox{\baselineskip4\p@ \lineskiplimit\z@
    \kern6\p@\hbox{.}\hbox{.}\hbox{.}}}
\DeclareFlexCompoundSymbol{\ddots}{Inn}{%
  \mkern1mu\raise7\p@
  \vbox{\kern7\p@\hbox{.}}\mkern2mu%
  \raise4\p@\hbox{.}\mkern2mu\raise\p@\hbox{.}\mkern1mu%
}
%    \end{macrocode}
% .   
%    \begin{macrocode}
\def\relbar{\begingroup \def\smash@{tb}% in case amsmath is loaded
    \mathpalette\mathsm@sh{\mathchar"200 }\endgroup}
%    \end{macrocode}
% For \cs{Relbar} we take an equal sign of class $0$ (Ord) from the
% operator family. For \fn{cmr} and \pkg{mathptmx} we know this is
% family $0$.
%    \begin{macrocode}
%<cmbase|mathptmx>\def\Relbar{\mathchar"3D }
%    \end{macrocode}
% For the \pkg{mathpazo} setup we need to use the equal sign from
% \fn{cmr} and so must insert class $0$ and use the symbol from the
% upright symbols.
%    \begin{macrocode}
%<mathpazo>\edef\Relbar{\mathchar\string"\hexnumber@\symupright3D }
%    \end{macrocode}
% Done.
%    \begin{macrocode}
%</cmbase|mathpazo|mathptmx>
%    \end{macrocode}
% Various synonyms such as \cs{le} for \cs{leq} and
% \cs{to} for \cs{rightarrow} are defined in
% \pkg{flexisym} with \cs{def} instead of \cs{let}, for
% slower execution speed but smaller chance of synchronization
% problems.   
%
%
%
%    \begin{macrocode}
%<*msabm>
\ProvidesSymbols{msabm}[2001/09/08 v0.91]
%    \end{macrocode}
%    \begin{macrocode}
\RequirePackage{amsfonts}\relax
%    \end{macrocode}
%    \begin{macrocode}
\@xp\xdef\csname mg@MSA\endcsname{\hexnumber@\symAMSa}%
\@xp\xdef\csname mg@MSB\endcsname{\hexnumber@\symAMSb}%
%    \end{macrocode}
%    \begin{macrocode}
\DeclareFlexSymbol{\boxdot}       {Bin}{MSA}{00}
\DeclareFlexSymbol{\boxplus}      {Bin}{MSA}{01}
\DeclareFlexSymbol{\boxtimes}     {Bin}{MSA}{02}
\DeclareFlexSymbol{\square}       {Ord}{MSA}{03}
\DeclareFlexSymbol{\blacksquare}  {Ord}{MSA}{04}
\DeclareFlexSymbol{\centerdot}    {Bin}{MSA}{05}
\DeclareFlexSymbol{\lozenge}      {Ord}{MSA}{06}
\DeclareFlexSymbol{\blacklozenge} {Ord}{MSA}{07}
\DeclareFlexSymbol{\circlearrowright}   {Rel}{MSA}{08}
\DeclareFlexSymbol{\circlearrowleft}    {Rel}{MSA}{09}
%    \end{macrocode}
% In amsfonts.sty:
%    \begin{macrocode}
%%\DeclareFlexSymbol{\rightleftharpoons}{Rel}{MSA}{0A}
\DeclareFlexSymbol{\leftrightharpoons}  {Rel}{MSA}{0B}
\DeclareFlexSymbol{\boxminus}     {Bin}{MSA}{0C}
\DeclareFlexSymbol{\Vdash}        {Rel}{MSA}{0D}
\DeclareFlexSymbol{\Vvdash}       {Rel}{MSA}{0E}
\DeclareFlexSymbol{\vDash}        {Rel}{MSA}{0F}
\DeclareFlexSymbol{\twoheadrightarrow}  {Rel}{MSA}{10}
\DeclareFlexSymbol{\twoheadleftarrow}   {Rel}{MSA}{11}
\DeclareFlexSymbol{\leftleftarrows}     {Rel}{MSA}{12}
\DeclareFlexSymbol{\rightrightarrows}   {Rel}{MSA}{13}
\DeclareFlexSymbol{\upuparrows}         {Rel}{MSA}{14}
\DeclareFlexSymbol{\downdownarrows} {Rel}{MSA}{15}
\DeclareFlexSymbol{\upharpoonright} {Rel}{MSA}{16}
 \let\restriction\upharpoonright
\DeclareFlexSymbol{\downharpoonright}   {Rel}{MSA}{17}
\DeclareFlexSymbol{\upharpoonleft}  {Rel}{MSA}{18}
\DeclareFlexSymbol{\downharpoonleft}{Rel}{MSA}{19}
\DeclareFlexSymbol{\rightarrowtail} {Rel}{MSA}{1A}
\DeclareFlexSymbol{\leftarrowtail}  {Rel}{MSA}{1B}
\DeclareFlexSymbol{\leftrightarrows}{Rel}{MSA}{1C}
\DeclareFlexSymbol{\rightleftarrows}{Rel}{MSA}{1D}
\DeclareFlexSymbol{\Lsh}            {Rel}{MSA}{1E}
\DeclareFlexSymbol{\Rsh}            {Rel}{MSA}{1F}
\DeclareFlexSymbol{\rightsquigarrow}  {Rel}{MSA}{20}
\DeclareFlexSymbol{\leftrightsquigarrow}{Rel}{MSA}{21}
\DeclareFlexSymbol{\looparrowleft}  {Rel}{MSA}{22}
\DeclareFlexSymbol{\looparrowright} {Rel}{MSA}{23}
\DeclareFlexSymbol{\circeq}       {Rel}{MSA}{24}
\DeclareFlexSymbol{\succsim}      {Rel}{MSA}{25}
\DeclareFlexSymbol{\gtrsim}       {Rel}{MSA}{26}
\DeclareFlexSymbol{\gtrapprox}    {Rel}{MSA}{27}
\DeclareFlexSymbol{\multimap}     {Rel}{MSA}{28}
\DeclareFlexSymbol{\therefore}    {Rel}{MSA}{29}
\DeclareFlexSymbol{\because}      {Rel}{MSA}{2A}
\DeclareFlexSymbol{\doteqdot}     {Rel}{MSA}{2B}
 \let\Doteq\doteqdot
\DeclareFlexSymbol{\triangleq}    {Rel}{MSA}{2C}
\DeclareFlexSymbol{\precsim}      {Rel}{MSA}{2D}
\DeclareFlexSymbol{\lesssim}      {Rel}{MSA}{2E}
\DeclareFlexSymbol{\lessapprox}   {Rel}{MSA}{2F}
\DeclareFlexSymbol{\eqslantless}  {Rel}{MSA}{30}
\DeclareFlexSymbol{\eqslantgtr}   {Rel}{MSA}{31}
\DeclareFlexSymbol{\curlyeqprec}  {Rel}{MSA}{32}
\DeclareFlexSymbol{\curlyeqsucc}  {Rel}{MSA}{33}
\DeclareFlexSymbol{\preccurlyeq}  {Rel}{MSA}{34}
\DeclareFlexSymbol{\leqq}         {Rel}{MSA}{35}
\DeclareFlexSymbol{\leqslant}     {Rel}{MSA}{36}
\DeclareFlexSymbol{\lessgtr}      {Rel}{MSA}{37}
\DeclareFlexSymbol{\backprime}    {Ord}{MSA}{38}
\DeclareFlexSymbol{\risingdotseq} {Rel}{MSA}{3A}
\DeclareFlexSymbol{\fallingdotseq}{Rel}{MSA}{3B}
\DeclareFlexSymbol{\succcurlyeq}  {Rel}{MSA}{3C}
\DeclareFlexSymbol{\geqq}         {Rel}{MSA}{3D}
\DeclareFlexSymbol{\geqslant}     {Rel}{MSA}{3E}
\DeclareFlexSymbol{\gtrless}      {Rel}{MSA}{3F}
%    \end{macrocode}
% in amsfonts.sty
%    \begin{macrocode}
%% \DeclareFlexSymbol{\sqsubset}    {Rel}{MSA}{40}
%% \DeclareFlexSymbol{\sqsupset}    {Rel}{MSA}{41}
\DeclareFlexSymbol{\vartriangleright}{Rel}{MSA}{42}
\DeclareFlexSymbol{\vartriangleleft} {Rel}{MSA}{43}
\DeclareFlexSymbol{\trianglerighteq} {Rel}{MSA}{44}
\DeclareFlexSymbol{\trianglelefteq}  {Rel}{MSA}{45}
\DeclareFlexSymbol{\bigstar}    {Ord}{MSA}{46}
\DeclareFlexSymbol{\between}    {Rel}{MSA}{47}
\DeclareFlexSymbol{\blacktriangledown}  {Ord}{MSA}{48}
\DeclareFlexSymbol{\blacktriangleright} {Rel}{MSA}{49}
\DeclareFlexSymbol{\blacktriangleleft}  {Rel}{MSA}{4A}
\DeclareFlexSymbol{\vartriangle}        {Rel}{MSA}{4D}
\DeclareFlexSymbol{\blacktriangle}      {Ord}{MSA}{4E}
\DeclareFlexSymbol{\triangledown}       {Ord}{MSA}{4F}
\DeclareFlexSymbol{\eqcirc}       {Rel}{MSA}{50}
\DeclareFlexSymbol{\lesseqgtr}    {Rel}{MSA}{51}
\DeclareFlexSymbol{\gtreqless}    {Rel}{MSA}{52}
\DeclareFlexSymbol{\lesseqqgtr}   {Rel}{MSA}{53}
\DeclareFlexSymbol{\gtreqqless}   {Rel}{MSA}{54}
\DeclareFlexSymbol{\Rrightarrow}  {Rel}{MSA}{56}
\DeclareFlexSymbol{\Lleftarrow}   {Rel}{MSA}{57}
\DeclareFlexSymbol{\veebar}       {Bin}{MSA}{59}
\DeclareFlexSymbol{\barwedge}     {Bin}{MSA}{5A}
\DeclareFlexSymbol{\doublebarwedge} {Bin}{MSA}{5B}
%    \end{macrocode}
% In amsfonts.sty
%    \begin{macrocode}
%%\DeclareFlexSymbol{\angle}        {Ord}{MSA}{5C}
\DeclareFlexSymbol{\measuredangle}  {Ord}{MSA}{5D}
\DeclareFlexSymbol{\sphericalangle} {Ord}{MSA}{5E}
\DeclareFlexSymbol{\varpropto}    {Rel}{MSA}{5F}
\DeclareFlexSymbol{\smallsmile}   {Rel}{MSA}{60}
\DeclareFlexSymbol{\smallfrown}   {Rel}{MSA}{61}
\DeclareFlexSymbol{\Subset}       {Rel}{MSA}{62}
\DeclareFlexSymbol{\Supset}       {Rel}{MSA}{63}
\DeclareFlexSymbol{\Cup}          {Bin}{MSA}{64}
 \let\doublecup\Cup
\DeclareFlexSymbol{\Cap}          {Bin}{MSA}{65}
 \let\doublecap\Cap
\DeclareFlexSymbol{\curlywedge}   {Bin}{MSA}{66}
\DeclareFlexSymbol{\curlyvee}     {Bin}{MSA}{67}
\DeclareFlexSymbol{\leftthreetimes} {Bin}{MSA}{68}
\DeclareFlexSymbol{\rightthreetimes}{Bin}{MSA}{69}
\DeclareFlexSymbol{\subseteqq}    {Rel}{MSA}{6A}
\DeclareFlexSymbol{\supseteqq}    {Rel}{MSA}{6B}
\DeclareFlexSymbol{\bumpeq}       {Rel}{MSA}{6C}
\DeclareFlexSymbol{\Bumpeq}       {Rel}{MSA}{6D}
\DeclareFlexSymbol{\lll}          {Rel}{MSA}{6E}
 \let\llless\lll
\DeclareFlexSymbol{\ggg}          {Rel}{MSA}{6F}
 \let\gggtr\ggg
\DeclareFlexSymbol{\circledS}     {Ord}{MSA}{73}
\DeclareFlexSymbol{\pitchfork}    {Rel}{MSA}{74}
\DeclareFlexSymbol{\dotplus}      {Bin}{MSA}{75}
\DeclareFlexSymbol{\backsim}      {Rel}{MSA}{76}
\DeclareFlexSymbol{\backsimeq}    {Rel}{MSA}{77}
\DeclareFlexSymbol{\complement}   {Ord}{MSA}{7B}
\DeclareFlexSymbol{\intercal}     {Bin}{MSA}{7C}
\DeclareFlexSymbol{\circledcirc}  {Bin}{MSA}{7D}
\DeclareFlexSymbol{\circledast}   {Bin}{MSA}{7E}
\DeclareFlexSymbol{\circleddash}  {Bin}{MSA}{7F}
%    \end{macrocode}
%   Begin AMSb declarations
%    \begin{macrocode}
\DeclareFlexSymbol{\lvertneqq}    {Rel}{MSB}{00}
\DeclareFlexSymbol{\gvertneqq}    {Rel}{MSB}{01}
\DeclareFlexSymbol{\nleq}         {Rel}{MSB}{02}
\DeclareFlexSymbol{\ngeq}         {Rel}{MSB}{03}
\DeclareFlexSymbol{\nless}        {Rel}{MSB}{04}
\DeclareFlexSymbol{\ngtr}         {Rel}{MSB}{05}
\DeclareFlexSymbol{\nprec}        {Rel}{MSB}{06}
\DeclareFlexSymbol{\nsucc}        {Rel}{MSB}{07}
\DeclareFlexSymbol{\lneqq}        {Rel}{MSB}{08}
\DeclareFlexSymbol{\gneqq}        {Rel}{MSB}{09}
\DeclareFlexSymbol{\nleqslant}    {Rel}{MSB}{0A}
\DeclareFlexSymbol{\ngeqslant}    {Rel}{MSB}{0B}
\DeclareFlexSymbol{\lneq}         {Rel}{MSB}{0C}
\DeclareFlexSymbol{\gneq}         {Rel}{MSB}{0D}
\DeclareFlexSymbol{\npreceq}      {Rel}{MSB}{0E}
\DeclareFlexSymbol{\nsucceq}      {Rel}{MSB}{0F}
\DeclareFlexSymbol{\precnsim}     {Rel}{MSB}{10}
\DeclareFlexSymbol{\succnsim}     {Rel}{MSB}{11}
\DeclareFlexSymbol{\lnsim}        {Rel}{MSB}{12}
\DeclareFlexSymbol{\gnsim}        {Rel}{MSB}{13}
\DeclareFlexSymbol{\nleqq}        {Rel}{MSB}{14}
\DeclareFlexSymbol{\ngeqq}        {Rel}{MSB}{15}
\DeclareFlexSymbol{\precneqq}     {Rel}{MSB}{16}
\DeclareFlexSymbol{\succneqq}     {Rel}{MSB}{17}
\DeclareFlexSymbol{\precnapprox}  {Rel}{MSB}{18}
\DeclareFlexSymbol{\succnapprox}  {Rel}{MSB}{19}
\DeclareFlexSymbol{\lnapprox}     {Rel}{MSB}{1A}
\DeclareFlexSymbol{\gnapprox}     {Rel}{MSB}{1B}
\DeclareFlexSymbol{\nsim}         {Rel}{MSB}{1C}
\DeclareFlexSymbol{\ncong}        {Rel}{MSB}{1D}
\DeclareFlexSymbol{\diagup}       {Ord}{MSB}{1E}
\DeclareFlexSymbol{\diagdown}     {Ord}{MSB}{1F}
\DeclareFlexSymbol{\varsubsetneq}   {Rel}{MSB}{20}
\DeclareFlexSymbol{\varsupsetneq}   {Rel}{MSB}{21}
\DeclareFlexSymbol{\nsubseteqq}     {Rel}{MSB}{22}
\DeclareFlexSymbol{\nsupseteqq}     {Rel}{MSB}{23}
\DeclareFlexSymbol{\subsetneqq}     {Rel}{MSB}{24}
\DeclareFlexSymbol{\supsetneqq}     {Rel}{MSB}{25}
\DeclareFlexSymbol{\varsubsetneqq}  {Rel}{MSB}{26}
\DeclareFlexSymbol{\varsupsetneqq}  {Rel}{MSB}{27}
\DeclareFlexSymbol{\subsetneq}      {Rel}{MSB}{28}
\DeclareFlexSymbol{\supsetneq}      {Rel}{MSB}{29}
\DeclareFlexSymbol{\nsubseteq}      {Rel}{MSB}{2A}
\DeclareFlexSymbol{\nsupseteq}      {Rel}{MSB}{2B}
\DeclareFlexSymbol{\nparallel}      {Rel}{MSB}{2C}
\DeclareFlexSymbol{\nmid}           {Rel}{MSB}{2D}
\DeclareFlexSymbol{\nshortmid}      {Rel}{MSB}{2E}
\DeclareFlexSymbol{\nshortparallel} {Rel}{MSB}{2F}
\DeclareFlexSymbol{\nvdash}         {Rel}{MSB}{30}
\DeclareFlexSymbol{\nVdash}         {Rel}{MSB}{31}
\DeclareFlexSymbol{\nvDash}         {Rel}{MSB}{32}
\DeclareFlexSymbol{\nVDash}         {Rel}{MSB}{33}
\DeclareFlexSymbol{\ntrianglerighteq}{Rel}{MSB}{34}
\DeclareFlexSymbol{\ntrianglelefteq}{Rel}{MSB}{35}
\DeclareFlexSymbol{\ntriangleleft}  {Rel}{MSB}{36}
\DeclareFlexSymbol{\ntriangleright} {Rel}{MSB}{37}
\DeclareFlexSymbol{\nleftarrow}     {Rel}{MSB}{38}
\DeclareFlexSymbol{\nrightarrow}    {Rel}{MSB}{39}
\DeclareFlexSymbol{\nLeftarrow}     {Rel}{MSB}{3A}
\DeclareFlexSymbol{\nRightarrow}    {Rel}{MSB}{3B}
\DeclareFlexSymbol{\nLeftrightarrow}{Rel}{MSB}{3C}
\DeclareFlexSymbol{\nleftrightarrow}{Rel}{MSB}{3D}
\DeclareFlexSymbol{\divideontimes}  {Bin}{MSB}{3E}
\DeclareFlexSymbol{\varnothing}     {Ord}{MSB}{3F}
\DeclareFlexSymbol{\nexists}        {Ord}{MSB}{40}
\DeclareFlexSymbol{\Finv}           {Ord}{MSB}{60}
\DeclareFlexSymbol{\Game}           {Ord}{MSB}{61}
%    \end{macrocode}
% In amsfonts.sty:
%    \begin{macrocode}
%%\DeclareFlexSymbol{\mho}          {Ord}{MSB}{66}
\DeclareFlexSymbol{\eth}            {Ord}{MSB}{67}
\DeclareFlexSymbol{\eqsim}          {Rel}{MSB}{68}
\DeclareFlexSymbol{\beth}           {Ord}{MSB}{69}
\DeclareFlexSymbol{\gimel}          {Ord}{MSB}{6A}
\DeclareFlexSymbol{\daleth}         {Ord}{MSB}{6B}
\DeclareFlexSymbol{\lessdot}        {Bin}{MSB}{6C}
\DeclareFlexSymbol{\gtrdot}         {Bin}{MSB}{6D}
\DeclareFlexSymbol{\ltimes}         {Bin}{MSB}{6E}
\DeclareFlexSymbol{\rtimes}         {Bin}{MSB}{6F}
\DeclareFlexSymbol{\shortmid}       {Rel}{MSB}{70}
\DeclareFlexSymbol{\shortparallel}  {Rel}{MSB}{71}
\DeclareFlexSymbol{\smallsetminus}  {Bin}{MSB}{72}
\DeclareFlexSymbol{\thicksim}       {Rel}{MSB}{73}
\DeclareFlexSymbol{\thickapprox}    {Rel}{MSB}{74}
\DeclareFlexSymbol{\approxeq}       {Rel}{MSB}{75}
\DeclareFlexSymbol{\succapprox}     {Rel}{MSB}{76}
\DeclareFlexSymbol{\precapprox}     {Rel}{MSB}{77}
\DeclareFlexSymbol{\curvearrowleft} {Rel}{MSB}{78}
\DeclareFlexSymbol{\curvearrowright}{Rel}{MSB}{79}
\DeclareFlexSymbol{\digamma}        {Ord}{MSB}{7A}
\DeclareFlexSymbol{\varkappa}       {Ord}{MSB}{7B}
\DeclareFlexSymbol{\Bbbk}           {Ord}{MSB}{7C}
\DeclareFlexSymbol{\hslash}         {Ord}{MSB}{7D}
%    \end{macrocode}
% In amsfonts.sty:
%    \begin{macrocode}
%%\DeclareFlexSymbol{\hbar}         {Ord}{MSB}{7E}
\DeclareFlexSymbol{\backepsilon}    {Rel}{MSB}{7F}
%</msabm>
%    \end{macrocode}
%
% \PrintIndex
%
% \Finale

%        (quote the arguments according to the demands of your shell)
%
% Documentation:
%    (a) If flexisym.drv is present:
%           latex flexisym.drv
%    (b) Without flexisym.drv:
%           latex flexisym.dtx; ...
%    The class ltxdoc loads the configuration file ltxdoc.cfg
%    if available. Here you can specify further options, e.g.
%    use A4 as paper format:
%       \PassOptionsToClass{a4paper}{article}
%
%    Programm calls to get the documentation (example):
%       pdflatex flexisym.dtx
%       makeindex -s gind.ist flexisym.idx
%       pdflatex flexisym.dtx
%       makeindex -s gind.ist flexisym.idx
%       pdflatex flexisym.dtx
%
% Installation:
%    TDS:tex/latex/mh/flexisym.sty
%    TDS:tex/latex/mh/cmbase.sym
%    TDS:tex/latex/mh/mathpazo.sym
%    TDS:tex/latex/mh/mathptmx.sym
%    TDS:tex/latex/mh/msabm.sym
%    TDS:doc/latex/mh/flexisym.pdf
%    TDS:source/latex/mh/flexisym.dtx
%
%<*ignore>
\begingroup
  \def\x{LaTeX2e}
\expandafter\endgroup
\ifcase 0\ifx\install y1\fi\expandafter
         \ifx\csname processbatchFile\endcsname\relax\else1\fi
         \ifx\fmtname\x\else 1\fi\relax
\else\csname fi\endcsname
%</ignore>
%<*install>
\input docstrip.tex
\Msg{************************************************************************}
\Msg{* Installation}
\Msg{* Package: flexisym 2007/12/19 v0.96 Flexisym (MH)}
\Msg{************************************************************************}

\keepsilent
\askforoverwritefalse

\preamble

This is a generated file.

Copyright (C) 1997-2003 by Michael J. Downes
Copyright (C) 2007 by Morten Hoegholm <mh.ctan@gmail.com>

This work may be distributed and/or modified under the
conditions of the LaTeX Project Public License, either
version 1.3 of this license or (at your option) any later
version. The latest version of this license is in
   http://www.latex-project.org/lppl.txt
and version 1.3 or later is part of all distributions of
LaTeX version 2005/12/01 or later.

This work has the LPPL maintenance status "maintained".

This Current Maintainer of this work is Morten Hoegholm.

This work consists of the main source file flexisym.dtx
and the derived files
   flexisym.sty, flexisym.pdf, flexisym.ins, flexisym.drv,
   cmbase.sym, mathpazo.sym, mathptmx.sym, msabm.sym.

\endpreamble

\generate{%
  \file{flexisym.ins}{\from{flexisym.dtx}{install}}%
  \file{flexisym.drv}{\from{flexisym.dtx}{driver}}%
  \usedir{tex/latex/mh}%
  \file{flexisym.sty}{\from{flexisym.dtx}{package}}%
  \file{cmbase.sym}{\from{flexisym.dtx}{cmbase}}%
  \file{mathpazo.sym}{\from{flexisym.dtx}{mathpazo}}%
  \file{mathptmx.sym}{\from{flexisym.dtx}{mathptmx}}%
  \file{msabm.sym}{\from{flexisym.dtx}{msabm}}%
}

\obeyspaces
\Msg{************************************************************************}
\Msg{*}
\Msg{* To finish the installation you have to move the following}
\Msg{* files into a directory searched by TeX:}
\Msg{*}
\Msg{*     flexisym.sty, cmbase.sym, mathpazo.sym, mathptmx.sym, msabm.sym}
\Msg{*}
\Msg{* To produce the documentation run the file `flexisym.drv'}
\Msg{* through LaTeX.}
\Msg{*}
\Msg{* Happy TeXing!}
\Msg{*}
\Msg{************************************************************************}

\endbatchfile
%</install>
%<*ignore>
\fi
%</ignore>
%<*driver>
\NeedsTeXFormat{LaTeX2e}
\ProvidesFile{flexisym.drv}%
  [2007/12/19 v0.96 flexisym (MH)]
\documentclass{ltxdoc}
\CodelineIndex
\EnableCrossrefs
\setcounter{IndexColumns}{2}
\providecommand*\pkg[1]{\textsf{#1}}
\providecommand*\cls[1]{\textsf{#1}}
\providecommand*\opt[1]{\texttt{#1}}
\providecommand*\env[1]{\texttt{#1}}
\providecommand*\fn[1]{\texttt{#1}}
\makeatletter
\providecommand{\AmS}{{\protect\AmSfont
  A\kern-.1667em\lower.5ex\hbox{M}\kern-.125emS}}
\providecommand{\AmSfont}{%
  \usefont{OMS}{cmsy}{\if\expandafter\@car\f@series\@nil bb\else m\fi}{n}}
\makeatother
\newenvironment{aside}{\begin{quote}\bfseries}{\end{quote}}
\begin{document}
  \DocInput{flexisym.dtx}
\end{document}
%</driver>
% \fi
%
% \title{The \textsf{flexisym} package}
% \date{2007/12/19 v0.96}
% \author{Morten H\o gholm \\\texttt{mh.ctan@gmail.com}}
%
% \maketitle
%
% \part*{User's guide}
%
% For now, the user's guide is in breqn.
%
% \StopEventually{}
% \part*{Implementation}
% 
% \section{flexisym}
%
%    \begin{macrocode}
%<*package>
\ProvidesPackage{flexisym}[2007/12/19 v0.96]
\let\@xp\expandafter \let\@nx\noexpand
\edef\do{%
  \@nx\AtEndOfPackage{%
    \catcode\number`\"=\number\catcode`\"
    \relax
  }%
}
\do \let\do\relax
\catcode`\"=12
\let\@sym\@gobble
\DeclareOption{robust}{%
  \def\@sym#1{%
    \ifx\protect\@typeset@protect \else\protect#1\@xp\@gobblefour\fi
  }%
}
\def\mg@bin{2}% binary operators
\def\mg@rel{2}% relations
%%\def\mg@nre{B}% negated relations
\def\mg@del{3}% delimiters
%%\def\mg@arr{B}% arrows
\def\mg@acc{0}% accents
\def\mg@cop{3}% cumulative operators (sum, int)
\def\mg@latin{1}% (Latin) letters
\def\mg@greek{1}% (lowercase) Greek
\def\mg@Greek{0}% (capital) Greek
%%\def\mg@bflatin{4}% bold upright Latin letters ?
%%\def\mg@Bbb{B}% blackboard bold
\def\mg@cal{2}% script/calligraphic
%%\def\mg@frak{5}% Fraktur letters
\def\mg@digit{0}% decimal digits % 1 = oldstyle, 0 = capital
\expandafter\let\csname MathChar \endcsname\mathchar
\expandafter\let\csname Delimiter \endcsname\delimiter
\expandafter\let\csname Radical \endcsname\radical
\newcommand{\MathChar}{}
\edef\MathChar{\csname MathChar \endcsname\noexpand\string}
\newcommand{\Delimiter}{}
\edef\Delimiter{\csname Delimiter \endcsname\noexpand\string}
\newcommand{\Radical}{}
\edef\Radical{\csname Radical \endcsname\noexpand\string}
\let\sumlimits\displaylimits
\let\intlimits\nolimits
\let\namelimits\displaylimits
\edef\m@Ord#1#2#3{\csname MathChar \endcsname"0#1#2#3 }
\edef\m@Var#1#2#3{\csname MathChar \endcsname"7#1#2#3 }
\edef\m@Bin#1#2#3{\csname MathChar \endcsname"2#1#2#3 }
\edef\m@Rel#1#2#3{\csname MathChar \endcsname"3#1#2#3 }
\edef\m@Pun#1#2#3{\csname MathChar \endcsname"6#1#2#3 }
\edef\m@COs#1#2#3{\csname MathChar \endcsname"1#1#2#3 \sumlimits}
\edef\m@COi#1#2#3{\csname MathChar \endcsname"1#1#2#3 \intlimits}
\def\delim@a#1#2#3#4{\ifx\relax#1#2#3#4\else#1\fi #2#3#4}
\def\delim@b#1#2#3#4{\ifx\relax#1#2#3#4\else#1\fi }
\def\@tempa{%
  \@nx\@xp\@nx\delim@a\@nx\csname sd@##1##2##3\@nx\endcsname ##1##2##3 }
\edef\m@DeL#1#2#3{\csname Delimiter \endcsname"4\@tempa}
\edef\m@DeR#1#2#3{\csname Delimiter \endcsname"5\@tempa}
\edef\m@DeB#1#2#3{\csname Delimiter \endcsname"0\@tempa}
\edef\m@DeA#1#2#3{\csname Delimiter \endcsname"3\@tempa}
\edef\m@Rad#1#2#3{\csname Radical \endcsname"\@tempa}
\def\do#1#2{\@xp\def\csname sd@#1\endcsname{#2}}
\do{300}{028}
\do{301}{029}
\do{302}{05B}
\do{303}{05D}
\do{304}{262}
\do{305}{263}
\do{306}{264}
\do{307}{265}
\do{308}{266}
\do{309}{267}
\do{30A}{268}
\do{30B}{269}
\do{30C}{26A}
\do{30D}{26B}
\do{30E}{13D}
\do{30F}{26E}
\do{340}{37A}
\do{341}{37B}
\do{33A}{33A}
\do{33B}{33B}
\do{33E}{33E}
\do{33C}{26A}
\do{33D}{26B}
\do{378}{222}
\do{379}{223}
\do{33F}{26C}
\do{37E}{22A}
\do{37F}{22B}
\do{377}{26D}
\do{30F}{26E}
\def\m@Acc#1#2#3#4{\mathaccent"#1#2#3{#4}}
\def\@symAcc{\@sym}
\let\@symtype\@firstofone
\def\@symOrd#1#2{\@symtype\mathord{\OrdSymbol{#2}}}
\def\@symVar{\@symOrd}
\def\@symBin#1#2{\@symtype\mathbin{\OrdSymbol{#2}}}
\def\@symRel#1#2{\@symtype\mathrel{\OrdSymbol{#2}}}
\def\@symPun#1#2{\@symtype\mathpunct{\OrdSymbol{#2}}}
\def\@symCOi#1#2{\@symtype{\mathop{\OrdSymbol{#2}}\intlimits}}
\def\@symCOs#1#2{\@symtype{\mathop{\OrdSymbol{#2}}\sumlimits}}
\def\@symOpe#1#2{\@symtype\mathopen{\OrdSymbol{#2}}}
\def\@symClo#1#2{\@symtype\mathclose{\OrdSymbol{#2}}}
\def\@symDeL#1#2{\@symtype\mathopen{\OrdSymbol{#2}}}
\def\@symDeR#1#2{\@symtype\mathclose{\OrdSymbol{#2}}}
\def\@symDeB#1#2{\@symtype\mathord{\OrdSymbol{#2}}}
\def\@symInn#1#2{\@symtype\mathinner{\OrdSymbol{#2}}}
\def\@xnce#1{\@xp\@nx\csname#1\endcsname}
\let\sym@global\global
\def\DeclareFlexSymbol#1#2#3#4{%
  \begingroup
  \edef\@tempb{\@nx\@sym\@nx#1\@xnce{m@#2}\@xnce{mg@#3}#4}%
  \ifcat\@nx#1\relax
    \sym@global\let#1\@tempb
  \else
    \sym@global\mathcode`#1="8000\relax
    \lccode`\~=`#1\relax
    \lowercase{\sym@global\let~\@tempb}%
  \fi
  \endgroup
}
\def\DeclareFlexCompoundSymbol#1#2#3{%
  \@xp\DeclareRobustCommand\@xp#1\@xp{\csname @sym#2\endcsname#1{#3}}%
  \sym@global\let#1#1\relax
}
\DeclareRobustCommand\textchar{\text@char\textfont}
\DeclareRobustCommand\scriptchar{\text@char\scriptfont}%
\def\text@char@a{\?\endgroup}%
\def\text@char@sym#1#2#3{%
  \begingroup
    \let\@sym\relax % defense against infinite loops
    \the\text@script@char#3%
    \afterassignment\text@char@a
    \chardef\?="%
}
\def\text@char#1#2{\begingroup\check@mathfonts
  \let\text@script@char#1\let\@sym\text@char@sym
  \let\@symtype\@secondoftwo \let\OrdSymbol\@firstofone
  \let\ifmmode\iftrue \everymath{$\@gobble}%$
  \def\mkern{\muskip\z@}\let\mskip\mkern
  \ifcat\relax\noexpand#2#2%
  \else
    \lccode`\~=\expandafter`\string#2\relax
    \lowercase{~}%
  \fi
  \endgroup
}
\providecommand\textprime{}
\DeclareRobustCommand\textprime{\leavevmode
  \raise.8ex\hbox{\text@char\scriptfont\prime}%
}
\@ifundefined{resetMathstrut@}{}{%
  \def\resetMathstrut@{%
    \setbox\z@\hbox{\textchar\vert}%
    \ht\Mathstrutbox@\ht\z@ \dp\Mathstrutbox@\dp\z@
  }%
}
\@ifundefined{rightarrowfill@}{}{%
  \def\rightarrowfill@#1{\m@th\setboxz@h{$#1\relbar$}\ht\z@\z@
    $#1\copy\z@\mkern-6mu\cleaders
    \hbox{$#1\mkern-2mu\box\z@\mkern-2mu$}\hfill
    \mkern-6mu\OrdSymbol{\rightarrow}$}
  \def\leftarrowfill@#1{\m@th\setboxz@h{$#1\relbar$}\ht\z@\z@
    $#1\OrdSymbol{\leftarrow}\mkern-6mu\cleaders
    \hbox{$#1\mkern-2mu\copy\z@\mkern-2mu$}\hfill
    \mkern-6mu\box\z@$}
  \def\leftrightarrowfill@#1{\m@th\setboxz@h{$#1\relbar$}\ht\z@\z@
    $#1\OrdSymbol{\leftarrow}\mkern-6mu\cleaders
    \hbox{$#1\mkern-2mu\box\z@\mkern-2mu$}\hfill
    \mkern-6mu\OrdSymbol{\rightarrow}$}
}
\def\binrel@sym#1#2#3#4#5{%
  \xdef\binrel@@##1{%
    \ifx\m@Ord#2\@nx\@symOrd
    \else\ifx\m@Var#2\@nx\@symVar
    \else\ifx\m@COs#2\@nx\@symCOs
    \else\ifx\m@COi#2\@nx\@symCOi
    \else\ifx\m@Bin#2\@nx\@symBin
    \else\ifx\m@Rel#2\@nx\@symRel
    \else\ifx\m@Pun#2\@nx\@symPun
    \else\@nx\@symErr \fi\fi\fi\fi\fi\fi\fi
  ?{\@nx\OrdSymbol{##1}}}%
}
\def\binrel@a{%
  \def\@symOrd##1##2{\gdef\binrel@@####1{\@symOrd##1{\OrdSymbol{####1}}}}%
  \def\@symVar##1##2{\gdef\binrel@@####1{\@symVar##1{\OrdSymbol{####1}}}}%
  \def\@symCOs##1##2{\gdef\binrel@@####1{\@symCOs##1{\OrdSymbol{####1}}}}%
  \def\@symCOi##1##2{\gdef\binrel@@####1{\@symCOi##1{\OrdSymbol{####1}}}}%
  \def\@symBin##1##2{\gdef\binrel@@####1{\@symBin##1{\OrdSymbol{####1}}}}%
  \def\@symRel##1##2{\gdef\binrel@@####1{\@symRel##1{\OrdSymbol{####1}}}}%
  \def\@symPun##1##2{\gdef\binrel@@####1{\@symPun##1{\OrdSymbol{####1}}}}%
}
\def\binrel@#1{%
  \setbox\z@\hbox{$%
    \let\mathchoice\@gobblethree
    \let\@sym\binrel@sym \binrel@a
    #1$}%
}
\def\@symextension{sym}
\newcommand\usesymbols[1]{%
  \@for\@tempb:=#1\do{%
    \@xp\@onefilewithoptions\@xp{\@tempb}[][]\@symextension
  }%
}
\newcommand\ProvidesSymbols[1]{\ProvidesFile{#1.sym}}
\DeclareRobustCommand{\not}[1]{\@symRel\not{\OrdSymbol{\notRel#1}}}
\DeclareRobustCommand{\OrdSymbol}[1]{%
  \begingroup\mathchars@reset#1\endgroup
}
\def\mathchars@reset{\let\@sym\@sym@ord \let\@symtype\@symtype@ord
  \let\OrdSymbol\relax}
\def\@symtype@ord#1#{}% a strange sort of \@gobble
\def\@sym@ord#1#2{\@xp\@sym@ord@a\string#2\@nil}%
\begingroup
\lccode`\.=`\@ \lowercase{\endgroup
\def\@sym@ord@a#1.}#2#3\@nil#4#5#6{%
  \csname MathChar \endcsname"0%
    \if D#2\@xp\delim@b\csname sd@#4#5#6\endcsname#4#5#6
    \else #4#5#6
    \fi
}
%    \end{macrocode}
%
%
% Before declaring any math characters active, we have to take care of
% a small problem with \pkg{amsmath} v2.x, if it is loaded before
% \pkg{flexisym}. \cs{std@minus} and \cs{std@equal} are defined as
% \begin{verbatim}
% \mathchardef\std@minus\mathcode`\-\relax
% \mathchardef\std@equal\mathcode`\=\relax
% \end{verbatim}
% in \fn{amsmath.sty} and again \cs{AtBeginDocument}. The
% latter is because
% \begin{quote}
%   In case some alternative math fonts are loaded
%   later. [\fn{amsmath.dtx}]
% \end{quote}
% The problem arises because \pkg{flexisym} sets the mathcode of all
% symbols to $32768$ which is illegal for a \cs{mathchardef}. 
%
% We have to remove the assignments from the \cs{AtBeginDocument} hook
% as they will cause an error there.
%    \begin{macrocode}
\@ifpackageloaded{amsmath}{%
  \begingroup
%    \end{macrocode}
% Split the contents of \cs{@begindocumenthook} by reading what we
% search for as a delimited argument and ensure these two assignments
% do not take place. It is questionable if anything reasonable can be
% done to them. In the case of a package such as \pkg{mathpazo} which defines
% \begin{verbatim}
%\DeclareMathSymbol{=}{\mathrel}{upright}{"3D}
% \end{verbatim}
% the \cs{Relbar} will look wrong if we don't use the correct
% symbol. The way to solve this is define additional \fn{.sym} files
% which contain the definition of \cs{relbar} and \cs{Relbar}
% needed. We need those additional files anyway for things like
% \cs{joinord}.
%    \begin{macrocode}
  \long\def\next#1\mathchardef\std@minus\mathcode`\-\relax
                  \mathchardef\std@equal\mathcode`\=\relax#2\flexi@stop{%
    \toks@{#1#2}%
    \xdef\@begindocumenthook{\the\toks@}%
  }%
  \expandafter\next\@begindocumenthook\flexi@stop
  \endgroup
}{}
%    \end{macrocode}
%
% There is problem when using \cs{DeclareMathOperator} as the
% operators defined call a command \cs{newmcodes@} which relies on the
% mathcode of \texttt{-} being less than 32768. We delay the
% definition \cs{AtBeginDocument} in case \pkg{amssymb} hasn't been
% loaded yet.
%    \begin{macrocode}
\AtBeginDocument{%
\def\newmcodes@{%
  \mathcode `\'39\mathcode `\*42\mathcode `\."613A
  \ifnum\mathcode`\-=45
  \else
%    \end{macrocode}
% The extra check. Don't do anything if \texttt{-} is math active.
%    \begin{macrocode}
    \ifnum\mathcode`\-=32768
    \else 
      \mathchardef \std@minus \mathcode `\-\relax
    \fi
  \fi
  \mathcode `\-45 \mathcode `\/47\mathcode `\:"603A\relax 
}%
}
%    \end{macrocode}
%
% And we then continue with the options.
%    \begin{macrocode}
\DeclareOption{mathstyleoff}{\PassOptionsToPackage{mathactivechars}{mathstyle}}
\DeclareOption{cmbase}{\usesymbols{cmbase}}
\DeclareOption{mathpazo}{\usesymbols{mathpazo}}
\DeclareOption{mathptmx}{\usesymbols{mathptmx}}
\ExecuteOptions{cmbase}
\ProcessOptions\relax
\renewcommand{\lnot}{\neg}
\renewcommand{\land}{\wedge}
\renewcommand{\lor}{\vee}
\renewcommand{\le}{\leq}
\renewcommand{\ge}{\geq}
\renewcommand{\ne}{\neq}
\renewcommand{\owns}{\ni}
\renewcommand{\gets}{\leftarrow}
\renewcommand{\to}{\rightarrow}
\renewcommand{\|}{\Vert}
\RequirePackage{mathstyle}
%</package>\endinput
%    \end{macrocode}
%
% \section{cmbase, mathpazo, mathptmx}
%
%
% For each math font package we define a corresponding symbol file
% with extension \fn{sym}. The Computer Modern base is called
% \opt{cmbase} and \opt{mathpazo} and \opt{mathptmx} corresponds to
% the packages. The definitions are almost identical as they mostly
% concern the positions in the math font encodings. Look for
% differences in \cs{joinord}, \cs{relbar} and \cs{Relbar}. If you
% inspect the source code, you'll see that the support for
% \pkg{mathptmx} didn't require any work but I thought it better to
% create a \fn{sym} file to maintain a uniform interface.
%
% \begin{aside}
% Open question on \verb"!" and \verb"?": maybe they
% should have type `Pun' instead of `DeR'.    Need to
% search for uses in math in AMS archives.    Or, maybe add a special
% `Clo' type for them: non-extensible closing delimiter.   
% \end{aside}
% 
% 
% 
% Default mathgroup setup.   
%    \begin{macrocode}
%<*cmbase|mathpazo|mathptmx>
%<cmbase>\ProvidesSymbols{cmbase}[2007/12/19 v0.92]
%<mathpazo>\ProvidesSymbols{mathpazo}[2007/12/19 v0.2]
%<mathptmx>\ProvidesSymbols{mathptmx}[2007/12/19 v0.2]
\@xp\xdef\csname mg@OT1\endcsname{\hexnumber@\symoperators}
\@xp\xdef\csname mg@OML\endcsname{\hexnumber@\symletters}
\@xp\xdef\csname mg@OMS\endcsname{\hexnumber@\symsymbols}
\@xp\xdef\csname mg@OMX\endcsname{\hexnumber@\symlargesymbols}
\gdef\mg@bin{\mg@OMS}
\gdef\mg@del{\mg@OMX}
\xdef\mg@digit{\@xp\@nx\csname mg@OT1\endcsname}
\gdef\mg@latin{\mg@OML}
\global\let\mg@Latin\mg@latin
\global\let\mg@greek\mg@latin
\global\let\mg@Greek\mg@digit
\global\let\mg@rel\mg@bin
\global\let\mg@ord\mg@bin
\global\let\mg@cop\mg@del
%    \end{macrocode}
% 
% 
% Symbols from the 128-character \fn{cmr} encoding.   
% Paren and square bracket delimiters from this encoding are covered
% by the definitions in the \fn{cmex} section, however.   
%    \begin{macrocode}
\DeclareFlexSymbol{!}     {Pun}{OT1}{21}
\DeclareFlexSymbol{+}     {Bin}{OT1}{2B}
\DeclareFlexSymbol{:}     {Rel}{OT1}{3A}
\DeclareFlexSymbol{\colon}{Pun}{OT1}{3A}
\DeclareFlexSymbol{;}     {Pun}{OT1}{3B}
\DeclareFlexSymbol{=}     {Rel}{OT1}{3D}
\DeclareFlexSymbol{?}     {Pun}{OT1}{3F}
%    \end{macrocode}
% \AmS\TeX, and therefore the \pkg{amsmath} package, make the
% uppercase Greek letters class 0 (nonvariable) instead of 7
% (variable), to eliminate the glaring inconsistency with lowercase
% Greek.    (In plain \TeX , \verb"{\bf\Delta}" works, while
% \verb"{\bf\delta}" doesn't.   ) Let us try to make them both
% variable (fonts permitting) instead of nonvariable.   
%    \begin{macrocode}
\DeclareFlexSymbol{\Gamma}  {Var}{Greek}{00}
\DeclareFlexSymbol{\Delta}  {Var}{Greek}{01}
\DeclareFlexSymbol{\Theta}  {Var}{Greek}{02}
\DeclareFlexSymbol{\Lambda} {Var}{Greek}{03}
\DeclareFlexSymbol{\Xi}     {Var}{Greek}{04}
\DeclareFlexSymbol{\Pi}     {Var}{Greek}{05}
\DeclareFlexSymbol{\Sigma}  {Var}{Greek}{06}
\DeclareFlexSymbol{\Upsilon}{Var}{Greek}{07}
\DeclareFlexSymbol{\Phi}    {Var}{Greek}{08}
\DeclareFlexSymbol{\Psi}    {Var}{Greek}{09}
\DeclareFlexSymbol{\Omega}  {Var}{Greek}{0A}
%    \end{macrocode}
% Decimal digits.   
%    \begin{macrocode}
\DeclareFlexSymbol{0}{Var}{digit}{30}
\DeclareFlexSymbol{1}{Var}{digit}{31}
\DeclareFlexSymbol{2}{Var}{digit}{32}
\DeclareFlexSymbol{3}{Var}{digit}{33}
\DeclareFlexSymbol{4}{Var}{digit}{34}
\DeclareFlexSymbol{5}{Var}{digit}{35}
\DeclareFlexSymbol{6}{Var}{digit}{36}
\DeclareFlexSymbol{7}{Var}{digit}{37}
\DeclareFlexSymbol{8}{Var}{digit}{38}
\DeclareFlexSymbol{9}{Var}{digit}{39}
%    \end{macrocode}
% Symbols from the 128-character \fn{cmmi} encoding.   
%    \begin{macrocode}
\DeclareFlexSymbol{,}{Pun}{OML}{3B}
\DeclareFlexSymbol{.}{Ord}{OML}{3A}
\DeclareFlexSymbol{/}{Ord}{OML}{3D}
\DeclareFlexSymbol{<}{Rel}{OML}{3C}
\DeclareFlexSymbol{>}{Rel}{OML}{3E}
%    \end{macrocode}
% To do: make the Var property of lc Greek work properly.   
%    \begin{macrocode}
\DeclareFlexSymbol{\alpha}{Var}{greek}{0B}
\DeclareFlexSymbol{\beta}{Var}{greek}{0C}
\DeclareFlexSymbol{\gamma}{Var}{greek}{0D}
\DeclareFlexSymbol{\delta}{Var}{greek}{0E}
\DeclareFlexSymbol{\epsilon}{Var}{greek}{0F}
\DeclareFlexSymbol{\zeta}{Var}{greek}{10}
\DeclareFlexSymbol{\eta}{Var}{greek}{11}
\DeclareFlexSymbol{\theta}{Var}{greek}{12}
\DeclareFlexSymbol{\iota}{Var}{greek}{13}
\DeclareFlexSymbol{\kappa}{Var}{greek}{14}
\DeclareFlexSymbol{\lambda}{Var}{greek}{15}
\DeclareFlexSymbol{\mu}{Var}{greek}{16}
\DeclareFlexSymbol{\nu}{Var}{greek}{17}
\DeclareFlexSymbol{\xi}{Var}{greek}{18}
\DeclareFlexSymbol{\pi}{Var}{greek}{19}
\DeclareFlexSymbol{\rho}{Var}{greek}{1A}
\DeclareFlexSymbol{\sigma}{Var}{greek}{1B}
\DeclareFlexSymbol{\tau}{Var}{greek}{1C}
\DeclareFlexSymbol{\upsilon}{Var}{greek}{1D}
\DeclareFlexSymbol{\phi}{Var}{greek}{1E}
\DeclareFlexSymbol{\chi}{Var}{greek}{1F}
\DeclareFlexSymbol{\psi}{Var}{greek}{20}
\DeclareFlexSymbol{\omega}{Var}{greek}{21}
\DeclareFlexSymbol{\varepsilon}{Var}{greek}{22}
\DeclareFlexSymbol{\vartheta}{Var}{greek}{23}
\DeclareFlexSymbol{\varpi}{Var}{greek}{24}
\DeclareFlexSymbol{\varrho}{Var}{greek}{25}
\DeclareFlexSymbol{\varsigma}{Var}{greek}{26}
\DeclareFlexSymbol{\varphi}{Var}{greek}{27}
%    \end{macrocode}
% Note that in plain \TeX\  \cs{imath} and \cs{jmath} are
% not variable-font.    But if a \verb"j" changes font to, let's
% say, sans serif or calligraphic, a dotless \verb"j" in the same
% context should change font in the same way.   
%    \begin{macrocode}
\DeclareFlexSymbol{\imath}{Var}{OML}{7B}
\DeclareFlexSymbol{\jmath}{Var}{OML}{7C}
\DeclareFlexSymbol{\ell}{Ord}{OML}{60}
\DeclareFlexSymbol{\wp}{Ord}{OML}{7D}
\DeclareFlexSymbol{\partial}{Ord}{OML}{40}
\DeclareFlexSymbol{\flat}{Ord}{OML}{5B}
\DeclareFlexSymbol{\natural}{Ord}{OML}{5C}
\DeclareFlexSymbol{\sharp}{Ord}{OML}{5D}
\DeclareFlexSymbol{\triangleleft}{Bin}{OML}{2F}
\DeclareFlexSymbol{\triangleright}{Bin}{OML}{2E}
\DeclareFlexSymbol{\star}{Bin}{OML}{3F}
\DeclareFlexSymbol{\smile}{Rel}{OML}{5E}
\DeclareFlexSymbol{\frown}{Rel}{OML}{5F}
\DeclareFlexSymbol{\leftharpoonup}{Rel}{OML}{28}
\DeclareFlexSymbol{\leftharpoondown}{Rel}{OML}{29}
\DeclareFlexSymbol{\rightharpoonup}{Rel}{OML}{2A}
\DeclareFlexSymbol{\rightharpoondown}{Rel}{OML}{2B}
\DeclareFlexSymbol{a}{Var}{latin}{61}
\DeclareFlexSymbol{b}{Var}{latin}{62}
\DeclareFlexSymbol{c}{Var}{latin}{63}
\DeclareFlexSymbol{d}{Var}{latin}{64}
\DeclareFlexSymbol{e}{Var}{latin}{65}
\DeclareFlexSymbol{f}{Var}{latin}{66}
\DeclareFlexSymbol{g}{Var}{latin}{67}
\DeclareFlexSymbol{h}{Var}{latin}{68}
\DeclareFlexSymbol{i}{Var}{latin}{69}
\DeclareFlexSymbol{j}{Var}{latin}{6A}
\DeclareFlexSymbol{k}{Var}{latin}{6B}
\DeclareFlexSymbol{l}{Var}{latin}{6C}
\DeclareFlexSymbol{m}{Var}{latin}{6D}
\DeclareFlexSymbol{n}{Var}{latin}{6E}
\DeclareFlexSymbol{o}{Var}{latin}{6F}
\DeclareFlexSymbol{p}{Var}{latin}{70}
\DeclareFlexSymbol{q}{Var}{latin}{71}
\DeclareFlexSymbol{r}{Var}{latin}{72}
\DeclareFlexSymbol{s}{Var}{latin}{73}
\DeclareFlexSymbol{t}{Var}{latin}{74}
\DeclareFlexSymbol{u}{Var}{latin}{75}
\DeclareFlexSymbol{v}{Var}{latin}{76}
\DeclareFlexSymbol{w}{Var}{latin}{77}
\DeclareFlexSymbol{x}{Var}{latin}{78}
\DeclareFlexSymbol{y}{Var}{latin}{79}
\DeclareFlexSymbol{z}{Var}{latin}{7A}
\DeclareFlexSymbol{A}{Var}{Latin}{41}
\DeclareFlexSymbol{B}{Var}{Latin}{42}
\DeclareFlexSymbol{C}{Var}{Latin}{43}
\DeclareFlexSymbol{D}{Var}{Latin}{44}
\DeclareFlexSymbol{E}{Var}{Latin}{45}
\DeclareFlexSymbol{F}{Var}{Latin}{46}
\DeclareFlexSymbol{G}{Var}{Latin}{47}
\DeclareFlexSymbol{H}{Var}{Latin}{48}
\DeclareFlexSymbol{I}{Var}{Latin}{49}
\DeclareFlexSymbol{J}{Var}{Latin}{4A}
\DeclareFlexSymbol{K}{Var}{Latin}{4B}
\DeclareFlexSymbol{L}{Var}{Latin}{4C}
\DeclareFlexSymbol{M}{Var}{Latin}{4D}
\DeclareFlexSymbol{N}{Var}{Latin}{4E}
\DeclareFlexSymbol{O}{Var}{Latin}{4F}
\DeclareFlexSymbol{P}{Var}{Latin}{50}
\DeclareFlexSymbol{Q}{Var}{Latin}{51}
\DeclareFlexSymbol{R}{Var}{Latin}{52}
\DeclareFlexSymbol{S}{Var}{Latin}{53}
\DeclareFlexSymbol{T}{Var}{Latin}{54}
\DeclareFlexSymbol{U}{Var}{Latin}{55}
\DeclareFlexSymbol{V}{Var}{Latin}{56}
\DeclareFlexSymbol{W}{Var}{Latin}{57}
\DeclareFlexSymbol{X}{Var}{Latin}{58}
\DeclareFlexSymbol{Y}{Var}{Latin}{59}
\DeclareFlexSymbol{Z}{Var}{Latin}{5A}
%    \end{macrocode}
% The \cs{ldotPun} glyph is used in constructing the
% \cs{ldots} symbol.    It is just a period with a different math
% symbol class.    \cs{lhookRel} and \cs{rhookRel} are used
% in a similar way for building hooked arrow symbols.   
%    \begin{macrocode}
\DeclareFlexSymbol{\ldotPun}{Pun}{OML}{3A}
\def\ldotp{\ldotPun}
\DeclareFlexSymbol{\lhookRel}{Rel}{OML}{2C}
\DeclareFlexSymbol{\rhookRel}{Rel}{OML}{2D}
%    \end{macrocode}
% Symbols from the 128-character \fn{cmsy} encoding.   
%    \begin{macrocode}
\DeclareFlexSymbol{*}{Bin}{bin}{03} % \ast
\DeclareFlexSymbol{-}{Bin}{bin}{00}
\DeclareFlexSymbol{|}{Ord}{OMS}{6A}
\DeclareFlexSymbol{\aleph}{Ord}{ord}{40}
\DeclareFlexSymbol{\Re}{Ord}{ord}{3C}
\DeclareFlexSymbol{\Im}{Ord}{ord}{3D}
\DeclareFlexSymbol{\infty}{Ord}{ord}{31}
\DeclareFlexSymbol{\prime}{Ord}{ord}{30}
\DeclareFlexSymbol{\emptyset}{Ord}{ord}{3B}
\DeclareFlexSymbol{\nabla}{Ord}{ord}{72}
\DeclareFlexSymbol{\top}{Ord}{ord}{3E}
\DeclareFlexSymbol{\bot}{Ord}{ord}{3F}
\DeclareFlexSymbol{\triangle}{Ord}{ord}{34}
\DeclareFlexSymbol{\forall}{Ord}{ord}{38}
\DeclareFlexSymbol{\exists}{Ord}{ord}{39}
\DeclareFlexSymbol{\neg}{Ord}{ord}{3A}
\DeclareFlexSymbol{\clubsuit}{Ord}{ord}{7C}
\DeclareFlexSymbol{\diamondsuit}{Ord}{ord}{7D}
\DeclareFlexSymbol{\heartsuit}{Ord}{ord}{7E}
\DeclareFlexSymbol{\spadesuit}{Ord}{ord}{7F}
\DeclareFlexSymbol{\smallint}{COs}{OMS}{73}
%    \end{macrocode}
% Binary operators.   
%    \begin{macrocode}
\DeclareFlexSymbol{\bigtriangleup}{Bin}{bin}{34}
\DeclareFlexSymbol{\bigtriangledown}{Bin}{bin}{35}
\DeclareFlexSymbol{\wedge}{Bin}{bin}{5E}
\DeclareFlexSymbol{\vee}{Bin}{bin}{5F}
\DeclareFlexSymbol{\cap}{Bin}{bin}{5C}
\DeclareFlexSymbol{\cup}{Bin}{bin}{5B}
\DeclareFlexSymbol{\ddagger}{Bin}{bin}{7A}
\DeclareFlexSymbol{\dagger}{Bin}{bin}{79}
\DeclareFlexSymbol{\sqcap}{Bin}{bin}{75}
\DeclareFlexSymbol{\sqcup}{Bin}{bin}{74}
\DeclareFlexSymbol{\uplus}{Bin}{bin}{5D}
\DeclareFlexSymbol{\amalg}{Bin}{bin}{71}
\DeclareFlexSymbol{\diamond}{Bin}{bin}{05}
\DeclareFlexSymbol{\bullet}{Bin}{bin}{0F}
\DeclareFlexSymbol{\wr}{Bin}{bin}{6F}
\DeclareFlexSymbol{\div}{Bin}{bin}{04}
\DeclareFlexSymbol{\odot}{Bin}{bin}{0C}
\DeclareFlexSymbol{\oslash}{Bin}{bin}{0B}
\DeclareFlexSymbol{\otimes}{Bin}{bin}{0A}
\DeclareFlexSymbol{\ominus}{Bin}{bin}{09}
\DeclareFlexSymbol{\oplus}{Bin}{bin}{08}
\DeclareFlexSymbol{\mp}{Bin}{bin}{07}
\DeclareFlexSymbol{\pm}{Bin}{bin}{06}
\DeclareFlexSymbol{\circ}{Bin}{bin}{0E}
\DeclareFlexSymbol{\bigcirc}{Bin}{bin}{0D}
\DeclareFlexSymbol{\setminus}{Bin}{bin}{6E}
\DeclareFlexSymbol{\cdot}{Bin}{bin}{01}
\DeclareFlexSymbol{\ast}{Bin}{bin}{03}
\DeclareFlexSymbol{\times}{Bin}{bin}{02}
%    \end{macrocode}
% Relation symbols.   
%    \begin{macrocode}
\DeclareFlexSymbol{\propto}{Rel}{rel}{2F}
\DeclareFlexSymbol{\sqsubseteq}{Rel}{rel}{76}
\DeclareFlexSymbol{\sqsupseteq}{Rel}{rel}{77}
\DeclareFlexSymbol{\parallel}{Rel}{rel}{6B}
\DeclareFlexSymbol{\mid}{Rel}{rel}{6A}
\DeclareFlexSymbol{\dashv}{Rel}{rel}{61}
\DeclareFlexSymbol{\vdash}{Rel}{rel}{60}
\DeclareFlexSymbol{\nearrow}{Rel}{rel}{25}
\DeclareFlexSymbol{\searrow}{Rel}{rel}{26}
\DeclareFlexSymbol{\nwarrow}{Rel}{rel}{2D}
\DeclareFlexSymbol{\swarrow}{Rel}{rel}{2E}
\DeclareFlexSymbol{\Leftrightarrow}{Rel}{rel}{2C}
\DeclareFlexSymbol{\Leftarrow}{Rel}{rel}{28}
\DeclareFlexSymbol{\Rightarrow}{Rel}{rel}{29}
\DeclareFlexSymbol{\leq}{Rel}{rel}{14}
\DeclareFlexSymbol{\geq}{Rel}{rel}{15}
\DeclareFlexSymbol{\succ}{Rel}{rel}{1F}
\DeclareFlexSymbol{\prec}{Rel}{rel}{1E}
\DeclareFlexSymbol{\approx}{Rel}{rel}{19}
\DeclareFlexSymbol{\succeq}{Rel}{rel}{17}
\DeclareFlexSymbol{\preceq}{Rel}{rel}{16}
\DeclareFlexSymbol{\supset}{Rel}{rel}{1B}
\DeclareFlexSymbol{\subset}{Rel}{rel}{1A}
\DeclareFlexSymbol{\supseteq}{Rel}{rel}{13}
\DeclareFlexSymbol{\subseteq}{Rel}{rel}{12}
\DeclareFlexSymbol{\in}{Rel}{rel}{32}
\DeclareFlexSymbol{\ni}{Rel}{rel}{33}
\DeclareFlexSymbol{\gg}{Rel}{rel}{1D}
\DeclareFlexSymbol{\ll}{Rel}{rel}{1C}
\DeclareFlexSymbol{\leftrightarrow}{Rel}{rel}{24}
\DeclareFlexSymbol{\leftarrow}{Rel}{rel}{20}
\DeclareFlexSymbol{\rightarrow}{Rel}{rel}{21}
\DeclareFlexSymbol{\sim}{Rel}{rel}{18}
\DeclareFlexSymbol{\simeq}{Rel}{rel}{27}
\DeclareFlexSymbol{\perp}{Rel}{rel}{3F}
\DeclareFlexSymbol{\equiv}{Rel}{rel}{11}
\DeclareFlexSymbol{\asymp}{Rel}{rel}{10}
%    \end{macrocode}
% The \cs{notRel} glyph is a special zero-width glyph intended only
% for use in constructing negated symbols.    \cs{mapstoRel} and
% \cs{cdotPun} have similar but more restricted applications.   
%    \begin{macrocode}
\DeclareFlexSymbol{\notRel}{Rel}{rel}{36}
\DeclareFlexSymbol{\mapstoOrd}{Ord}{OMS}{37}
\DeclareFlexSymbol{\cdotOrd}{Ord}{OMS}{01}
\def\cdotp{\mathpunct{\cdotOrd}}
%    \end{macrocode}
% Symbols from the 128-character \fn{cmex} encoding.   
% \verb"COs" stands for `cumulative operator
% (sum-like)'.   
% \verb"COi" stands for `cumulative operator
% (integral-like)'.    These typically differ only in the
% default placement of limits.    \verb"cop" stands for
% `cumulative operator math group'.   
%    \begin{macrocode}
\DeclareFlexSymbol{\coprod}{COs}{cop}{60}
\DeclareFlexSymbol{\bigvee}{COs}{cop}{57}
\DeclareFlexSymbol{\bigwedge}{COs}{cop}{56}
\DeclareFlexSymbol{\biguplus}{COs}{cop}{55}
\DeclareFlexSymbol{\bigcap}{COs}{cop}{54}
\DeclareFlexSymbol{\bigcup}{COs}{cop}{53}
\DeclareFlexSymbol{\int}{COi}{cop}{52}
\DeclareFlexSymbol{\prod}{COs}{cop}{51}
\DeclareFlexSymbol{\sum}{COs}{cop}{50}
\DeclareFlexSymbol{\bigotimes}{COs}{cop}{4E}
\DeclareFlexSymbol{\bigoplus}{COs}{cop}{4C}
\DeclareFlexSymbol{\bigodot}{COs}{cop}{4A}
\DeclareFlexSymbol{\oint}{COi}{cop}{48}
\DeclareFlexSymbol{\bigsqcup}{COs}{cop}{46}
%    \end{macrocode}
% Delimiter symbols.   
% \verb"DeL" stands for `delimiter (left)'.   
% \verb"DeR" stands for `delimiter (right)'.   
% \verb"DeB" stands for `delimiter (bidirectional)'.   
% The principal encoding point for an extensible delimiter is the
% first link in the list of linked sizes as specified in the font metric
% information.   
% For a math encoding such as OT1/OML/OMS/OMX where not all sizes of a
% given delimiter reside in a given font, the extra encoding point for the
% smallest delimiter must be supplied by defining
% \begin{verbatim}
% \sd@GXX
% \end{verbatim}
% where G is the mathgroup and XX is the hexadecimal glyph position.   
%    \begin{macrocode}
\DeclareFlexSymbol{\rangle}{DeR}{del}{0B}
\DeclareFlexSymbol{\langle}{DeL}{del}{0A}
\DeclareFlexSymbol{\rbrace}{DeR}{del}{09}
\DeclareFlexSymbol{\lbrace}{DeL}{del}{08}
\DeclareFlexSymbol{\rceil}{DeR}{del}{07}
\DeclareFlexSymbol{\lceil}{DeL}{del}{06}
\DeclareFlexSymbol{\rfloor}{DeR}{del}{05}
\DeclareFlexSymbol{\lfloor}{DeL}{del}{04}
\DeclareFlexSymbol{(}{DeL}{del}{00}
\DeclareFlexSymbol{)}{DeR}{del}{01}
\DeclareFlexSymbol{[}{DeL}{del}{02}
\DeclareFlexSymbol{]}{DeR}{del}{03}
\DeclareFlexSymbol{\lVert}{DeL}{del}{0D}
\DeclareFlexSymbol{\rVert}{DeR}{del}{0D}
\DeclareFlexSymbol{\lvert}{DeL}{del}{0C}
\DeclareFlexSymbol{\rvert}{DeR}{del}{0C}
\DeclareFlexSymbol{\Vert}{DeB}{del}{0D}
\DeclareFlexSymbol{\vert}{DeB}{del}{0C}
%    \end{macrocode}
% Maybe make the vert bars mathord instead of delimiter, to discourage
% poor usage.   
%    \begin{macrocode}
\DeclareFlexSymbol{|}{DeB}{del}{0C}
\DeclareFlexSymbol{/}{DeB}{del}{0E}
%    \end{macrocode}
% 
% 
% These wacky delimiters need to be supported I guess for
% compabitility reasons.   
% The DeA delimiter type is a special case used only for these
% arrows.   
%    \begin{macrocode}
\DeclareFlexSymbol{\lmoustache}{DeL}{del}{40}
\DeclareFlexSymbol{\rmoustache}{DeR}{del}{41}
\DeclareFlexSymbol{\lgroup}{DeL}{del}{3A}
\DeclareFlexSymbol{\rgroup}{DeR}{del}{3B}
\DeclareFlexSymbol{\bracevert}{DeB}{del}{3E}
\DeclareFlexSymbol{\arrowvert}{DeB}{del}{3C}
\DeclareFlexSymbol{\Arrowvert}{DeB}{del}{3D}
\DeclareFlexSymbol{\uparrow}{DeA}{del}{78}
\DeclareFlexSymbol{\downarrow}{DeA}{del}{79}
\DeclareFlexSymbol{\updownarrow}{DeA}{del}{3F}
\DeclareFlexSymbol{\Uparrow}{DeA}{del}{7E}
\DeclareFlexSymbol{\Downarrow}{DeA}{del}{7F}
\DeclareFlexSymbol{\Updownarrow}{DeA}{del}{77}
\DeclareFlexSymbol{\backslash}{DeB}{del}{0F}
%    \end{macrocode}
% 
% 
% 
% 
% \section{Some compound symbols}
% The following symbols are not robust in standard \LaTeX\ 
% because they use \verb"#" or \cs{mathpalette} (which is not
% robust and contains a \verb"#" in its expansion): \cs{angle},
% \cs{cong}, \cs{notin}, \cs{rightleftharpoons}.   
% 
% In this definition of \cs{hbar}, the symbol is cobbled together
% from a math italic h and the cmr overbar accent glyph.   
%    \begin{macrocode}
\DeclareFlexSymbol{\hbarOrd}{Ord}{OT1}{16}
\DeclareFlexCompoundSymbol{\hbar}{Ord}{\hbarOrd\mkern-9mu h}
%    \end{macrocode}
% For \cs{surd}, the interior symbol gets math class 1
% (cumulative operator) to make the glyph vertically centered on the
% math axis, but the desired horizontal spacing is the spacing for a
% mathord.    (Couldn't it just be class mathopen, though?   )
%    \begin{macrocode}
\DeclareFlexSymbol{\surdOrd}{Ord}{OMS}{70}
\DeclareFlexCompoundSymbol{\surd}{Ord}{\mathop{\surdOrd}}
%    \end{macrocode}
% As shown in this definition of \cs{angle}, rule dimens are not
% allowed to use math-units, unfortunately.   
%    \begin{macrocode}
\DeclareFlexCompoundSymbol{\angle}{Ord}{%
  \vbox{\ialign{%
      $\m@th\scriptstyle##$\crcr
      \notRel\mathrel{\mkern14mu}\crcr
      \noalign{\nointerlineskip}%
      \mkern2.5mu\leaders\hrule \@height.34pt\hfill\mkern2.5mu\crcr
  }}%
}
%    \end{macrocode}
% The \cs{not} function, which is defined in the \pkg{flexisym}
% package, requires a suitably defined \cs{notRel} symbol.   
%    \begin{macrocode}
\DeclareFlexCompoundSymbol{\neq}{Rel}{\not{=}}
%    \end{macrocode}
% .   
%    \begin{macrocode}
\DeclareFlexCompoundSymbol{\mapsto}{Rel}{\mapstoOrd\rightarrow}
%    \end{macrocode}
% The \cs{@vereq} function ends by centering the whole
% construction on the math axis, unlike \cs{buildrel} where the base
% symbol remains at its normal altitude.    Furthermore,
% \cs{@vereq} leaves the math style of the top symbol as given
% instead of downsizing to scriptstyle.   
%    \begin{macrocode}
\DeclareFlexCompoundSymbol{\cong}{Rel}{\mathpalette\@vereq\sim}
%    \end{macrocode}
% The \cs{m@th} in the \fn{fontmath.ltx} definition of
% \cs{notin} is superfluous unless \cs{c@ncel} doesn't include
% it (which was perhaps true in an older version of
% \fn{plain.tex}?).   
%    \begin{macrocode}
\providecommand*\joinord{}
%<cmbase|mathptmx>\renewcommand*\joinord{\mkern-3mu }
%<mathpazo>\renewcommand*\joinord{\mkern-3.45mu }
\DeclareFlexCompoundSymbol{\notin}{Rel}{\mathpalette\c@ncel\in}
\DeclareFlexCompoundSymbol{\rightleftharpoons}{Rel}{\mathpalette\rlh@{}}
\DeclareFlexCompoundSymbol{\doteq}{Rel}{\buildrel\textstyle.\over=}
\DeclareFlexCompoundSymbol{\hookrightarrow}{Rel}{\lhookRel\joinord\rightarrow}
\DeclareFlexCompoundSymbol{\hookleftarrow}{Rel}{\leftarrow\joinord\rhookRel}
\DeclareFlexCompoundSymbol{\bowtie}{Rel}{\triangleright\joinord\triangleleft}
\DeclareFlexCompoundSymbol{\models}{Rel}{\vert\joinord=}
\DeclareFlexCompoundSymbol{\Longrightarrow}{Rel}{\Relbar\joinord\Rightarrow}
\DeclareFlexCompoundSymbol{\longrightarrow}{Rel}{\relbar\joinord\rightarrow}
\DeclareFlexCompoundSymbol{\Longleftarrow}{Rel}{\Leftarrow\joinord\Relbar}
\DeclareFlexCompoundSymbol{\longleftarrow}{Rel}{\leftarrow\joinord\relbar}
\DeclareFlexCompoundSymbol{\longmapsto}{Rel}{\mapstochar\longrightarrow}
\DeclareFlexCompoundSymbol{\longleftrightarrow}{Rel}{\leftarrow\joinord\rightarrow}
\DeclareFlexCompoundSymbol{\Longleftrightarrow}{Rel}{\Leftarrow\joinord\Rightarrow}
%    \end{macrocode}
% Here is what you get from the old definition of \cs{iff}.   
% \begin{verbatim}
% \glue 2.77771 plus 2.77771
% \glue(\thickmuskip) 2.77771 plus 2.77771
% \OMS/cmsy/m/n/10 (
% \hbox(0.0+0.0)x-1.66663
% .\kern -1.66663
% \OMS/cmsy/m/n/10 )
% \penalty 500
% \glue 2.77771 plus 2.77771
% \glue(\thickmuskip) 2.77771 plus 2.77771
% \end{verbatim}
% Looks like it could be simplified slightly.    But it's not so
% easy as it looks to do it without screwing up the line breaking
% possibilities.   
%    \begin{macrocode}
\renewcommand*\iff{%
  \mskip\thickmuskip\Longleftrightarrow\mskip\thickmuskip
}
%    \end{macrocode}
% Some dotly symbols.   
%    \begin{macrocode}
\DeclareFlexCompoundSymbol{\cdots}{Inn}{\cdotp\cdotp\cdotp}%
\DeclareFlexCompoundSymbol{\vdots}{Ord}{%
  \vbox{\baselineskip4\p@ \lineskiplimit\z@
    \kern6\p@\hbox{.}\hbox{.}\hbox{.}}}
\DeclareFlexCompoundSymbol{\ddots}{Inn}{%
  \mkern1mu\raise7\p@
  \vbox{\kern7\p@\hbox{.}}\mkern2mu%
  \raise4\p@\hbox{.}\mkern2mu\raise\p@\hbox{.}\mkern1mu%
}
%    \end{macrocode}
% .   
%    \begin{macrocode}
\def\relbar{\begingroup \def\smash@{tb}% in case amsmath is loaded
    \mathpalette\mathsm@sh{\mathchar"200 }\endgroup}
%    \end{macrocode}
% For \cs{Relbar} we take an equal sign of class $0$ (Ord) from the
% operator family. For \fn{cmr} and \pkg{mathptmx} we know this is
% family $0$.
%    \begin{macrocode}
%<cmbase|mathptmx>\def\Relbar{\mathchar"3D }
%    \end{macrocode}
% For the \pkg{mathpazo} setup we need to use the equal sign from
% \fn{cmr} and so must insert class $0$ and use the symbol from the
% upright symbols.
%    \begin{macrocode}
%<mathpazo>\edef\Relbar{\mathchar\string"\hexnumber@\symupright3D }
%    \end{macrocode}
% Done.
%    \begin{macrocode}
%</cmbase|mathpazo|mathptmx>
%    \end{macrocode}
% Various synonyms such as \cs{le} for \cs{leq} and
% \cs{to} for \cs{rightarrow} are defined in
% \pkg{flexisym} with \cs{def} instead of \cs{let}, for
% slower execution speed but smaller chance of synchronization
% problems.   
%
%
%
%    \begin{macrocode}
%<*msabm>
\ProvidesSymbols{msabm}[2001/09/08 v0.91]
%    \end{macrocode}
%    \begin{macrocode}
\RequirePackage{amsfonts}\relax
%    \end{macrocode}
%    \begin{macrocode}
\@xp\xdef\csname mg@MSA\endcsname{\hexnumber@\symAMSa}%
\@xp\xdef\csname mg@MSB\endcsname{\hexnumber@\symAMSb}%
%    \end{macrocode}
%    \begin{macrocode}
\DeclareFlexSymbol{\boxdot}       {Bin}{MSA}{00}
\DeclareFlexSymbol{\boxplus}      {Bin}{MSA}{01}
\DeclareFlexSymbol{\boxtimes}     {Bin}{MSA}{02}
\DeclareFlexSymbol{\square}       {Ord}{MSA}{03}
\DeclareFlexSymbol{\blacksquare}  {Ord}{MSA}{04}
\DeclareFlexSymbol{\centerdot}    {Bin}{MSA}{05}
\DeclareFlexSymbol{\lozenge}      {Ord}{MSA}{06}
\DeclareFlexSymbol{\blacklozenge} {Ord}{MSA}{07}
\DeclareFlexSymbol{\circlearrowright}   {Rel}{MSA}{08}
\DeclareFlexSymbol{\circlearrowleft}    {Rel}{MSA}{09}
%    \end{macrocode}
% In amsfonts.sty:
%    \begin{macrocode}
%%\DeclareFlexSymbol{\rightleftharpoons}{Rel}{MSA}{0A}
\DeclareFlexSymbol{\leftrightharpoons}  {Rel}{MSA}{0B}
\DeclareFlexSymbol{\boxminus}     {Bin}{MSA}{0C}
\DeclareFlexSymbol{\Vdash}        {Rel}{MSA}{0D}
\DeclareFlexSymbol{\Vvdash}       {Rel}{MSA}{0E}
\DeclareFlexSymbol{\vDash}        {Rel}{MSA}{0F}
\DeclareFlexSymbol{\twoheadrightarrow}  {Rel}{MSA}{10}
\DeclareFlexSymbol{\twoheadleftarrow}   {Rel}{MSA}{11}
\DeclareFlexSymbol{\leftleftarrows}     {Rel}{MSA}{12}
\DeclareFlexSymbol{\rightrightarrows}   {Rel}{MSA}{13}
\DeclareFlexSymbol{\upuparrows}         {Rel}{MSA}{14}
\DeclareFlexSymbol{\downdownarrows} {Rel}{MSA}{15}
\DeclareFlexSymbol{\upharpoonright} {Rel}{MSA}{16}
 \let\restriction\upharpoonright
\DeclareFlexSymbol{\downharpoonright}   {Rel}{MSA}{17}
\DeclareFlexSymbol{\upharpoonleft}  {Rel}{MSA}{18}
\DeclareFlexSymbol{\downharpoonleft}{Rel}{MSA}{19}
\DeclareFlexSymbol{\rightarrowtail} {Rel}{MSA}{1A}
\DeclareFlexSymbol{\leftarrowtail}  {Rel}{MSA}{1B}
\DeclareFlexSymbol{\leftrightarrows}{Rel}{MSA}{1C}
\DeclareFlexSymbol{\rightleftarrows}{Rel}{MSA}{1D}
\DeclareFlexSymbol{\Lsh}            {Rel}{MSA}{1E}
\DeclareFlexSymbol{\Rsh}            {Rel}{MSA}{1F}
\DeclareFlexSymbol{\rightsquigarrow}  {Rel}{MSA}{20}
\DeclareFlexSymbol{\leftrightsquigarrow}{Rel}{MSA}{21}
\DeclareFlexSymbol{\looparrowleft}  {Rel}{MSA}{22}
\DeclareFlexSymbol{\looparrowright} {Rel}{MSA}{23}
\DeclareFlexSymbol{\circeq}       {Rel}{MSA}{24}
\DeclareFlexSymbol{\succsim}      {Rel}{MSA}{25}
\DeclareFlexSymbol{\gtrsim}       {Rel}{MSA}{26}
\DeclareFlexSymbol{\gtrapprox}    {Rel}{MSA}{27}
\DeclareFlexSymbol{\multimap}     {Rel}{MSA}{28}
\DeclareFlexSymbol{\therefore}    {Rel}{MSA}{29}
\DeclareFlexSymbol{\because}      {Rel}{MSA}{2A}
\DeclareFlexSymbol{\doteqdot}     {Rel}{MSA}{2B}
 \let\Doteq\doteqdot
\DeclareFlexSymbol{\triangleq}    {Rel}{MSA}{2C}
\DeclareFlexSymbol{\precsim}      {Rel}{MSA}{2D}
\DeclareFlexSymbol{\lesssim}      {Rel}{MSA}{2E}
\DeclareFlexSymbol{\lessapprox}   {Rel}{MSA}{2F}
\DeclareFlexSymbol{\eqslantless}  {Rel}{MSA}{30}
\DeclareFlexSymbol{\eqslantgtr}   {Rel}{MSA}{31}
\DeclareFlexSymbol{\curlyeqprec}  {Rel}{MSA}{32}
\DeclareFlexSymbol{\curlyeqsucc}  {Rel}{MSA}{33}
\DeclareFlexSymbol{\preccurlyeq}  {Rel}{MSA}{34}
\DeclareFlexSymbol{\leqq}         {Rel}{MSA}{35}
\DeclareFlexSymbol{\leqslant}     {Rel}{MSA}{36}
\DeclareFlexSymbol{\lessgtr}      {Rel}{MSA}{37}
\DeclareFlexSymbol{\backprime}    {Ord}{MSA}{38}
\DeclareFlexSymbol{\risingdotseq} {Rel}{MSA}{3A}
\DeclareFlexSymbol{\fallingdotseq}{Rel}{MSA}{3B}
\DeclareFlexSymbol{\succcurlyeq}  {Rel}{MSA}{3C}
\DeclareFlexSymbol{\geqq}         {Rel}{MSA}{3D}
\DeclareFlexSymbol{\geqslant}     {Rel}{MSA}{3E}
\DeclareFlexSymbol{\gtrless}      {Rel}{MSA}{3F}
%    \end{macrocode}
% in amsfonts.sty
%    \begin{macrocode}
%% \DeclareFlexSymbol{\sqsubset}    {Rel}{MSA}{40}
%% \DeclareFlexSymbol{\sqsupset}    {Rel}{MSA}{41}
\DeclareFlexSymbol{\vartriangleright}{Rel}{MSA}{42}
\DeclareFlexSymbol{\vartriangleleft} {Rel}{MSA}{43}
\DeclareFlexSymbol{\trianglerighteq} {Rel}{MSA}{44}
\DeclareFlexSymbol{\trianglelefteq}  {Rel}{MSA}{45}
\DeclareFlexSymbol{\bigstar}    {Ord}{MSA}{46}
\DeclareFlexSymbol{\between}    {Rel}{MSA}{47}
\DeclareFlexSymbol{\blacktriangledown}  {Ord}{MSA}{48}
\DeclareFlexSymbol{\blacktriangleright} {Rel}{MSA}{49}
\DeclareFlexSymbol{\blacktriangleleft}  {Rel}{MSA}{4A}
\DeclareFlexSymbol{\vartriangle}        {Rel}{MSA}{4D}
\DeclareFlexSymbol{\blacktriangle}      {Ord}{MSA}{4E}
\DeclareFlexSymbol{\triangledown}       {Ord}{MSA}{4F}
\DeclareFlexSymbol{\eqcirc}       {Rel}{MSA}{50}
\DeclareFlexSymbol{\lesseqgtr}    {Rel}{MSA}{51}
\DeclareFlexSymbol{\gtreqless}    {Rel}{MSA}{52}
\DeclareFlexSymbol{\lesseqqgtr}   {Rel}{MSA}{53}
\DeclareFlexSymbol{\gtreqqless}   {Rel}{MSA}{54}
\DeclareFlexSymbol{\Rrightarrow}  {Rel}{MSA}{56}
\DeclareFlexSymbol{\Lleftarrow}   {Rel}{MSA}{57}
\DeclareFlexSymbol{\veebar}       {Bin}{MSA}{59}
\DeclareFlexSymbol{\barwedge}     {Bin}{MSA}{5A}
\DeclareFlexSymbol{\doublebarwedge} {Bin}{MSA}{5B}
%    \end{macrocode}
% In amsfonts.sty
%    \begin{macrocode}
%%\DeclareFlexSymbol{\angle}        {Ord}{MSA}{5C}
\DeclareFlexSymbol{\measuredangle}  {Ord}{MSA}{5D}
\DeclareFlexSymbol{\sphericalangle} {Ord}{MSA}{5E}
\DeclareFlexSymbol{\varpropto}    {Rel}{MSA}{5F}
\DeclareFlexSymbol{\smallsmile}   {Rel}{MSA}{60}
\DeclareFlexSymbol{\smallfrown}   {Rel}{MSA}{61}
\DeclareFlexSymbol{\Subset}       {Rel}{MSA}{62}
\DeclareFlexSymbol{\Supset}       {Rel}{MSA}{63}
\DeclareFlexSymbol{\Cup}          {Bin}{MSA}{64}
 \let\doublecup\Cup
\DeclareFlexSymbol{\Cap}          {Bin}{MSA}{65}
 \let\doublecap\Cap
\DeclareFlexSymbol{\curlywedge}   {Bin}{MSA}{66}
\DeclareFlexSymbol{\curlyvee}     {Bin}{MSA}{67}
\DeclareFlexSymbol{\leftthreetimes} {Bin}{MSA}{68}
\DeclareFlexSymbol{\rightthreetimes}{Bin}{MSA}{69}
\DeclareFlexSymbol{\subseteqq}    {Rel}{MSA}{6A}
\DeclareFlexSymbol{\supseteqq}    {Rel}{MSA}{6B}
\DeclareFlexSymbol{\bumpeq}       {Rel}{MSA}{6C}
\DeclareFlexSymbol{\Bumpeq}       {Rel}{MSA}{6D}
\DeclareFlexSymbol{\lll}          {Rel}{MSA}{6E}
 \let\llless\lll
\DeclareFlexSymbol{\ggg}          {Rel}{MSA}{6F}
 \let\gggtr\ggg
\DeclareFlexSymbol{\circledS}     {Ord}{MSA}{73}
\DeclareFlexSymbol{\pitchfork}    {Rel}{MSA}{74}
\DeclareFlexSymbol{\dotplus}      {Bin}{MSA}{75}
\DeclareFlexSymbol{\backsim}      {Rel}{MSA}{76}
\DeclareFlexSymbol{\backsimeq}    {Rel}{MSA}{77}
\DeclareFlexSymbol{\complement}   {Ord}{MSA}{7B}
\DeclareFlexSymbol{\intercal}     {Bin}{MSA}{7C}
\DeclareFlexSymbol{\circledcirc}  {Bin}{MSA}{7D}
\DeclareFlexSymbol{\circledast}   {Bin}{MSA}{7E}
\DeclareFlexSymbol{\circleddash}  {Bin}{MSA}{7F}
%    \end{macrocode}
%   Begin AMSb declarations
%    \begin{macrocode}
\DeclareFlexSymbol{\lvertneqq}    {Rel}{MSB}{00}
\DeclareFlexSymbol{\gvertneqq}    {Rel}{MSB}{01}
\DeclareFlexSymbol{\nleq}         {Rel}{MSB}{02}
\DeclareFlexSymbol{\ngeq}         {Rel}{MSB}{03}
\DeclareFlexSymbol{\nless}        {Rel}{MSB}{04}
\DeclareFlexSymbol{\ngtr}         {Rel}{MSB}{05}
\DeclareFlexSymbol{\nprec}        {Rel}{MSB}{06}
\DeclareFlexSymbol{\nsucc}        {Rel}{MSB}{07}
\DeclareFlexSymbol{\lneqq}        {Rel}{MSB}{08}
\DeclareFlexSymbol{\gneqq}        {Rel}{MSB}{09}
\DeclareFlexSymbol{\nleqslant}    {Rel}{MSB}{0A}
\DeclareFlexSymbol{\ngeqslant}    {Rel}{MSB}{0B}
\DeclareFlexSymbol{\lneq}         {Rel}{MSB}{0C}
\DeclareFlexSymbol{\gneq}         {Rel}{MSB}{0D}
\DeclareFlexSymbol{\npreceq}      {Rel}{MSB}{0E}
\DeclareFlexSymbol{\nsucceq}      {Rel}{MSB}{0F}
\DeclareFlexSymbol{\precnsim}     {Rel}{MSB}{10}
\DeclareFlexSymbol{\succnsim}     {Rel}{MSB}{11}
\DeclareFlexSymbol{\lnsim}        {Rel}{MSB}{12}
\DeclareFlexSymbol{\gnsim}        {Rel}{MSB}{13}
\DeclareFlexSymbol{\nleqq}        {Rel}{MSB}{14}
\DeclareFlexSymbol{\ngeqq}        {Rel}{MSB}{15}
\DeclareFlexSymbol{\precneqq}     {Rel}{MSB}{16}
\DeclareFlexSymbol{\succneqq}     {Rel}{MSB}{17}
\DeclareFlexSymbol{\precnapprox}  {Rel}{MSB}{18}
\DeclareFlexSymbol{\succnapprox}  {Rel}{MSB}{19}
\DeclareFlexSymbol{\lnapprox}     {Rel}{MSB}{1A}
\DeclareFlexSymbol{\gnapprox}     {Rel}{MSB}{1B}
\DeclareFlexSymbol{\nsim}         {Rel}{MSB}{1C}
\DeclareFlexSymbol{\ncong}        {Rel}{MSB}{1D}
\DeclareFlexSymbol{\diagup}       {Ord}{MSB}{1E}
\DeclareFlexSymbol{\diagdown}     {Ord}{MSB}{1F}
\DeclareFlexSymbol{\varsubsetneq}   {Rel}{MSB}{20}
\DeclareFlexSymbol{\varsupsetneq}   {Rel}{MSB}{21}
\DeclareFlexSymbol{\nsubseteqq}     {Rel}{MSB}{22}
\DeclareFlexSymbol{\nsupseteqq}     {Rel}{MSB}{23}
\DeclareFlexSymbol{\subsetneqq}     {Rel}{MSB}{24}
\DeclareFlexSymbol{\supsetneqq}     {Rel}{MSB}{25}
\DeclareFlexSymbol{\varsubsetneqq}  {Rel}{MSB}{26}
\DeclareFlexSymbol{\varsupsetneqq}  {Rel}{MSB}{27}
\DeclareFlexSymbol{\subsetneq}      {Rel}{MSB}{28}
\DeclareFlexSymbol{\supsetneq}      {Rel}{MSB}{29}
\DeclareFlexSymbol{\nsubseteq}      {Rel}{MSB}{2A}
\DeclareFlexSymbol{\nsupseteq}      {Rel}{MSB}{2B}
\DeclareFlexSymbol{\nparallel}      {Rel}{MSB}{2C}
\DeclareFlexSymbol{\nmid}           {Rel}{MSB}{2D}
\DeclareFlexSymbol{\nshortmid}      {Rel}{MSB}{2E}
\DeclareFlexSymbol{\nshortparallel} {Rel}{MSB}{2F}
\DeclareFlexSymbol{\nvdash}         {Rel}{MSB}{30}
\DeclareFlexSymbol{\nVdash}         {Rel}{MSB}{31}
\DeclareFlexSymbol{\nvDash}         {Rel}{MSB}{32}
\DeclareFlexSymbol{\nVDash}         {Rel}{MSB}{33}
\DeclareFlexSymbol{\ntrianglerighteq}{Rel}{MSB}{34}
\DeclareFlexSymbol{\ntrianglelefteq}{Rel}{MSB}{35}
\DeclareFlexSymbol{\ntriangleleft}  {Rel}{MSB}{36}
\DeclareFlexSymbol{\ntriangleright} {Rel}{MSB}{37}
\DeclareFlexSymbol{\nleftarrow}     {Rel}{MSB}{38}
\DeclareFlexSymbol{\nrightarrow}    {Rel}{MSB}{39}
\DeclareFlexSymbol{\nLeftarrow}     {Rel}{MSB}{3A}
\DeclareFlexSymbol{\nRightarrow}    {Rel}{MSB}{3B}
\DeclareFlexSymbol{\nLeftrightarrow}{Rel}{MSB}{3C}
\DeclareFlexSymbol{\nleftrightarrow}{Rel}{MSB}{3D}
\DeclareFlexSymbol{\divideontimes}  {Bin}{MSB}{3E}
\DeclareFlexSymbol{\varnothing}     {Ord}{MSB}{3F}
\DeclareFlexSymbol{\nexists}        {Ord}{MSB}{40}
\DeclareFlexSymbol{\Finv}           {Ord}{MSB}{60}
\DeclareFlexSymbol{\Game}           {Ord}{MSB}{61}
%    \end{macrocode}
% In amsfonts.sty:
%    \begin{macrocode}
%%\DeclareFlexSymbol{\mho}          {Ord}{MSB}{66}
\DeclareFlexSymbol{\eth}            {Ord}{MSB}{67}
\DeclareFlexSymbol{\eqsim}          {Rel}{MSB}{68}
\DeclareFlexSymbol{\beth}           {Ord}{MSB}{69}
\DeclareFlexSymbol{\gimel}          {Ord}{MSB}{6A}
\DeclareFlexSymbol{\daleth}         {Ord}{MSB}{6B}
\DeclareFlexSymbol{\lessdot}        {Bin}{MSB}{6C}
\DeclareFlexSymbol{\gtrdot}         {Bin}{MSB}{6D}
\DeclareFlexSymbol{\ltimes}         {Bin}{MSB}{6E}
\DeclareFlexSymbol{\rtimes}         {Bin}{MSB}{6F}
\DeclareFlexSymbol{\shortmid}       {Rel}{MSB}{70}
\DeclareFlexSymbol{\shortparallel}  {Rel}{MSB}{71}
\DeclareFlexSymbol{\smallsetminus}  {Bin}{MSB}{72}
\DeclareFlexSymbol{\thicksim}       {Rel}{MSB}{73}
\DeclareFlexSymbol{\thickapprox}    {Rel}{MSB}{74}
\DeclareFlexSymbol{\approxeq}       {Rel}{MSB}{75}
\DeclareFlexSymbol{\succapprox}     {Rel}{MSB}{76}
\DeclareFlexSymbol{\precapprox}     {Rel}{MSB}{77}
\DeclareFlexSymbol{\curvearrowleft} {Rel}{MSB}{78}
\DeclareFlexSymbol{\curvearrowright}{Rel}{MSB}{79}
\DeclareFlexSymbol{\digamma}        {Ord}{MSB}{7A}
\DeclareFlexSymbol{\varkappa}       {Ord}{MSB}{7B}
\DeclareFlexSymbol{\Bbbk}           {Ord}{MSB}{7C}
\DeclareFlexSymbol{\hslash}         {Ord}{MSB}{7D}
%    \end{macrocode}
% In amsfonts.sty:
%    \begin{macrocode}
%%\DeclareFlexSymbol{\hbar}         {Ord}{MSB}{7E}
\DeclareFlexSymbol{\backepsilon}    {Rel}{MSB}{7F}
%</msabm>
%    \end{macrocode}
%
% \PrintIndex
%
% \Finale

%        (quote the arguments according to the demands of your shell)
%
% Documentation:
%    (a) If flexisym.drv is present:
%           latex flexisym.drv
%    (b) Without flexisym.drv:
%           latex flexisym.dtx; ...
%    The class ltxdoc loads the configuration file ltxdoc.cfg
%    if available. Here you can specify further options, e.g.
%    use A4 as paper format:
%       \PassOptionsToClass{a4paper}{article}
%
%    Programm calls to get the documentation (example):
%       pdflatex flexisym.dtx
%       makeindex -s gind.ist flexisym.idx
%       pdflatex flexisym.dtx
%       makeindex -s gind.ist flexisym.idx
%       pdflatex flexisym.dtx
%
% Installation:
%    TDS:tex/latex/mh/flexisym.sty
%    TDS:tex/latex/mh/cmbase.sym
%    TDS:tex/latex/mh/mathpazo.sym
%    TDS:tex/latex/mh/mathptmx.sym
%    TDS:tex/latex/mh/msabm.sym
%    TDS:doc/latex/mh/flexisym.pdf
%    TDS:source/latex/mh/flexisym.dtx
%
%<*ignore>
\begingroup
  \def\x{LaTeX2e}
\expandafter\endgroup
\ifcase 0\ifx\install y1\fi\expandafter
         \ifx\csname processbatchFile\endcsname\relax\else1\fi
         \ifx\fmtname\x\else 1\fi\relax
\else\csname fi\endcsname
%</ignore>
%<*install>
\input docstrip.tex
\Msg{************************************************************************}
\Msg{* Installation}
\Msg{* Package: flexisym 2007/12/19 v0.96 Flexisym (MH)}
\Msg{************************************************************************}

\keepsilent
\askforoverwritefalse

\preamble

This is a generated file.

Copyright (C) 1997-2003 by Michael J. Downes
Copyright (C) 2007 by Morten Hoegholm <mh.ctan@gmail.com>

This work may be distributed and/or modified under the
conditions of the LaTeX Project Public License, either
version 1.3 of this license or (at your option) any later
version. The latest version of this license is in
   http://www.latex-project.org/lppl.txt
and version 1.3 or later is part of all distributions of
LaTeX version 2005/12/01 or later.

This work has the LPPL maintenance status "maintained".

This Current Maintainer of this work is Morten Hoegholm.

This work consists of the main source file flexisym.dtx
and the derived files
   flexisym.sty, flexisym.pdf, flexisym.ins, flexisym.drv,
   cmbase.sym, mathpazo.sym, mathptmx.sym, msabm.sym.

\endpreamble

\generate{%
  \file{flexisym.ins}{\from{flexisym.dtx}{install}}%
  \file{flexisym.drv}{\from{flexisym.dtx}{driver}}%
  \usedir{tex/latex/mh}%
  \file{flexisym.sty}{\from{flexisym.dtx}{package}}%
  \file{cmbase.sym}{\from{flexisym.dtx}{cmbase}}%
  \file{mathpazo.sym}{\from{flexisym.dtx}{mathpazo}}%
  \file{mathptmx.sym}{\from{flexisym.dtx}{mathptmx}}%
  \file{msabm.sym}{\from{flexisym.dtx}{msabm}}%
}

\obeyspaces
\Msg{************************************************************************}
\Msg{*}
\Msg{* To finish the installation you have to move the following}
\Msg{* files into a directory searched by TeX:}
\Msg{*}
\Msg{*     flexisym.sty, cmbase.sym, mathpazo.sym, mathptmx.sym, msabm.sym}
\Msg{*}
\Msg{* To produce the documentation run the file `flexisym.drv'}
\Msg{* through LaTeX.}
\Msg{*}
\Msg{* Happy TeXing!}
\Msg{*}
\Msg{************************************************************************}

\endbatchfile
%</install>
%<*ignore>
\fi
%</ignore>
%<*driver>
\NeedsTeXFormat{LaTeX2e}
\ProvidesFile{flexisym.drv}%
  [2007/12/19 v0.96 flexisym (MH)]
\documentclass{ltxdoc}
\CodelineIndex
\EnableCrossrefs
\setcounter{IndexColumns}{2}
\providecommand*\pkg[1]{\textsf{#1}}
\providecommand*\cls[1]{\textsf{#1}}
\providecommand*\opt[1]{\texttt{#1}}
\providecommand*\env[1]{\texttt{#1}}
\providecommand*\fn[1]{\texttt{#1}}
\makeatletter
\providecommand{\AmS}{{\protect\AmSfont
  A\kern-.1667em\lower.5ex\hbox{M}\kern-.125emS}}
\providecommand{\AmSfont}{%
  \usefont{OMS}{cmsy}{\if\expandafter\@car\f@series\@nil bb\else m\fi}{n}}
\makeatother
\newenvironment{aside}{\begin{quote}\bfseries}{\end{quote}}
\begin{document}
  \DocInput{flexisym.dtx}
\end{document}
%</driver>
% \fi
%
% \title{The \textsf{flexisym} package}
% \date{2007/12/19 v0.96}
% \author{Morten H\o gholm \\\texttt{mh.ctan@gmail.com}}
%
% \maketitle
%
% \part*{User's guide}
%
% For now, the user's guide is in breqn.
%
% \StopEventually{}
% \part*{Implementation}
% 
% \section{flexisym}
%
%    \begin{macrocode}
%<*package>
\ProvidesPackage{flexisym}[2007/12/19 v0.96]
\let\@xp\expandafter \let\@nx\noexpand
\edef\do{%
  \@nx\AtEndOfPackage{%
    \catcode\number`\"=\number\catcode`\"
    \relax
  }%
}
\do \let\do\relax
\catcode`\"=12
\let\@sym\@gobble
\DeclareOption{robust}{%
  \def\@sym#1{%
    \ifx\protect\@typeset@protect \else\protect#1\@xp\@gobblefour\fi
  }%
}
\def\mg@bin{2}% binary operators
\def\mg@rel{2}% relations
%%\def\mg@nre{B}% negated relations
\def\mg@del{3}% delimiters
%%\def\mg@arr{B}% arrows
\def\mg@acc{0}% accents
\def\mg@cop{3}% cumulative operators (sum, int)
\def\mg@latin{1}% (Latin) letters
\def\mg@greek{1}% (lowercase) Greek
\def\mg@Greek{0}% (capital) Greek
%%\def\mg@bflatin{4}% bold upright Latin letters ?
%%\def\mg@Bbb{B}% blackboard bold
\def\mg@cal{2}% script/calligraphic
%%\def\mg@frak{5}% Fraktur letters
\def\mg@digit{0}% decimal digits % 1 = oldstyle, 0 = capital
\expandafter\let\csname MathChar \endcsname\mathchar
\expandafter\let\csname Delimiter \endcsname\delimiter
\expandafter\let\csname Radical \endcsname\radical
\newcommand{\MathChar}{}
\edef\MathChar{\csname MathChar \endcsname\noexpand\string}
\newcommand{\Delimiter}{}
\edef\Delimiter{\csname Delimiter \endcsname\noexpand\string}
\newcommand{\Radical}{}
\edef\Radical{\csname Radical \endcsname\noexpand\string}
\let\sumlimits\displaylimits
\let\intlimits\nolimits
\let\namelimits\displaylimits
\edef\m@Ord#1#2#3{\csname MathChar \endcsname"0#1#2#3 }
\edef\m@Var#1#2#3{\csname MathChar \endcsname"7#1#2#3 }
\edef\m@Bin#1#2#3{\csname MathChar \endcsname"2#1#2#3 }
\edef\m@Rel#1#2#3{\csname MathChar \endcsname"3#1#2#3 }
\edef\m@Pun#1#2#3{\csname MathChar \endcsname"6#1#2#3 }
\edef\m@COs#1#2#3{\csname MathChar \endcsname"1#1#2#3 \sumlimits}
\edef\m@COi#1#2#3{\csname MathChar \endcsname"1#1#2#3 \intlimits}
\def\delim@a#1#2#3#4{\ifx\relax#1#2#3#4\else#1\fi #2#3#4}
\def\delim@b#1#2#3#4{\ifx\relax#1#2#3#4\else#1\fi }
\def\@tempa{%
  \@nx\@xp\@nx\delim@a\@nx\csname sd@##1##2##3\@nx\endcsname ##1##2##3 }
\edef\m@DeL#1#2#3{\csname Delimiter \endcsname"4\@tempa}
\edef\m@DeR#1#2#3{\csname Delimiter \endcsname"5\@tempa}
\edef\m@DeB#1#2#3{\csname Delimiter \endcsname"0\@tempa}
\edef\m@DeA#1#2#3{\csname Delimiter \endcsname"3\@tempa}
\edef\m@Rad#1#2#3{\csname Radical \endcsname"\@tempa}
\def\do#1#2{\@xp\def\csname sd@#1\endcsname{#2}}
\do{300}{028}
\do{301}{029}
\do{302}{05B}
\do{303}{05D}
\do{304}{262}
\do{305}{263}
\do{306}{264}
\do{307}{265}
\do{308}{266}
\do{309}{267}
\do{30A}{268}
\do{30B}{269}
\do{30C}{26A}
\do{30D}{26B}
\do{30E}{13D}
\do{30F}{26E}
\do{340}{37A}
\do{341}{37B}
\do{33A}{33A}
\do{33B}{33B}
\do{33E}{33E}
\do{33C}{26A}
\do{33D}{26B}
\do{378}{222}
\do{379}{223}
\do{33F}{26C}
\do{37E}{22A}
\do{37F}{22B}
\do{377}{26D}
\do{30F}{26E}
\def\m@Acc#1#2#3#4{\mathaccent"#1#2#3{#4}}
\def\@symAcc{\@sym}
\let\@symtype\@firstofone
\def\@symOrd#1#2{\@symtype\mathord{\OrdSymbol{#2}}}
\def\@symVar{\@symOrd}
\def\@symBin#1#2{\@symtype\mathbin{\OrdSymbol{#2}}}
\def\@symRel#1#2{\@symtype\mathrel{\OrdSymbol{#2}}}
\def\@symPun#1#2{\@symtype\mathpunct{\OrdSymbol{#2}}}
\def\@symCOi#1#2{\@symtype{\mathop{\OrdSymbol{#2}}\intlimits}}
\def\@symCOs#1#2{\@symtype{\mathop{\OrdSymbol{#2}}\sumlimits}}
\def\@symOpe#1#2{\@symtype\mathopen{\OrdSymbol{#2}}}
\def\@symClo#1#2{\@symtype\mathclose{\OrdSymbol{#2}}}
\def\@symDeL#1#2{\@symtype\mathopen{\OrdSymbol{#2}}}
\def\@symDeR#1#2{\@symtype\mathclose{\OrdSymbol{#2}}}
\def\@symDeB#1#2{\@symtype\mathord{\OrdSymbol{#2}}}
\def\@symInn#1#2{\@symtype\mathinner{\OrdSymbol{#2}}}
\def\@xnce#1{\@xp\@nx\csname#1\endcsname}
\let\sym@global\global
\def\DeclareFlexSymbol#1#2#3#4{%
  \begingroup
  \edef\@tempb{\@nx\@sym\@nx#1\@xnce{m@#2}\@xnce{mg@#3}#4}%
  \ifcat\@nx#1\relax
    \sym@global\let#1\@tempb
  \else
    \sym@global\mathcode`#1="8000\relax
    \lccode`\~=`#1\relax
    \lowercase{\sym@global\let~\@tempb}%
  \fi
  \endgroup
}
\def\DeclareFlexCompoundSymbol#1#2#3{%
  \@xp\DeclareRobustCommand\@xp#1\@xp{\csname @sym#2\endcsname#1{#3}}%
  \sym@global\let#1#1\relax
}
\DeclareRobustCommand\textchar{\text@char\textfont}
\DeclareRobustCommand\scriptchar{\text@char\scriptfont}%
\def\text@char@a{\?\endgroup}%
\def\text@char@sym#1#2#3{%
  \begingroup
    \let\@sym\relax % defense against infinite loops
    \the\text@script@char#3%
    \afterassignment\text@char@a
    \chardef\?="%
}
\def\text@char#1#2{\begingroup\check@mathfonts
  \let\text@script@char#1\let\@sym\text@char@sym
  \let\@symtype\@secondoftwo \let\OrdSymbol\@firstofone
  \let\ifmmode\iftrue \everymath{$\@gobble}%$
  \def\mkern{\muskip\z@}\let\mskip\mkern
  \ifcat\relax\noexpand#2#2%
  \else
    \lccode`\~=\expandafter`\string#2\relax
    \lowercase{~}%
  \fi
  \endgroup
}
\providecommand\textprime{}
\DeclareRobustCommand\textprime{\leavevmode
  \raise.8ex\hbox{\text@char\scriptfont\prime}%
}
\@ifundefined{resetMathstrut@}{}{%
  \def\resetMathstrut@{%
    \setbox\z@\hbox{\textchar\vert}%
    \ht\Mathstrutbox@\ht\z@ \dp\Mathstrutbox@\dp\z@
  }%
}
\@ifundefined{rightarrowfill@}{}{%
  \def\rightarrowfill@#1{\m@th\setboxz@h{$#1\relbar$}\ht\z@\z@
    $#1\copy\z@\mkern-6mu\cleaders
    \hbox{$#1\mkern-2mu\box\z@\mkern-2mu$}\hfill
    \mkern-6mu\OrdSymbol{\rightarrow}$}
  \def\leftarrowfill@#1{\m@th\setboxz@h{$#1\relbar$}\ht\z@\z@
    $#1\OrdSymbol{\leftarrow}\mkern-6mu\cleaders
    \hbox{$#1\mkern-2mu\copy\z@\mkern-2mu$}\hfill
    \mkern-6mu\box\z@$}
  \def\leftrightarrowfill@#1{\m@th\setboxz@h{$#1\relbar$}\ht\z@\z@
    $#1\OrdSymbol{\leftarrow}\mkern-6mu\cleaders
    \hbox{$#1\mkern-2mu\box\z@\mkern-2mu$}\hfill
    \mkern-6mu\OrdSymbol{\rightarrow}$}
}
\def\binrel@sym#1#2#3#4#5{%
  \xdef\binrel@@##1{%
    \ifx\m@Ord#2\@nx\@symOrd
    \else\ifx\m@Var#2\@nx\@symVar
    \else\ifx\m@COs#2\@nx\@symCOs
    \else\ifx\m@COi#2\@nx\@symCOi
    \else\ifx\m@Bin#2\@nx\@symBin
    \else\ifx\m@Rel#2\@nx\@symRel
    \else\ifx\m@Pun#2\@nx\@symPun
    \else\@nx\@symErr \fi\fi\fi\fi\fi\fi\fi
  ?{\@nx\OrdSymbol{##1}}}%
}
\def\binrel@a{%
  \def\@symOrd##1##2{\gdef\binrel@@####1{\@symOrd##1{\OrdSymbol{####1}}}}%
  \def\@symVar##1##2{\gdef\binrel@@####1{\@symVar##1{\OrdSymbol{####1}}}}%
  \def\@symCOs##1##2{\gdef\binrel@@####1{\@symCOs##1{\OrdSymbol{####1}}}}%
  \def\@symCOi##1##2{\gdef\binrel@@####1{\@symCOi##1{\OrdSymbol{####1}}}}%
  \def\@symBin##1##2{\gdef\binrel@@####1{\@symBin##1{\OrdSymbol{####1}}}}%
  \def\@symRel##1##2{\gdef\binrel@@####1{\@symRel##1{\OrdSymbol{####1}}}}%
  \def\@symPun##1##2{\gdef\binrel@@####1{\@symPun##1{\OrdSymbol{####1}}}}%
}
\def\binrel@#1{%
  \setbox\z@\hbox{$%
    \let\mathchoice\@gobblethree
    \let\@sym\binrel@sym \binrel@a
    #1$}%
}
\def\@symextension{sym}
\newcommand\usesymbols[1]{%
  \@for\@tempb:=#1\do{%
    \@xp\@onefilewithoptions\@xp{\@tempb}[][]\@symextension
  }%
}
\newcommand\ProvidesSymbols[1]{\ProvidesFile{#1.sym}}
\DeclareRobustCommand{\not}[1]{\@symRel\not{\OrdSymbol{\notRel#1}}}
\DeclareRobustCommand{\OrdSymbol}[1]{%
  \begingroup\mathchars@reset#1\endgroup
}
\def\mathchars@reset{\let\@sym\@sym@ord \let\@symtype\@symtype@ord
  \let\OrdSymbol\relax}
\def\@symtype@ord#1#{}% a strange sort of \@gobble
\def\@sym@ord#1#2{\@xp\@sym@ord@a\string#2\@nil}%
\begingroup
\lccode`\.=`\@ \lowercase{\endgroup
\def\@sym@ord@a#1.}#2#3\@nil#4#5#6{%
  \csname MathChar \endcsname"0%
    \if D#2\@xp\delim@b\csname sd@#4#5#6\endcsname#4#5#6
    \else #4#5#6
    \fi
}
%    \end{macrocode}
%
%
% Before declaring any math characters active, we have to take care of
% a small problem with \pkg{amsmath} v2.x, if it is loaded before
% \pkg{flexisym}. \cs{std@minus} and \cs{std@equal} are defined as
% \begin{verbatim}
% \mathchardef\std@minus\mathcode`\-\relax
% \mathchardef\std@equal\mathcode`\=\relax
% \end{verbatim}
% in \fn{amsmath.sty} and again \cs{AtBeginDocument}. The
% latter is because
% \begin{quote}
%   In case some alternative math fonts are loaded
%   later. [\fn{amsmath.dtx}]
% \end{quote}
% The problem arises because \pkg{flexisym} sets the mathcode of all
% symbols to $32768$ which is illegal for a \cs{mathchardef}. 
%
% We have to remove the assignments from the \cs{AtBeginDocument} hook
% as they will cause an error there.
%    \begin{macrocode}
\@ifpackageloaded{amsmath}{%
  \begingroup
%    \end{macrocode}
% Split the contents of \cs{@begindocumenthook} by reading what we
% search for as a delimited argument and ensure these two assignments
% do not take place. It is questionable if anything reasonable can be
% done to them. In the case of a package such as \pkg{mathpazo} which defines
% \begin{verbatim}
%\DeclareMathSymbol{=}{\mathrel}{upright}{"3D}
% \end{verbatim}
% the \cs{Relbar} will look wrong if we don't use the correct
% symbol. The way to solve this is define additional \fn{.sym} files
% which contain the definition of \cs{relbar} and \cs{Relbar}
% needed. We need those additional files anyway for things like
% \cs{joinord}.
%    \begin{macrocode}
  \long\def\next#1\mathchardef\std@minus\mathcode`\-\relax
                  \mathchardef\std@equal\mathcode`\=\relax#2\flexi@stop{%
    \toks@{#1#2}%
    \xdef\@begindocumenthook{\the\toks@}%
  }%
  \expandafter\next\@begindocumenthook\flexi@stop
  \endgroup
}{}
%    \end{macrocode}
%
% There is problem when using \cs{DeclareMathOperator} as the
% operators defined call a command \cs{newmcodes@} which relies on the
% mathcode of \texttt{-} being less than 32768. We delay the
% definition \cs{AtBeginDocument} in case \pkg{amssymb} hasn't been
% loaded yet.
%    \begin{macrocode}
\AtBeginDocument{%
\def\newmcodes@{%
  \mathcode `\'39\mathcode `\*42\mathcode `\."613A
  \ifnum\mathcode`\-=45
  \else
%    \end{macrocode}
% The extra check. Don't do anything if \texttt{-} is math active.
%    \begin{macrocode}
    \ifnum\mathcode`\-=32768
    \else 
      \mathchardef \std@minus \mathcode `\-\relax
    \fi
  \fi
  \mathcode `\-45 \mathcode `\/47\mathcode `\:"603A\relax 
}%
}
%    \end{macrocode}
%
% And we then continue with the options.
%    \begin{macrocode}
\DeclareOption{mathstyleoff}{\PassOptionsToPackage{mathactivechars}{mathstyle}}
\DeclareOption{cmbase}{\usesymbols{cmbase}}
\DeclareOption{mathpazo}{\usesymbols{mathpazo}}
\DeclareOption{mathptmx}{\usesymbols{mathptmx}}
\ExecuteOptions{cmbase}
\ProcessOptions\relax
\renewcommand{\lnot}{\neg}
\renewcommand{\land}{\wedge}
\renewcommand{\lor}{\vee}
\renewcommand{\le}{\leq}
\renewcommand{\ge}{\geq}
\renewcommand{\ne}{\neq}
\renewcommand{\owns}{\ni}
\renewcommand{\gets}{\leftarrow}
\renewcommand{\to}{\rightarrow}
\renewcommand{\|}{\Vert}
\RequirePackage{mathstyle}
%</package>\endinput
%    \end{macrocode}
%
% \section{cmbase, mathpazo, mathptmx}
%
%
% For each math font package we define a corresponding symbol file
% with extension \fn{sym}. The Computer Modern base is called
% \opt{cmbase} and \opt{mathpazo} and \opt{mathptmx} corresponds to
% the packages. The definitions are almost identical as they mostly
% concern the positions in the math font encodings. Look for
% differences in \cs{joinord}, \cs{relbar} and \cs{Relbar}. If you
% inspect the source code, you'll see that the support for
% \pkg{mathptmx} didn't require any work but I thought it better to
% create a \fn{sym} file to maintain a uniform interface.
%
% \begin{aside}
% Open question on \verb"!" and \verb"?": maybe they
% should have type `Pun' instead of `DeR'.    Need to
% search for uses in math in AMS archives.    Or, maybe add a special
% `Clo' type for them: non-extensible closing delimiter.   
% \end{aside}
% 
% 
% 
% Default mathgroup setup.   
%    \begin{macrocode}
%<*cmbase|mathpazo|mathptmx>
%<cmbase>\ProvidesSymbols{cmbase}[2007/12/19 v0.92]
%<mathpazo>\ProvidesSymbols{mathpazo}[2007/12/19 v0.2]
%<mathptmx>\ProvidesSymbols{mathptmx}[2007/12/19 v0.2]
\@xp\xdef\csname mg@OT1\endcsname{\hexnumber@\symoperators}
\@xp\xdef\csname mg@OML\endcsname{\hexnumber@\symletters}
\@xp\xdef\csname mg@OMS\endcsname{\hexnumber@\symsymbols}
\@xp\xdef\csname mg@OMX\endcsname{\hexnumber@\symlargesymbols}
\gdef\mg@bin{\mg@OMS}
\gdef\mg@del{\mg@OMX}
\xdef\mg@digit{\@xp\@nx\csname mg@OT1\endcsname}
\gdef\mg@latin{\mg@OML}
\global\let\mg@Latin\mg@latin
\global\let\mg@greek\mg@latin
\global\let\mg@Greek\mg@digit
\global\let\mg@rel\mg@bin
\global\let\mg@ord\mg@bin
\global\let\mg@cop\mg@del
%    \end{macrocode}
% 
% 
% Symbols from the 128-character \fn{cmr} encoding.   
% Paren and square bracket delimiters from this encoding are covered
% by the definitions in the \fn{cmex} section, however.   
%    \begin{macrocode}
\DeclareFlexSymbol{!}     {Pun}{OT1}{21}
\DeclareFlexSymbol{+}     {Bin}{OT1}{2B}
\DeclareFlexSymbol{:}     {Rel}{OT1}{3A}
\DeclareFlexSymbol{\colon}{Pun}{OT1}{3A}
\DeclareFlexSymbol{;}     {Pun}{OT1}{3B}
\DeclareFlexSymbol{=}     {Rel}{OT1}{3D}
\DeclareFlexSymbol{?}     {Pun}{OT1}{3F}
%    \end{macrocode}
% \AmS\TeX, and therefore the \pkg{amsmath} package, make the
% uppercase Greek letters class 0 (nonvariable) instead of 7
% (variable), to eliminate the glaring inconsistency with lowercase
% Greek.    (In plain \TeX , \verb"{\bf\Delta}" works, while
% \verb"{\bf\delta}" doesn't.   ) Let us try to make them both
% variable (fonts permitting) instead of nonvariable.   
%    \begin{macrocode}
\DeclareFlexSymbol{\Gamma}  {Var}{Greek}{00}
\DeclareFlexSymbol{\Delta}  {Var}{Greek}{01}
\DeclareFlexSymbol{\Theta}  {Var}{Greek}{02}
\DeclareFlexSymbol{\Lambda} {Var}{Greek}{03}
\DeclareFlexSymbol{\Xi}     {Var}{Greek}{04}
\DeclareFlexSymbol{\Pi}     {Var}{Greek}{05}
\DeclareFlexSymbol{\Sigma}  {Var}{Greek}{06}
\DeclareFlexSymbol{\Upsilon}{Var}{Greek}{07}
\DeclareFlexSymbol{\Phi}    {Var}{Greek}{08}
\DeclareFlexSymbol{\Psi}    {Var}{Greek}{09}
\DeclareFlexSymbol{\Omega}  {Var}{Greek}{0A}
%    \end{macrocode}
% Decimal digits.   
%    \begin{macrocode}
\DeclareFlexSymbol{0}{Var}{digit}{30}
\DeclareFlexSymbol{1}{Var}{digit}{31}
\DeclareFlexSymbol{2}{Var}{digit}{32}
\DeclareFlexSymbol{3}{Var}{digit}{33}
\DeclareFlexSymbol{4}{Var}{digit}{34}
\DeclareFlexSymbol{5}{Var}{digit}{35}
\DeclareFlexSymbol{6}{Var}{digit}{36}
\DeclareFlexSymbol{7}{Var}{digit}{37}
\DeclareFlexSymbol{8}{Var}{digit}{38}
\DeclareFlexSymbol{9}{Var}{digit}{39}
%    \end{macrocode}
% Symbols from the 128-character \fn{cmmi} encoding.   
%    \begin{macrocode}
\DeclareFlexSymbol{,}{Pun}{OML}{3B}
\DeclareFlexSymbol{.}{Ord}{OML}{3A}
\DeclareFlexSymbol{/}{Ord}{OML}{3D}
\DeclareFlexSymbol{<}{Rel}{OML}{3C}
\DeclareFlexSymbol{>}{Rel}{OML}{3E}
%    \end{macrocode}
% To do: make the Var property of lc Greek work properly.   
%    \begin{macrocode}
\DeclareFlexSymbol{\alpha}{Var}{greek}{0B}
\DeclareFlexSymbol{\beta}{Var}{greek}{0C}
\DeclareFlexSymbol{\gamma}{Var}{greek}{0D}
\DeclareFlexSymbol{\delta}{Var}{greek}{0E}
\DeclareFlexSymbol{\epsilon}{Var}{greek}{0F}
\DeclareFlexSymbol{\zeta}{Var}{greek}{10}
\DeclareFlexSymbol{\eta}{Var}{greek}{11}
\DeclareFlexSymbol{\theta}{Var}{greek}{12}
\DeclareFlexSymbol{\iota}{Var}{greek}{13}
\DeclareFlexSymbol{\kappa}{Var}{greek}{14}
\DeclareFlexSymbol{\lambda}{Var}{greek}{15}
\DeclareFlexSymbol{\mu}{Var}{greek}{16}
\DeclareFlexSymbol{\nu}{Var}{greek}{17}
\DeclareFlexSymbol{\xi}{Var}{greek}{18}
\DeclareFlexSymbol{\pi}{Var}{greek}{19}
\DeclareFlexSymbol{\rho}{Var}{greek}{1A}
\DeclareFlexSymbol{\sigma}{Var}{greek}{1B}
\DeclareFlexSymbol{\tau}{Var}{greek}{1C}
\DeclareFlexSymbol{\upsilon}{Var}{greek}{1D}
\DeclareFlexSymbol{\phi}{Var}{greek}{1E}
\DeclareFlexSymbol{\chi}{Var}{greek}{1F}
\DeclareFlexSymbol{\psi}{Var}{greek}{20}
\DeclareFlexSymbol{\omega}{Var}{greek}{21}
\DeclareFlexSymbol{\varepsilon}{Var}{greek}{22}
\DeclareFlexSymbol{\vartheta}{Var}{greek}{23}
\DeclareFlexSymbol{\varpi}{Var}{greek}{24}
\DeclareFlexSymbol{\varrho}{Var}{greek}{25}
\DeclareFlexSymbol{\varsigma}{Var}{greek}{26}
\DeclareFlexSymbol{\varphi}{Var}{greek}{27}
%    \end{macrocode}
% Note that in plain \TeX\  \cs{imath} and \cs{jmath} are
% not variable-font.    But if a \verb"j" changes font to, let's
% say, sans serif or calligraphic, a dotless \verb"j" in the same
% context should change font in the same way.   
%    \begin{macrocode}
\DeclareFlexSymbol{\imath}{Var}{OML}{7B}
\DeclareFlexSymbol{\jmath}{Var}{OML}{7C}
\DeclareFlexSymbol{\ell}{Ord}{OML}{60}
\DeclareFlexSymbol{\wp}{Ord}{OML}{7D}
\DeclareFlexSymbol{\partial}{Ord}{OML}{40}
\DeclareFlexSymbol{\flat}{Ord}{OML}{5B}
\DeclareFlexSymbol{\natural}{Ord}{OML}{5C}
\DeclareFlexSymbol{\sharp}{Ord}{OML}{5D}
\DeclareFlexSymbol{\triangleleft}{Bin}{OML}{2F}
\DeclareFlexSymbol{\triangleright}{Bin}{OML}{2E}
\DeclareFlexSymbol{\star}{Bin}{OML}{3F}
\DeclareFlexSymbol{\smile}{Rel}{OML}{5E}
\DeclareFlexSymbol{\frown}{Rel}{OML}{5F}
\DeclareFlexSymbol{\leftharpoonup}{Rel}{OML}{28}
\DeclareFlexSymbol{\leftharpoondown}{Rel}{OML}{29}
\DeclareFlexSymbol{\rightharpoonup}{Rel}{OML}{2A}
\DeclareFlexSymbol{\rightharpoondown}{Rel}{OML}{2B}
\DeclareFlexSymbol{a}{Var}{latin}{61}
\DeclareFlexSymbol{b}{Var}{latin}{62}
\DeclareFlexSymbol{c}{Var}{latin}{63}
\DeclareFlexSymbol{d}{Var}{latin}{64}
\DeclareFlexSymbol{e}{Var}{latin}{65}
\DeclareFlexSymbol{f}{Var}{latin}{66}
\DeclareFlexSymbol{g}{Var}{latin}{67}
\DeclareFlexSymbol{h}{Var}{latin}{68}
\DeclareFlexSymbol{i}{Var}{latin}{69}
\DeclareFlexSymbol{j}{Var}{latin}{6A}
\DeclareFlexSymbol{k}{Var}{latin}{6B}
\DeclareFlexSymbol{l}{Var}{latin}{6C}
\DeclareFlexSymbol{m}{Var}{latin}{6D}
\DeclareFlexSymbol{n}{Var}{latin}{6E}
\DeclareFlexSymbol{o}{Var}{latin}{6F}
\DeclareFlexSymbol{p}{Var}{latin}{70}
\DeclareFlexSymbol{q}{Var}{latin}{71}
\DeclareFlexSymbol{r}{Var}{latin}{72}
\DeclareFlexSymbol{s}{Var}{latin}{73}
\DeclareFlexSymbol{t}{Var}{latin}{74}
\DeclareFlexSymbol{u}{Var}{latin}{75}
\DeclareFlexSymbol{v}{Var}{latin}{76}
\DeclareFlexSymbol{w}{Var}{latin}{77}
\DeclareFlexSymbol{x}{Var}{latin}{78}
\DeclareFlexSymbol{y}{Var}{latin}{79}
\DeclareFlexSymbol{z}{Var}{latin}{7A}
\DeclareFlexSymbol{A}{Var}{Latin}{41}
\DeclareFlexSymbol{B}{Var}{Latin}{42}
\DeclareFlexSymbol{C}{Var}{Latin}{43}
\DeclareFlexSymbol{D}{Var}{Latin}{44}
\DeclareFlexSymbol{E}{Var}{Latin}{45}
\DeclareFlexSymbol{F}{Var}{Latin}{46}
\DeclareFlexSymbol{G}{Var}{Latin}{47}
\DeclareFlexSymbol{H}{Var}{Latin}{48}
\DeclareFlexSymbol{I}{Var}{Latin}{49}
\DeclareFlexSymbol{J}{Var}{Latin}{4A}
\DeclareFlexSymbol{K}{Var}{Latin}{4B}
\DeclareFlexSymbol{L}{Var}{Latin}{4C}
\DeclareFlexSymbol{M}{Var}{Latin}{4D}
\DeclareFlexSymbol{N}{Var}{Latin}{4E}
\DeclareFlexSymbol{O}{Var}{Latin}{4F}
\DeclareFlexSymbol{P}{Var}{Latin}{50}
\DeclareFlexSymbol{Q}{Var}{Latin}{51}
\DeclareFlexSymbol{R}{Var}{Latin}{52}
\DeclareFlexSymbol{S}{Var}{Latin}{53}
\DeclareFlexSymbol{T}{Var}{Latin}{54}
\DeclareFlexSymbol{U}{Var}{Latin}{55}
\DeclareFlexSymbol{V}{Var}{Latin}{56}
\DeclareFlexSymbol{W}{Var}{Latin}{57}
\DeclareFlexSymbol{X}{Var}{Latin}{58}
\DeclareFlexSymbol{Y}{Var}{Latin}{59}
\DeclareFlexSymbol{Z}{Var}{Latin}{5A}
%    \end{macrocode}
% The \cs{ldotPun} glyph is used in constructing the
% \cs{ldots} symbol.    It is just a period with a different math
% symbol class.    \cs{lhookRel} and \cs{rhookRel} are used
% in a similar way for building hooked arrow symbols.   
%    \begin{macrocode}
\DeclareFlexSymbol{\ldotPun}{Pun}{OML}{3A}
\def\ldotp{\ldotPun}
\DeclareFlexSymbol{\lhookRel}{Rel}{OML}{2C}
\DeclareFlexSymbol{\rhookRel}{Rel}{OML}{2D}
%    \end{macrocode}
% Symbols from the 128-character \fn{cmsy} encoding.   
%    \begin{macrocode}
\DeclareFlexSymbol{*}{Bin}{bin}{03} % \ast
\DeclareFlexSymbol{-}{Bin}{bin}{00}
\DeclareFlexSymbol{|}{Ord}{OMS}{6A}
\DeclareFlexSymbol{\aleph}{Ord}{ord}{40}
\DeclareFlexSymbol{\Re}{Ord}{ord}{3C}
\DeclareFlexSymbol{\Im}{Ord}{ord}{3D}
\DeclareFlexSymbol{\infty}{Ord}{ord}{31}
\DeclareFlexSymbol{\prime}{Ord}{ord}{30}
\DeclareFlexSymbol{\emptyset}{Ord}{ord}{3B}
\DeclareFlexSymbol{\nabla}{Ord}{ord}{72}
\DeclareFlexSymbol{\top}{Ord}{ord}{3E}
\DeclareFlexSymbol{\bot}{Ord}{ord}{3F}
\DeclareFlexSymbol{\triangle}{Ord}{ord}{34}
\DeclareFlexSymbol{\forall}{Ord}{ord}{38}
\DeclareFlexSymbol{\exists}{Ord}{ord}{39}
\DeclareFlexSymbol{\neg}{Ord}{ord}{3A}
\DeclareFlexSymbol{\clubsuit}{Ord}{ord}{7C}
\DeclareFlexSymbol{\diamondsuit}{Ord}{ord}{7D}
\DeclareFlexSymbol{\heartsuit}{Ord}{ord}{7E}
\DeclareFlexSymbol{\spadesuit}{Ord}{ord}{7F}
\DeclareFlexSymbol{\smallint}{COs}{OMS}{73}
%    \end{macrocode}
% Binary operators.   
%    \begin{macrocode}
\DeclareFlexSymbol{\bigtriangleup}{Bin}{bin}{34}
\DeclareFlexSymbol{\bigtriangledown}{Bin}{bin}{35}
\DeclareFlexSymbol{\wedge}{Bin}{bin}{5E}
\DeclareFlexSymbol{\vee}{Bin}{bin}{5F}
\DeclareFlexSymbol{\cap}{Bin}{bin}{5C}
\DeclareFlexSymbol{\cup}{Bin}{bin}{5B}
\DeclareFlexSymbol{\ddagger}{Bin}{bin}{7A}
\DeclareFlexSymbol{\dagger}{Bin}{bin}{79}
\DeclareFlexSymbol{\sqcap}{Bin}{bin}{75}
\DeclareFlexSymbol{\sqcup}{Bin}{bin}{74}
\DeclareFlexSymbol{\uplus}{Bin}{bin}{5D}
\DeclareFlexSymbol{\amalg}{Bin}{bin}{71}
\DeclareFlexSymbol{\diamond}{Bin}{bin}{05}
\DeclareFlexSymbol{\bullet}{Bin}{bin}{0F}
\DeclareFlexSymbol{\wr}{Bin}{bin}{6F}
\DeclareFlexSymbol{\div}{Bin}{bin}{04}
\DeclareFlexSymbol{\odot}{Bin}{bin}{0C}
\DeclareFlexSymbol{\oslash}{Bin}{bin}{0B}
\DeclareFlexSymbol{\otimes}{Bin}{bin}{0A}
\DeclareFlexSymbol{\ominus}{Bin}{bin}{09}
\DeclareFlexSymbol{\oplus}{Bin}{bin}{08}
\DeclareFlexSymbol{\mp}{Bin}{bin}{07}
\DeclareFlexSymbol{\pm}{Bin}{bin}{06}
\DeclareFlexSymbol{\circ}{Bin}{bin}{0E}
\DeclareFlexSymbol{\bigcirc}{Bin}{bin}{0D}
\DeclareFlexSymbol{\setminus}{Bin}{bin}{6E}
\DeclareFlexSymbol{\cdot}{Bin}{bin}{01}
\DeclareFlexSymbol{\ast}{Bin}{bin}{03}
\DeclareFlexSymbol{\times}{Bin}{bin}{02}
%    \end{macrocode}
% Relation symbols.   
%    \begin{macrocode}
\DeclareFlexSymbol{\propto}{Rel}{rel}{2F}
\DeclareFlexSymbol{\sqsubseteq}{Rel}{rel}{76}
\DeclareFlexSymbol{\sqsupseteq}{Rel}{rel}{77}
\DeclareFlexSymbol{\parallel}{Rel}{rel}{6B}
\DeclareFlexSymbol{\mid}{Rel}{rel}{6A}
\DeclareFlexSymbol{\dashv}{Rel}{rel}{61}
\DeclareFlexSymbol{\vdash}{Rel}{rel}{60}
\DeclareFlexSymbol{\nearrow}{Rel}{rel}{25}
\DeclareFlexSymbol{\searrow}{Rel}{rel}{26}
\DeclareFlexSymbol{\nwarrow}{Rel}{rel}{2D}
\DeclareFlexSymbol{\swarrow}{Rel}{rel}{2E}
\DeclareFlexSymbol{\Leftrightarrow}{Rel}{rel}{2C}
\DeclareFlexSymbol{\Leftarrow}{Rel}{rel}{28}
\DeclareFlexSymbol{\Rightarrow}{Rel}{rel}{29}
\DeclareFlexSymbol{\leq}{Rel}{rel}{14}
\DeclareFlexSymbol{\geq}{Rel}{rel}{15}
\DeclareFlexSymbol{\succ}{Rel}{rel}{1F}
\DeclareFlexSymbol{\prec}{Rel}{rel}{1E}
\DeclareFlexSymbol{\approx}{Rel}{rel}{19}
\DeclareFlexSymbol{\succeq}{Rel}{rel}{17}
\DeclareFlexSymbol{\preceq}{Rel}{rel}{16}
\DeclareFlexSymbol{\supset}{Rel}{rel}{1B}
\DeclareFlexSymbol{\subset}{Rel}{rel}{1A}
\DeclareFlexSymbol{\supseteq}{Rel}{rel}{13}
\DeclareFlexSymbol{\subseteq}{Rel}{rel}{12}
\DeclareFlexSymbol{\in}{Rel}{rel}{32}
\DeclareFlexSymbol{\ni}{Rel}{rel}{33}
\DeclareFlexSymbol{\gg}{Rel}{rel}{1D}
\DeclareFlexSymbol{\ll}{Rel}{rel}{1C}
\DeclareFlexSymbol{\leftrightarrow}{Rel}{rel}{24}
\DeclareFlexSymbol{\leftarrow}{Rel}{rel}{20}
\DeclareFlexSymbol{\rightarrow}{Rel}{rel}{21}
\DeclareFlexSymbol{\sim}{Rel}{rel}{18}
\DeclareFlexSymbol{\simeq}{Rel}{rel}{27}
\DeclareFlexSymbol{\perp}{Rel}{rel}{3F}
\DeclareFlexSymbol{\equiv}{Rel}{rel}{11}
\DeclareFlexSymbol{\asymp}{Rel}{rel}{10}
%    \end{macrocode}
% The \cs{notRel} glyph is a special zero-width glyph intended only
% for use in constructing negated symbols.    \cs{mapstoRel} and
% \cs{cdotPun} have similar but more restricted applications.   
%    \begin{macrocode}
\DeclareFlexSymbol{\notRel}{Rel}{rel}{36}
\DeclareFlexSymbol{\mapstoOrd}{Ord}{OMS}{37}
\DeclareFlexSymbol{\cdotOrd}{Ord}{OMS}{01}
\def\cdotp{\mathpunct{\cdotOrd}}
%    \end{macrocode}
% Symbols from the 128-character \fn{cmex} encoding.   
% \verb"COs" stands for `cumulative operator
% (sum-like)'.   
% \verb"COi" stands for `cumulative operator
% (integral-like)'.    These typically differ only in the
% default placement of limits.    \verb"cop" stands for
% `cumulative operator math group'.   
%    \begin{macrocode}
\DeclareFlexSymbol{\coprod}{COs}{cop}{60}
\DeclareFlexSymbol{\bigvee}{COs}{cop}{57}
\DeclareFlexSymbol{\bigwedge}{COs}{cop}{56}
\DeclareFlexSymbol{\biguplus}{COs}{cop}{55}
\DeclareFlexSymbol{\bigcap}{COs}{cop}{54}
\DeclareFlexSymbol{\bigcup}{COs}{cop}{53}
\DeclareFlexSymbol{\int}{COi}{cop}{52}
\DeclareFlexSymbol{\prod}{COs}{cop}{51}
\DeclareFlexSymbol{\sum}{COs}{cop}{50}
\DeclareFlexSymbol{\bigotimes}{COs}{cop}{4E}
\DeclareFlexSymbol{\bigoplus}{COs}{cop}{4C}
\DeclareFlexSymbol{\bigodot}{COs}{cop}{4A}
\DeclareFlexSymbol{\oint}{COi}{cop}{48}
\DeclareFlexSymbol{\bigsqcup}{COs}{cop}{46}
%    \end{macrocode}
% Delimiter symbols.   
% \verb"DeL" stands for `delimiter (left)'.   
% \verb"DeR" stands for `delimiter (right)'.   
% \verb"DeB" stands for `delimiter (bidirectional)'.   
% The principal encoding point for an extensible delimiter is the
% first link in the list of linked sizes as specified in the font metric
% information.   
% For a math encoding such as OT1/OML/OMS/OMX where not all sizes of a
% given delimiter reside in a given font, the extra encoding point for the
% smallest delimiter must be supplied by defining
% \begin{verbatim}
% \sd@GXX
% \end{verbatim}
% where G is the mathgroup and XX is the hexadecimal glyph position.   
%    \begin{macrocode}
\DeclareFlexSymbol{\rangle}{DeR}{del}{0B}
\DeclareFlexSymbol{\langle}{DeL}{del}{0A}
\DeclareFlexSymbol{\rbrace}{DeR}{del}{09}
\DeclareFlexSymbol{\lbrace}{DeL}{del}{08}
\DeclareFlexSymbol{\rceil}{DeR}{del}{07}
\DeclareFlexSymbol{\lceil}{DeL}{del}{06}
\DeclareFlexSymbol{\rfloor}{DeR}{del}{05}
\DeclareFlexSymbol{\lfloor}{DeL}{del}{04}
\DeclareFlexSymbol{(}{DeL}{del}{00}
\DeclareFlexSymbol{)}{DeR}{del}{01}
\DeclareFlexSymbol{[}{DeL}{del}{02}
\DeclareFlexSymbol{]}{DeR}{del}{03}
\DeclareFlexSymbol{\lVert}{DeL}{del}{0D}
\DeclareFlexSymbol{\rVert}{DeR}{del}{0D}
\DeclareFlexSymbol{\lvert}{DeL}{del}{0C}
\DeclareFlexSymbol{\rvert}{DeR}{del}{0C}
\DeclareFlexSymbol{\Vert}{DeB}{del}{0D}
\DeclareFlexSymbol{\vert}{DeB}{del}{0C}
%    \end{macrocode}
% Maybe make the vert bars mathord instead of delimiter, to discourage
% poor usage.   
%    \begin{macrocode}
\DeclareFlexSymbol{|}{DeB}{del}{0C}
\DeclareFlexSymbol{/}{DeB}{del}{0E}
%    \end{macrocode}
% 
% 
% These wacky delimiters need to be supported I guess for
% compabitility reasons.   
% The DeA delimiter type is a special case used only for these
% arrows.   
%    \begin{macrocode}
\DeclareFlexSymbol{\lmoustache}{DeL}{del}{40}
\DeclareFlexSymbol{\rmoustache}{DeR}{del}{41}
\DeclareFlexSymbol{\lgroup}{DeL}{del}{3A}
\DeclareFlexSymbol{\rgroup}{DeR}{del}{3B}
\DeclareFlexSymbol{\bracevert}{DeB}{del}{3E}
\DeclareFlexSymbol{\arrowvert}{DeB}{del}{3C}
\DeclareFlexSymbol{\Arrowvert}{DeB}{del}{3D}
\DeclareFlexSymbol{\uparrow}{DeA}{del}{78}
\DeclareFlexSymbol{\downarrow}{DeA}{del}{79}
\DeclareFlexSymbol{\updownarrow}{DeA}{del}{3F}
\DeclareFlexSymbol{\Uparrow}{DeA}{del}{7E}
\DeclareFlexSymbol{\Downarrow}{DeA}{del}{7F}
\DeclareFlexSymbol{\Updownarrow}{DeA}{del}{77}
\DeclareFlexSymbol{\backslash}{DeB}{del}{0F}
%    \end{macrocode}
% 
% 
% 
% 
% \section{Some compound symbols}
% The following symbols are not robust in standard \LaTeX\ 
% because they use \verb"#" or \cs{mathpalette} (which is not
% robust and contains a \verb"#" in its expansion): \cs{angle},
% \cs{cong}, \cs{notin}, \cs{rightleftharpoons}.   
% 
% In this definition of \cs{hbar}, the symbol is cobbled together
% from a math italic h and the cmr overbar accent glyph.   
%    \begin{macrocode}
\DeclareFlexSymbol{\hbarOrd}{Ord}{OT1}{16}
\DeclareFlexCompoundSymbol{\hbar}{Ord}{\hbarOrd\mkern-9mu h}
%    \end{macrocode}
% For \cs{surd}, the interior symbol gets math class 1
% (cumulative operator) to make the glyph vertically centered on the
% math axis, but the desired horizontal spacing is the spacing for a
% mathord.    (Couldn't it just be class mathopen, though?   )
%    \begin{macrocode}
\DeclareFlexSymbol{\surdOrd}{Ord}{OMS}{70}
\DeclareFlexCompoundSymbol{\surd}{Ord}{\mathop{\surdOrd}}
%    \end{macrocode}
% As shown in this definition of \cs{angle}, rule dimens are not
% allowed to use math-units, unfortunately.   
%    \begin{macrocode}
\DeclareFlexCompoundSymbol{\angle}{Ord}{%
  \vbox{\ialign{%
      $\m@th\scriptstyle##$\crcr
      \notRel\mathrel{\mkern14mu}\crcr
      \noalign{\nointerlineskip}%
      \mkern2.5mu\leaders\hrule \@height.34pt\hfill\mkern2.5mu\crcr
  }}%
}
%    \end{macrocode}
% The \cs{not} function, which is defined in the \pkg{flexisym}
% package, requires a suitably defined \cs{notRel} symbol.   
%    \begin{macrocode}
\DeclareFlexCompoundSymbol{\neq}{Rel}{\not{=}}
%    \end{macrocode}
% .   
%    \begin{macrocode}
\DeclareFlexCompoundSymbol{\mapsto}{Rel}{\mapstoOrd\rightarrow}
%    \end{macrocode}
% The \cs{@vereq} function ends by centering the whole
% construction on the math axis, unlike \cs{buildrel} where the base
% symbol remains at its normal altitude.    Furthermore,
% \cs{@vereq} leaves the math style of the top symbol as given
% instead of downsizing to scriptstyle.   
%    \begin{macrocode}
\DeclareFlexCompoundSymbol{\cong}{Rel}{\mathpalette\@vereq\sim}
%    \end{macrocode}
% The \cs{m@th} in the \fn{fontmath.ltx} definition of
% \cs{notin} is superfluous unless \cs{c@ncel} doesn't include
% it (which was perhaps true in an older version of
% \fn{plain.tex}?).   
%    \begin{macrocode}
\providecommand*\joinord{}
%<cmbase|mathptmx>\renewcommand*\joinord{\mkern-3mu }
%<mathpazo>\renewcommand*\joinord{\mkern-3.45mu }
\DeclareFlexCompoundSymbol{\notin}{Rel}{\mathpalette\c@ncel\in}
\DeclareFlexCompoundSymbol{\rightleftharpoons}{Rel}{\mathpalette\rlh@{}}
\DeclareFlexCompoundSymbol{\doteq}{Rel}{\buildrel\textstyle.\over=}
\DeclareFlexCompoundSymbol{\hookrightarrow}{Rel}{\lhookRel\joinord\rightarrow}
\DeclareFlexCompoundSymbol{\hookleftarrow}{Rel}{\leftarrow\joinord\rhookRel}
\DeclareFlexCompoundSymbol{\bowtie}{Rel}{\triangleright\joinord\triangleleft}
\DeclareFlexCompoundSymbol{\models}{Rel}{\vert\joinord=}
\DeclareFlexCompoundSymbol{\Longrightarrow}{Rel}{\Relbar\joinord\Rightarrow}
\DeclareFlexCompoundSymbol{\longrightarrow}{Rel}{\relbar\joinord\rightarrow}
\DeclareFlexCompoundSymbol{\Longleftarrow}{Rel}{\Leftarrow\joinord\Relbar}
\DeclareFlexCompoundSymbol{\longleftarrow}{Rel}{\leftarrow\joinord\relbar}
\DeclareFlexCompoundSymbol{\longmapsto}{Rel}{\mapstochar\longrightarrow}
\DeclareFlexCompoundSymbol{\longleftrightarrow}{Rel}{\leftarrow\joinord\rightarrow}
\DeclareFlexCompoundSymbol{\Longleftrightarrow}{Rel}{\Leftarrow\joinord\Rightarrow}
%    \end{macrocode}
% Here is what you get from the old definition of \cs{iff}.   
% \begin{verbatim}
% \glue 2.77771 plus 2.77771
% \glue(\thickmuskip) 2.77771 plus 2.77771
% \OMS/cmsy/m/n/10 (
% \hbox(0.0+0.0)x-1.66663
% .\kern -1.66663
% \OMS/cmsy/m/n/10 )
% \penalty 500
% \glue 2.77771 plus 2.77771
% \glue(\thickmuskip) 2.77771 plus 2.77771
% \end{verbatim}
% Looks like it could be simplified slightly.    But it's not so
% easy as it looks to do it without screwing up the line breaking
% possibilities.   
%    \begin{macrocode}
\renewcommand*\iff{%
  \mskip\thickmuskip\Longleftrightarrow\mskip\thickmuskip
}
%    \end{macrocode}
% Some dotly symbols.   
%    \begin{macrocode}
\DeclareFlexCompoundSymbol{\cdots}{Inn}{\cdotp\cdotp\cdotp}%
\DeclareFlexCompoundSymbol{\vdots}{Ord}{%
  \vbox{\baselineskip4\p@ \lineskiplimit\z@
    \kern6\p@\hbox{.}\hbox{.}\hbox{.}}}
\DeclareFlexCompoundSymbol{\ddots}{Inn}{%
  \mkern1mu\raise7\p@
  \vbox{\kern7\p@\hbox{.}}\mkern2mu%
  \raise4\p@\hbox{.}\mkern2mu\raise\p@\hbox{.}\mkern1mu%
}
%    \end{macrocode}
% .   
%    \begin{macrocode}
\def\relbar{\begingroup \def\smash@{tb}% in case amsmath is loaded
    \mathpalette\mathsm@sh{\mathchar"200 }\endgroup}
%    \end{macrocode}
% For \cs{Relbar} we take an equal sign of class $0$ (Ord) from the
% operator family. For \fn{cmr} and \pkg{mathptmx} we know this is
% family $0$.
%    \begin{macrocode}
%<cmbase|mathptmx>\def\Relbar{\mathchar"3D }
%    \end{macrocode}
% For the \pkg{mathpazo} setup we need to use the equal sign from
% \fn{cmr} and so must insert class $0$ and use the symbol from the
% upright symbols.
%    \begin{macrocode}
%<mathpazo>\edef\Relbar{\mathchar\string"\hexnumber@\symupright3D }
%    \end{macrocode}
% Done.
%    \begin{macrocode}
%</cmbase|mathpazo|mathptmx>
%    \end{macrocode}
% Various synonyms such as \cs{le} for \cs{leq} and
% \cs{to} for \cs{rightarrow} are defined in
% \pkg{flexisym} with \cs{def} instead of \cs{let}, for
% slower execution speed but smaller chance of synchronization
% problems.   
%
%
%
%    \begin{macrocode}
%<*msabm>
\ProvidesSymbols{msabm}[2001/09/08 v0.91]
%    \end{macrocode}
%    \begin{macrocode}
\RequirePackage{amsfonts}\relax
%    \end{macrocode}
%    \begin{macrocode}
\@xp\xdef\csname mg@MSA\endcsname{\hexnumber@\symAMSa}%
\@xp\xdef\csname mg@MSB\endcsname{\hexnumber@\symAMSb}%
%    \end{macrocode}
%    \begin{macrocode}
\DeclareFlexSymbol{\boxdot}       {Bin}{MSA}{00}
\DeclareFlexSymbol{\boxplus}      {Bin}{MSA}{01}
\DeclareFlexSymbol{\boxtimes}     {Bin}{MSA}{02}
\DeclareFlexSymbol{\square}       {Ord}{MSA}{03}
\DeclareFlexSymbol{\blacksquare}  {Ord}{MSA}{04}
\DeclareFlexSymbol{\centerdot}    {Bin}{MSA}{05}
\DeclareFlexSymbol{\lozenge}      {Ord}{MSA}{06}
\DeclareFlexSymbol{\blacklozenge} {Ord}{MSA}{07}
\DeclareFlexSymbol{\circlearrowright}   {Rel}{MSA}{08}
\DeclareFlexSymbol{\circlearrowleft}    {Rel}{MSA}{09}
%    \end{macrocode}
% In amsfonts.sty:
%    \begin{macrocode}
%%\DeclareFlexSymbol{\rightleftharpoons}{Rel}{MSA}{0A}
\DeclareFlexSymbol{\leftrightharpoons}  {Rel}{MSA}{0B}
\DeclareFlexSymbol{\boxminus}     {Bin}{MSA}{0C}
\DeclareFlexSymbol{\Vdash}        {Rel}{MSA}{0D}
\DeclareFlexSymbol{\Vvdash}       {Rel}{MSA}{0E}
\DeclareFlexSymbol{\vDash}        {Rel}{MSA}{0F}
\DeclareFlexSymbol{\twoheadrightarrow}  {Rel}{MSA}{10}
\DeclareFlexSymbol{\twoheadleftarrow}   {Rel}{MSA}{11}
\DeclareFlexSymbol{\leftleftarrows}     {Rel}{MSA}{12}
\DeclareFlexSymbol{\rightrightarrows}   {Rel}{MSA}{13}
\DeclareFlexSymbol{\upuparrows}         {Rel}{MSA}{14}
\DeclareFlexSymbol{\downdownarrows} {Rel}{MSA}{15}
\DeclareFlexSymbol{\upharpoonright} {Rel}{MSA}{16}
 \let\restriction\upharpoonright
\DeclareFlexSymbol{\downharpoonright}   {Rel}{MSA}{17}
\DeclareFlexSymbol{\upharpoonleft}  {Rel}{MSA}{18}
\DeclareFlexSymbol{\downharpoonleft}{Rel}{MSA}{19}
\DeclareFlexSymbol{\rightarrowtail} {Rel}{MSA}{1A}
\DeclareFlexSymbol{\leftarrowtail}  {Rel}{MSA}{1B}
\DeclareFlexSymbol{\leftrightarrows}{Rel}{MSA}{1C}
\DeclareFlexSymbol{\rightleftarrows}{Rel}{MSA}{1D}
\DeclareFlexSymbol{\Lsh}            {Rel}{MSA}{1E}
\DeclareFlexSymbol{\Rsh}            {Rel}{MSA}{1F}
\DeclareFlexSymbol{\rightsquigarrow}  {Rel}{MSA}{20}
\DeclareFlexSymbol{\leftrightsquigarrow}{Rel}{MSA}{21}
\DeclareFlexSymbol{\looparrowleft}  {Rel}{MSA}{22}
\DeclareFlexSymbol{\looparrowright} {Rel}{MSA}{23}
\DeclareFlexSymbol{\circeq}       {Rel}{MSA}{24}
\DeclareFlexSymbol{\succsim}      {Rel}{MSA}{25}
\DeclareFlexSymbol{\gtrsim}       {Rel}{MSA}{26}
\DeclareFlexSymbol{\gtrapprox}    {Rel}{MSA}{27}
\DeclareFlexSymbol{\multimap}     {Rel}{MSA}{28}
\DeclareFlexSymbol{\therefore}    {Rel}{MSA}{29}
\DeclareFlexSymbol{\because}      {Rel}{MSA}{2A}
\DeclareFlexSymbol{\doteqdot}     {Rel}{MSA}{2B}
 \let\Doteq\doteqdot
\DeclareFlexSymbol{\triangleq}    {Rel}{MSA}{2C}
\DeclareFlexSymbol{\precsim}      {Rel}{MSA}{2D}
\DeclareFlexSymbol{\lesssim}      {Rel}{MSA}{2E}
\DeclareFlexSymbol{\lessapprox}   {Rel}{MSA}{2F}
\DeclareFlexSymbol{\eqslantless}  {Rel}{MSA}{30}
\DeclareFlexSymbol{\eqslantgtr}   {Rel}{MSA}{31}
\DeclareFlexSymbol{\curlyeqprec}  {Rel}{MSA}{32}
\DeclareFlexSymbol{\curlyeqsucc}  {Rel}{MSA}{33}
\DeclareFlexSymbol{\preccurlyeq}  {Rel}{MSA}{34}
\DeclareFlexSymbol{\leqq}         {Rel}{MSA}{35}
\DeclareFlexSymbol{\leqslant}     {Rel}{MSA}{36}
\DeclareFlexSymbol{\lessgtr}      {Rel}{MSA}{37}
\DeclareFlexSymbol{\backprime}    {Ord}{MSA}{38}
\DeclareFlexSymbol{\risingdotseq} {Rel}{MSA}{3A}
\DeclareFlexSymbol{\fallingdotseq}{Rel}{MSA}{3B}
\DeclareFlexSymbol{\succcurlyeq}  {Rel}{MSA}{3C}
\DeclareFlexSymbol{\geqq}         {Rel}{MSA}{3D}
\DeclareFlexSymbol{\geqslant}     {Rel}{MSA}{3E}
\DeclareFlexSymbol{\gtrless}      {Rel}{MSA}{3F}
%    \end{macrocode}
% in amsfonts.sty
%    \begin{macrocode}
%% \DeclareFlexSymbol{\sqsubset}    {Rel}{MSA}{40}
%% \DeclareFlexSymbol{\sqsupset}    {Rel}{MSA}{41}
\DeclareFlexSymbol{\vartriangleright}{Rel}{MSA}{42}
\DeclareFlexSymbol{\vartriangleleft} {Rel}{MSA}{43}
\DeclareFlexSymbol{\trianglerighteq} {Rel}{MSA}{44}
\DeclareFlexSymbol{\trianglelefteq}  {Rel}{MSA}{45}
\DeclareFlexSymbol{\bigstar}    {Ord}{MSA}{46}
\DeclareFlexSymbol{\between}    {Rel}{MSA}{47}
\DeclareFlexSymbol{\blacktriangledown}  {Ord}{MSA}{48}
\DeclareFlexSymbol{\blacktriangleright} {Rel}{MSA}{49}
\DeclareFlexSymbol{\blacktriangleleft}  {Rel}{MSA}{4A}
\DeclareFlexSymbol{\vartriangle}        {Rel}{MSA}{4D}
\DeclareFlexSymbol{\blacktriangle}      {Ord}{MSA}{4E}
\DeclareFlexSymbol{\triangledown}       {Ord}{MSA}{4F}
\DeclareFlexSymbol{\eqcirc}       {Rel}{MSA}{50}
\DeclareFlexSymbol{\lesseqgtr}    {Rel}{MSA}{51}
\DeclareFlexSymbol{\gtreqless}    {Rel}{MSA}{52}
\DeclareFlexSymbol{\lesseqqgtr}   {Rel}{MSA}{53}
\DeclareFlexSymbol{\gtreqqless}   {Rel}{MSA}{54}
\DeclareFlexSymbol{\Rrightarrow}  {Rel}{MSA}{56}
\DeclareFlexSymbol{\Lleftarrow}   {Rel}{MSA}{57}
\DeclareFlexSymbol{\veebar}       {Bin}{MSA}{59}
\DeclareFlexSymbol{\barwedge}     {Bin}{MSA}{5A}
\DeclareFlexSymbol{\doublebarwedge} {Bin}{MSA}{5B}
%    \end{macrocode}
% In amsfonts.sty
%    \begin{macrocode}
%%\DeclareFlexSymbol{\angle}        {Ord}{MSA}{5C}
\DeclareFlexSymbol{\measuredangle}  {Ord}{MSA}{5D}
\DeclareFlexSymbol{\sphericalangle} {Ord}{MSA}{5E}
\DeclareFlexSymbol{\varpropto}    {Rel}{MSA}{5F}
\DeclareFlexSymbol{\smallsmile}   {Rel}{MSA}{60}
\DeclareFlexSymbol{\smallfrown}   {Rel}{MSA}{61}
\DeclareFlexSymbol{\Subset}       {Rel}{MSA}{62}
\DeclareFlexSymbol{\Supset}       {Rel}{MSA}{63}
\DeclareFlexSymbol{\Cup}          {Bin}{MSA}{64}
 \let\doublecup\Cup
\DeclareFlexSymbol{\Cap}          {Bin}{MSA}{65}
 \let\doublecap\Cap
\DeclareFlexSymbol{\curlywedge}   {Bin}{MSA}{66}
\DeclareFlexSymbol{\curlyvee}     {Bin}{MSA}{67}
\DeclareFlexSymbol{\leftthreetimes} {Bin}{MSA}{68}
\DeclareFlexSymbol{\rightthreetimes}{Bin}{MSA}{69}
\DeclareFlexSymbol{\subseteqq}    {Rel}{MSA}{6A}
\DeclareFlexSymbol{\supseteqq}    {Rel}{MSA}{6B}
\DeclareFlexSymbol{\bumpeq}       {Rel}{MSA}{6C}
\DeclareFlexSymbol{\Bumpeq}       {Rel}{MSA}{6D}
\DeclareFlexSymbol{\lll}          {Rel}{MSA}{6E}
 \let\llless\lll
\DeclareFlexSymbol{\ggg}          {Rel}{MSA}{6F}
 \let\gggtr\ggg
\DeclareFlexSymbol{\circledS}     {Ord}{MSA}{73}
\DeclareFlexSymbol{\pitchfork}    {Rel}{MSA}{74}
\DeclareFlexSymbol{\dotplus}      {Bin}{MSA}{75}
\DeclareFlexSymbol{\backsim}      {Rel}{MSA}{76}
\DeclareFlexSymbol{\backsimeq}    {Rel}{MSA}{77}
\DeclareFlexSymbol{\complement}   {Ord}{MSA}{7B}
\DeclareFlexSymbol{\intercal}     {Bin}{MSA}{7C}
\DeclareFlexSymbol{\circledcirc}  {Bin}{MSA}{7D}
\DeclareFlexSymbol{\circledast}   {Bin}{MSA}{7E}
\DeclareFlexSymbol{\circleddash}  {Bin}{MSA}{7F}
%    \end{macrocode}
%   Begin AMSb declarations
%    \begin{macrocode}
\DeclareFlexSymbol{\lvertneqq}    {Rel}{MSB}{00}
\DeclareFlexSymbol{\gvertneqq}    {Rel}{MSB}{01}
\DeclareFlexSymbol{\nleq}         {Rel}{MSB}{02}
\DeclareFlexSymbol{\ngeq}         {Rel}{MSB}{03}
\DeclareFlexSymbol{\nless}        {Rel}{MSB}{04}
\DeclareFlexSymbol{\ngtr}         {Rel}{MSB}{05}
\DeclareFlexSymbol{\nprec}        {Rel}{MSB}{06}
\DeclareFlexSymbol{\nsucc}        {Rel}{MSB}{07}
\DeclareFlexSymbol{\lneqq}        {Rel}{MSB}{08}
\DeclareFlexSymbol{\gneqq}        {Rel}{MSB}{09}
\DeclareFlexSymbol{\nleqslant}    {Rel}{MSB}{0A}
\DeclareFlexSymbol{\ngeqslant}    {Rel}{MSB}{0B}
\DeclareFlexSymbol{\lneq}         {Rel}{MSB}{0C}
\DeclareFlexSymbol{\gneq}         {Rel}{MSB}{0D}
\DeclareFlexSymbol{\npreceq}      {Rel}{MSB}{0E}
\DeclareFlexSymbol{\nsucceq}      {Rel}{MSB}{0F}
\DeclareFlexSymbol{\precnsim}     {Rel}{MSB}{10}
\DeclareFlexSymbol{\succnsim}     {Rel}{MSB}{11}
\DeclareFlexSymbol{\lnsim}        {Rel}{MSB}{12}
\DeclareFlexSymbol{\gnsim}        {Rel}{MSB}{13}
\DeclareFlexSymbol{\nleqq}        {Rel}{MSB}{14}
\DeclareFlexSymbol{\ngeqq}        {Rel}{MSB}{15}
\DeclareFlexSymbol{\precneqq}     {Rel}{MSB}{16}
\DeclareFlexSymbol{\succneqq}     {Rel}{MSB}{17}
\DeclareFlexSymbol{\precnapprox}  {Rel}{MSB}{18}
\DeclareFlexSymbol{\succnapprox}  {Rel}{MSB}{19}
\DeclareFlexSymbol{\lnapprox}     {Rel}{MSB}{1A}
\DeclareFlexSymbol{\gnapprox}     {Rel}{MSB}{1B}
\DeclareFlexSymbol{\nsim}         {Rel}{MSB}{1C}
\DeclareFlexSymbol{\ncong}        {Rel}{MSB}{1D}
\DeclareFlexSymbol{\diagup}       {Ord}{MSB}{1E}
\DeclareFlexSymbol{\diagdown}     {Ord}{MSB}{1F}
\DeclareFlexSymbol{\varsubsetneq}   {Rel}{MSB}{20}
\DeclareFlexSymbol{\varsupsetneq}   {Rel}{MSB}{21}
\DeclareFlexSymbol{\nsubseteqq}     {Rel}{MSB}{22}
\DeclareFlexSymbol{\nsupseteqq}     {Rel}{MSB}{23}
\DeclareFlexSymbol{\subsetneqq}     {Rel}{MSB}{24}
\DeclareFlexSymbol{\supsetneqq}     {Rel}{MSB}{25}
\DeclareFlexSymbol{\varsubsetneqq}  {Rel}{MSB}{26}
\DeclareFlexSymbol{\varsupsetneqq}  {Rel}{MSB}{27}
\DeclareFlexSymbol{\subsetneq}      {Rel}{MSB}{28}
\DeclareFlexSymbol{\supsetneq}      {Rel}{MSB}{29}
\DeclareFlexSymbol{\nsubseteq}      {Rel}{MSB}{2A}
\DeclareFlexSymbol{\nsupseteq}      {Rel}{MSB}{2B}
\DeclareFlexSymbol{\nparallel}      {Rel}{MSB}{2C}
\DeclareFlexSymbol{\nmid}           {Rel}{MSB}{2D}
\DeclareFlexSymbol{\nshortmid}      {Rel}{MSB}{2E}
\DeclareFlexSymbol{\nshortparallel} {Rel}{MSB}{2F}
\DeclareFlexSymbol{\nvdash}         {Rel}{MSB}{30}
\DeclareFlexSymbol{\nVdash}         {Rel}{MSB}{31}
\DeclareFlexSymbol{\nvDash}         {Rel}{MSB}{32}
\DeclareFlexSymbol{\nVDash}         {Rel}{MSB}{33}
\DeclareFlexSymbol{\ntrianglerighteq}{Rel}{MSB}{34}
\DeclareFlexSymbol{\ntrianglelefteq}{Rel}{MSB}{35}
\DeclareFlexSymbol{\ntriangleleft}  {Rel}{MSB}{36}
\DeclareFlexSymbol{\ntriangleright} {Rel}{MSB}{37}
\DeclareFlexSymbol{\nleftarrow}     {Rel}{MSB}{38}
\DeclareFlexSymbol{\nrightarrow}    {Rel}{MSB}{39}
\DeclareFlexSymbol{\nLeftarrow}     {Rel}{MSB}{3A}
\DeclareFlexSymbol{\nRightarrow}    {Rel}{MSB}{3B}
\DeclareFlexSymbol{\nLeftrightarrow}{Rel}{MSB}{3C}
\DeclareFlexSymbol{\nleftrightarrow}{Rel}{MSB}{3D}
\DeclareFlexSymbol{\divideontimes}  {Bin}{MSB}{3E}
\DeclareFlexSymbol{\varnothing}     {Ord}{MSB}{3F}
\DeclareFlexSymbol{\nexists}        {Ord}{MSB}{40}
\DeclareFlexSymbol{\Finv}           {Ord}{MSB}{60}
\DeclareFlexSymbol{\Game}           {Ord}{MSB}{61}
%    \end{macrocode}
% In amsfonts.sty:
%    \begin{macrocode}
%%\DeclareFlexSymbol{\mho}          {Ord}{MSB}{66}
\DeclareFlexSymbol{\eth}            {Ord}{MSB}{67}
\DeclareFlexSymbol{\eqsim}          {Rel}{MSB}{68}
\DeclareFlexSymbol{\beth}           {Ord}{MSB}{69}
\DeclareFlexSymbol{\gimel}          {Ord}{MSB}{6A}
\DeclareFlexSymbol{\daleth}         {Ord}{MSB}{6B}
\DeclareFlexSymbol{\lessdot}        {Bin}{MSB}{6C}
\DeclareFlexSymbol{\gtrdot}         {Bin}{MSB}{6D}
\DeclareFlexSymbol{\ltimes}         {Bin}{MSB}{6E}
\DeclareFlexSymbol{\rtimes}         {Bin}{MSB}{6F}
\DeclareFlexSymbol{\shortmid}       {Rel}{MSB}{70}
\DeclareFlexSymbol{\shortparallel}  {Rel}{MSB}{71}
\DeclareFlexSymbol{\smallsetminus}  {Bin}{MSB}{72}
\DeclareFlexSymbol{\thicksim}       {Rel}{MSB}{73}
\DeclareFlexSymbol{\thickapprox}    {Rel}{MSB}{74}
\DeclareFlexSymbol{\approxeq}       {Rel}{MSB}{75}
\DeclareFlexSymbol{\succapprox}     {Rel}{MSB}{76}
\DeclareFlexSymbol{\precapprox}     {Rel}{MSB}{77}
\DeclareFlexSymbol{\curvearrowleft} {Rel}{MSB}{78}
\DeclareFlexSymbol{\curvearrowright}{Rel}{MSB}{79}
\DeclareFlexSymbol{\digamma}        {Ord}{MSB}{7A}
\DeclareFlexSymbol{\varkappa}       {Ord}{MSB}{7B}
\DeclareFlexSymbol{\Bbbk}           {Ord}{MSB}{7C}
\DeclareFlexSymbol{\hslash}         {Ord}{MSB}{7D}
%    \end{macrocode}
% In amsfonts.sty:
%    \begin{macrocode}
%%\DeclareFlexSymbol{\hbar}         {Ord}{MSB}{7E}
\DeclareFlexSymbol{\backepsilon}    {Rel}{MSB}{7F}
%</msabm>
%    \end{macrocode}
%
% \PrintIndex
%
% \Finale

%        (quote the arguments according to the demands of your shell)
%
% Documentation:
%    (a) If flexisym.drv is present:
%           latex flexisym.drv
%    (b) Without flexisym.drv:
%           latex flexisym.dtx; ...
%    The class ltxdoc loads the configuration file ltxdoc.cfg
%    if available. Here you can specify further options, e.g.
%    use A4 as paper format:
%       \PassOptionsToClass{a4paper}{article}
%
%    Programm calls to get the documentation (example):
%       pdflatex flexisym.dtx
%       makeindex -s gind.ist flexisym.idx
%       pdflatex flexisym.dtx
%       makeindex -s gind.ist flexisym.idx
%       pdflatex flexisym.dtx
%
% Installation:
%    TDS:tex/latex/mh/flexisym.sty
%    TDS:tex/latex/mh/cmbase.sym
%    TDS:tex/latex/mh/mathpazo.sym
%    TDS:tex/latex/mh/mathptmx.sym
%    TDS:tex/latex/mh/msabm.sym
%    TDS:doc/latex/mh/flexisym.pdf
%    TDS:source/latex/mh/flexisym.dtx
%
%<*ignore>
\begingroup
  \def\x{LaTeX2e}
\expandafter\endgroup
\ifcase 0\ifx\install y1\fi\expandafter
         \ifx\csname processbatchFile\endcsname\relax\else1\fi
         \ifx\fmtname\x\else 1\fi\relax
\else\csname fi\endcsname
%</ignore>
%<*install>
\input docstrip.tex
\Msg{************************************************************************}
\Msg{* Installation}
\Msg{* Package: flexisym 2007/12/19 v0.96 Flexisym (MH)}
\Msg{************************************************************************}

\keepsilent
\askforoverwritefalse

\preamble

This is a generated file.

Copyright (C) 1997-2003 by Michael J. Downes
Copyright (C) 2007 by Morten Hoegholm <mh.ctan@gmail.com>

This work may be distributed and/or modified under the
conditions of the LaTeX Project Public License, either
version 1.3 of this license or (at your option) any later
version. The latest version of this license is in
   http://www.latex-project.org/lppl.txt
and version 1.3 or later is part of all distributions of
LaTeX version 2005/12/01 or later.

This work has the LPPL maintenance status "maintained".

This Current Maintainer of this work is Morten Hoegholm.

This work consists of the main source file flexisym.dtx
and the derived files
   flexisym.sty, flexisym.pdf, flexisym.ins, flexisym.drv,
   cmbase.sym, mathpazo.sym, mathptmx.sym, msabm.sym.

\endpreamble

\generate{%
  \file{flexisym.ins}{\from{flexisym.dtx}{install}}%
  \file{flexisym.drv}{\from{flexisym.dtx}{driver}}%
  \usedir{tex/latex/mh}%
  \file{flexisym.sty}{\from{flexisym.dtx}{package}}%
  \file{cmbase.sym}{\from{flexisym.dtx}{cmbase}}%
  \file{mathpazo.sym}{\from{flexisym.dtx}{mathpazo}}%
  \file{mathptmx.sym}{\from{flexisym.dtx}{mathptmx}}%
  \file{msabm.sym}{\from{flexisym.dtx}{msabm}}%
}

\obeyspaces
\Msg{************************************************************************}
\Msg{*}
\Msg{* To finish the installation you have to move the following}
\Msg{* files into a directory searched by TeX:}
\Msg{*}
\Msg{*     flexisym.sty, cmbase.sym, mathpazo.sym, mathptmx.sym, msabm.sym}
\Msg{*}
\Msg{* To produce the documentation run the file `flexisym.drv'}
\Msg{* through LaTeX.}
\Msg{*}
\Msg{* Happy TeXing!}
\Msg{*}
\Msg{************************************************************************}

\endbatchfile
%</install>
%<*ignore>
\fi
%</ignore>
%<*driver>
\NeedsTeXFormat{LaTeX2e}
\ProvidesFile{flexisym.drv}%
  [2007/12/19 v0.96 flexisym (MH)]
\documentclass{ltxdoc}
\CodelineIndex
\EnableCrossrefs
\setcounter{IndexColumns}{2}
\providecommand*\pkg[1]{\textsf{#1}}
\providecommand*\cls[1]{\textsf{#1}}
\providecommand*\opt[1]{\texttt{#1}}
\providecommand*\env[1]{\texttt{#1}}
\providecommand*\fn[1]{\texttt{#1}}
\makeatletter
\providecommand{\AmS}{{\protect\AmSfont
  A\kern-.1667em\lower.5ex\hbox{M}\kern-.125emS}}
\providecommand{\AmSfont}{%
  \usefont{OMS}{cmsy}{\if\expandafter\@car\f@series\@nil bb\else m\fi}{n}}
\makeatother
\newenvironment{aside}{\begin{quote}\bfseries}{\end{quote}}
\begin{document}
  \DocInput{flexisym.dtx}
\end{document}
%</driver>
% \fi
%
% \title{The \textsf{flexisym} package}
% \date{2007/12/19 v0.96}
% \author{Morten H\o gholm \\\texttt{mh.ctan@gmail.com}}
%
% \maketitle
%
% \part*{User's guide}
%
% For now, the user's guide is in breqn.
%
% \StopEventually{}
% \part*{Implementation}
% 
% \section{flexisym}
%
%    \begin{macrocode}
%<*package>
\ProvidesPackage{flexisym}[2007/12/19 v0.96]
\let\@xp\expandafter \let\@nx\noexpand
\edef\do{%
  \@nx\AtEndOfPackage{%
    \catcode\number`\"=\number\catcode`\"
    \relax
  }%
}
\do \let\do\relax
\catcode`\"=12
\let\@sym\@gobble
\DeclareOption{robust}{%
  \def\@sym#1{%
    \ifx\protect\@typeset@protect \else\protect#1\@xp\@gobblefour\fi
  }%
}
\def\mg@bin{2}% binary operators
\def\mg@rel{2}% relations
%%\def\mg@nre{B}% negated relations
\def\mg@del{3}% delimiters
%%\def\mg@arr{B}% arrows
\def\mg@acc{0}% accents
\def\mg@cop{3}% cumulative operators (sum, int)
\def\mg@latin{1}% (Latin) letters
\def\mg@greek{1}% (lowercase) Greek
\def\mg@Greek{0}% (capital) Greek
%%\def\mg@bflatin{4}% bold upright Latin letters ?
%%\def\mg@Bbb{B}% blackboard bold
\def\mg@cal{2}% script/calligraphic
%%\def\mg@frak{5}% Fraktur letters
\def\mg@digit{0}% decimal digits % 1 = oldstyle, 0 = capital
\expandafter\let\csname MathChar \endcsname\mathchar
\expandafter\let\csname Delimiter \endcsname\delimiter
\expandafter\let\csname Radical \endcsname\radical
\newcommand{\MathChar}{}
\edef\MathChar{\csname MathChar \endcsname\noexpand\string}
\newcommand{\Delimiter}{}
\edef\Delimiter{\csname Delimiter \endcsname\noexpand\string}
\newcommand{\Radical}{}
\edef\Radical{\csname Radical \endcsname\noexpand\string}
\let\sumlimits\displaylimits
\let\intlimits\nolimits
\let\namelimits\displaylimits
\edef\m@Ord#1#2#3{\csname MathChar \endcsname"0#1#2#3 }
\edef\m@Var#1#2#3{\csname MathChar \endcsname"7#1#2#3 }
\edef\m@Bin#1#2#3{\csname MathChar \endcsname"2#1#2#3 }
\edef\m@Rel#1#2#3{\csname MathChar \endcsname"3#1#2#3 }
\edef\m@Pun#1#2#3{\csname MathChar \endcsname"6#1#2#3 }
\edef\m@COs#1#2#3{\csname MathChar \endcsname"1#1#2#3 \sumlimits}
\edef\m@COi#1#2#3{\csname MathChar \endcsname"1#1#2#3 \intlimits}
\def\delim@a#1#2#3#4{\ifx\relax#1#2#3#4\else#1\fi #2#3#4}
\def\delim@b#1#2#3#4{\ifx\relax#1#2#3#4\else#1\fi }
\def\@tempa{%
  \@nx\@xp\@nx\delim@a\@nx\csname sd@##1##2##3\@nx\endcsname ##1##2##3 }
\edef\m@DeL#1#2#3{\csname Delimiter \endcsname"4\@tempa}
\edef\m@DeR#1#2#3{\csname Delimiter \endcsname"5\@tempa}
\edef\m@DeB#1#2#3{\csname Delimiter \endcsname"0\@tempa}
\edef\m@DeA#1#2#3{\csname Delimiter \endcsname"3\@tempa}
\edef\m@Rad#1#2#3{\csname Radical \endcsname"\@tempa}
\def\do#1#2{\@xp\def\csname sd@#1\endcsname{#2}}
\do{300}{028}
\do{301}{029}
\do{302}{05B}
\do{303}{05D}
\do{304}{262}
\do{305}{263}
\do{306}{264}
\do{307}{265}
\do{308}{266}
\do{309}{267}
\do{30A}{268}
\do{30B}{269}
\do{30C}{26A}
\do{30D}{26B}
\do{30E}{13D}
\do{30F}{26E}
\do{340}{37A}
\do{341}{37B}
\do{33A}{33A}
\do{33B}{33B}
\do{33E}{33E}
\do{33C}{26A}
\do{33D}{26B}
\do{378}{222}
\do{379}{223}
\do{33F}{26C}
\do{37E}{22A}
\do{37F}{22B}
\do{377}{26D}
\do{30F}{26E}
\def\m@Acc#1#2#3#4{\mathaccent"#1#2#3{#4}}
\def\@symAcc{\@sym}
\let\@symtype\@firstofone
\def\@symOrd#1#2{\@symtype\mathord{\OrdSymbol{#2}}}
\def\@symVar{\@symOrd}
\def\@symBin#1#2{\@symtype\mathbin{\OrdSymbol{#2}}}
\def\@symRel#1#2{\@symtype\mathrel{\OrdSymbol{#2}}}
\def\@symPun#1#2{\@symtype\mathpunct{\OrdSymbol{#2}}}
\def\@symCOi#1#2{\@symtype{\mathop{\OrdSymbol{#2}}\intlimits}}
\def\@symCOs#1#2{\@symtype{\mathop{\OrdSymbol{#2}}\sumlimits}}
\def\@symOpe#1#2{\@symtype\mathopen{\OrdSymbol{#2}}}
\def\@symClo#1#2{\@symtype\mathclose{\OrdSymbol{#2}}}
\def\@symDeL#1#2{\@symtype\mathopen{\OrdSymbol{#2}}}
\def\@symDeR#1#2{\@symtype\mathclose{\OrdSymbol{#2}}}
\def\@symDeB#1#2{\@symtype\mathord{\OrdSymbol{#2}}}
\def\@symInn#1#2{\@symtype\mathinner{\OrdSymbol{#2}}}
\def\@xnce#1{\@xp\@nx\csname#1\endcsname}
\let\sym@global\global
\def\DeclareFlexSymbol#1#2#3#4{%
  \begingroup
  \edef\@tempb{\@nx\@sym\@nx#1\@xnce{m@#2}\@xnce{mg@#3}#4}%
  \ifcat\@nx#1\relax
    \sym@global\let#1\@tempb
  \else
    \sym@global\mathcode`#1="8000\relax
    \lccode`\~=`#1\relax
    \lowercase{\sym@global\let~\@tempb}%
  \fi
  \endgroup
}
\def\DeclareFlexCompoundSymbol#1#2#3{%
  \@xp\DeclareRobustCommand\@xp#1\@xp{\csname @sym#2\endcsname#1{#3}}%
  \sym@global\let#1#1\relax
}
\DeclareRobustCommand\textchar{\text@char\textfont}
\DeclareRobustCommand\scriptchar{\text@char\scriptfont}%
\def\text@char@a{\?\endgroup}%
\def\text@char@sym#1#2#3{%
  \begingroup
    \let\@sym\relax % defense against infinite loops
    \the\text@script@char#3%
    \afterassignment\text@char@a
    \chardef\?="%
}
\def\text@char#1#2{\begingroup\check@mathfonts
  \let\text@script@char#1\let\@sym\text@char@sym
  \let\@symtype\@secondoftwo \let\OrdSymbol\@firstofone
  \let\ifmmode\iftrue \everymath{$\@gobble}%$
  \def\mkern{\muskip\z@}\let\mskip\mkern
  \ifcat\relax\noexpand#2#2%
  \else
    \lccode`\~=\expandafter`\string#2\relax
    \lowercase{~}%
  \fi
  \endgroup
}
\providecommand\textprime{}
\DeclareRobustCommand\textprime{\leavevmode
  \raise.8ex\hbox{\text@char\scriptfont\prime}%
}
\@ifundefined{resetMathstrut@}{}{%
  \def\resetMathstrut@{%
    \setbox\z@\hbox{\textchar\vert}%
    \ht\Mathstrutbox@\ht\z@ \dp\Mathstrutbox@\dp\z@
  }%
}
\@ifundefined{rightarrowfill@}{}{%
  \def\rightarrowfill@#1{\m@th\setboxz@h{$#1\relbar$}\ht\z@\z@
    $#1\copy\z@\mkern-6mu\cleaders
    \hbox{$#1\mkern-2mu\box\z@\mkern-2mu$}\hfill
    \mkern-6mu\OrdSymbol{\rightarrow}$}
  \def\leftarrowfill@#1{\m@th\setboxz@h{$#1\relbar$}\ht\z@\z@
    $#1\OrdSymbol{\leftarrow}\mkern-6mu\cleaders
    \hbox{$#1\mkern-2mu\copy\z@\mkern-2mu$}\hfill
    \mkern-6mu\box\z@$}
  \def\leftrightarrowfill@#1{\m@th\setboxz@h{$#1\relbar$}\ht\z@\z@
    $#1\OrdSymbol{\leftarrow}\mkern-6mu\cleaders
    \hbox{$#1\mkern-2mu\box\z@\mkern-2mu$}\hfill
    \mkern-6mu\OrdSymbol{\rightarrow}$}
}
\def\binrel@sym#1#2#3#4#5{%
  \xdef\binrel@@##1{%
    \ifx\m@Ord#2\@nx\@symOrd
    \else\ifx\m@Var#2\@nx\@symVar
    \else\ifx\m@COs#2\@nx\@symCOs
    \else\ifx\m@COi#2\@nx\@symCOi
    \else\ifx\m@Bin#2\@nx\@symBin
    \else\ifx\m@Rel#2\@nx\@symRel
    \else\ifx\m@Pun#2\@nx\@symPun
    \else\@nx\@symErr \fi\fi\fi\fi\fi\fi\fi
  ?{\@nx\OrdSymbol{##1}}}%
}
\def\binrel@a{%
  \def\@symOrd##1##2{\gdef\binrel@@####1{\@symOrd##1{\OrdSymbol{####1}}}}%
  \def\@symVar##1##2{\gdef\binrel@@####1{\@symVar##1{\OrdSymbol{####1}}}}%
  \def\@symCOs##1##2{\gdef\binrel@@####1{\@symCOs##1{\OrdSymbol{####1}}}}%
  \def\@symCOi##1##2{\gdef\binrel@@####1{\@symCOi##1{\OrdSymbol{####1}}}}%
  \def\@symBin##1##2{\gdef\binrel@@####1{\@symBin##1{\OrdSymbol{####1}}}}%
  \def\@symRel##1##2{\gdef\binrel@@####1{\@symRel##1{\OrdSymbol{####1}}}}%
  \def\@symPun##1##2{\gdef\binrel@@####1{\@symPun##1{\OrdSymbol{####1}}}}%
}
\def\binrel@#1{%
  \setbox\z@\hbox{$%
    \let\mathchoice\@gobblethree
    \let\@sym\binrel@sym \binrel@a
    #1$}%
}
\def\@symextension{sym}
\newcommand\usesymbols[1]{%
  \@for\@tempb:=#1\do{%
    \@xp\@onefilewithoptions\@xp{\@tempb}[][]\@symextension
  }%
}
\newcommand\ProvidesSymbols[1]{\ProvidesFile{#1.sym}}
\DeclareRobustCommand{\not}[1]{\@symRel\not{\OrdSymbol{\notRel#1}}}
\DeclareRobustCommand{\OrdSymbol}[1]{%
  \begingroup\mathchars@reset#1\endgroup
}
\def\mathchars@reset{\let\@sym\@sym@ord \let\@symtype\@symtype@ord
  \let\OrdSymbol\relax}
\def\@symtype@ord#1#{}% a strange sort of \@gobble
\def\@sym@ord#1#2{\@xp\@sym@ord@a\string#2\@nil}%
\begingroup
\lccode`\.=`\@ \lowercase{\endgroup
\def\@sym@ord@a#1.}#2#3\@nil#4#5#6{%
  \csname MathChar \endcsname"0%
    \if D#2\@xp\delim@b\csname sd@#4#5#6\endcsname#4#5#6
    \else #4#5#6
    \fi
}
%    \end{macrocode}
%
%
% Before declaring any math characters active, we have to take care of
% a small problem with \pkg{amsmath} v2.x, if it is loaded before
% \pkg{flexisym}. \cs{std@minus} and \cs{std@equal} are defined as
% \begin{verbatim}
% \mathchardef\std@minus\mathcode`\-\relax
% \mathchardef\std@equal\mathcode`\=\relax
% \end{verbatim}
% in \fn{amsmath.sty} and again \cs{AtBeginDocument}. The
% latter is because
% \begin{quote}
%   In case some alternative math fonts are loaded
%   later. [\fn{amsmath.dtx}]
% \end{quote}
% The problem arises because \pkg{flexisym} sets the mathcode of all
% symbols to $32768$ which is illegal for a \cs{mathchardef}. 
%
% We have to remove the assignments from the \cs{AtBeginDocument} hook
% as they will cause an error there.
%    \begin{macrocode}
\@ifpackageloaded{amsmath}{%
  \begingroup
%    \end{macrocode}
% Split the contents of \cs{@begindocumenthook} by reading what we
% search for as a delimited argument and ensure these two assignments
% do not take place. It is questionable if anything reasonable can be
% done to them. In the case of a package such as \pkg{mathpazo} which defines
% \begin{verbatim}
%\DeclareMathSymbol{=}{\mathrel}{upright}{"3D}
% \end{verbatim}
% the \cs{Relbar} will look wrong if we don't use the correct
% symbol. The way to solve this is define additional \fn{.sym} files
% which contain the definition of \cs{relbar} and \cs{Relbar}
% needed. We need those additional files anyway for things like
% \cs{joinord}.
%    \begin{macrocode}
  \long\def\next#1\mathchardef\std@minus\mathcode`\-\relax
                  \mathchardef\std@equal\mathcode`\=\relax#2\flexi@stop{%
    \toks@{#1#2}%
    \xdef\@begindocumenthook{\the\toks@}%
  }%
  \expandafter\next\@begindocumenthook\flexi@stop
  \endgroup
}{}
%    \end{macrocode}
%
% There is problem when using \cs{DeclareMathOperator} as the
% operators defined call a command \cs{newmcodes@} which relies on the
% mathcode of \texttt{-} being less than 32768. We delay the
% definition \cs{AtBeginDocument} in case \pkg{amssymb} hasn't been
% loaded yet.
%    \begin{macrocode}
\AtBeginDocument{%
\def\newmcodes@{%
  \mathcode `\'39\mathcode `\*42\mathcode `\."613A
  \ifnum\mathcode`\-=45
  \else
%    \end{macrocode}
% The extra check. Don't do anything if \texttt{-} is math active.
%    \begin{macrocode}
    \ifnum\mathcode`\-=32768
    \else 
      \mathchardef \std@minus \mathcode `\-\relax
    \fi
  \fi
  \mathcode `\-45 \mathcode `\/47\mathcode `\:"603A\relax 
}%
}
%    \end{macrocode}
%
% And we then continue with the options.
%    \begin{macrocode}
\DeclareOption{mathstyleoff}{\PassOptionsToPackage{mathactivechars}{mathstyle}}
\DeclareOption{cmbase}{\usesymbols{cmbase}}
\DeclareOption{mathpazo}{\usesymbols{mathpazo}}
\DeclareOption{mathptmx}{\usesymbols{mathptmx}}
\ExecuteOptions{cmbase}
\ProcessOptions\relax
\renewcommand{\lnot}{\neg}
\renewcommand{\land}{\wedge}
\renewcommand{\lor}{\vee}
\renewcommand{\le}{\leq}
\renewcommand{\ge}{\geq}
\renewcommand{\ne}{\neq}
\renewcommand{\owns}{\ni}
\renewcommand{\gets}{\leftarrow}
\renewcommand{\to}{\rightarrow}
\renewcommand{\|}{\Vert}
\RequirePackage{mathstyle}
%</package>\endinput
%    \end{macrocode}
%
% \section{cmbase, mathpazo, mathptmx}
%
%
% For each math font package we define a corresponding symbol file
% with extension \fn{sym}. The Computer Modern base is called
% \opt{cmbase} and \opt{mathpazo} and \opt{mathptmx} corresponds to
% the packages. The definitions are almost identical as they mostly
% concern the positions in the math font encodings. Look for
% differences in \cs{joinord}, \cs{relbar} and \cs{Relbar}. If you
% inspect the source code, you'll see that the support for
% \pkg{mathptmx} didn't require any work but I thought it better to
% create a \fn{sym} file to maintain a uniform interface.
%
% \begin{aside}
% Open question on \verb"!" and \verb"?": maybe they
% should have type `Pun' instead of `DeR'.    Need to
% search for uses in math in AMS archives.    Or, maybe add a special
% `Clo' type for them: non-extensible closing delimiter.   
% \end{aside}
% 
% 
% 
% Default mathgroup setup.   
%    \begin{macrocode}
%<*cmbase|mathpazo|mathptmx>
%<cmbase>\ProvidesSymbols{cmbase}[2007/12/19 v0.92]
%<mathpazo>\ProvidesSymbols{mathpazo}[2007/12/19 v0.2]
%<mathptmx>\ProvidesSymbols{mathptmx}[2007/12/19 v0.2]
\@xp\xdef\csname mg@OT1\endcsname{\hexnumber@\symoperators}
\@xp\xdef\csname mg@OML\endcsname{\hexnumber@\symletters}
\@xp\xdef\csname mg@OMS\endcsname{\hexnumber@\symsymbols}
\@xp\xdef\csname mg@OMX\endcsname{\hexnumber@\symlargesymbols}
\gdef\mg@bin{\mg@OMS}
\gdef\mg@del{\mg@OMX}
\xdef\mg@digit{\@xp\@nx\csname mg@OT1\endcsname}
\gdef\mg@latin{\mg@OML}
\global\let\mg@Latin\mg@latin
\global\let\mg@greek\mg@latin
\global\let\mg@Greek\mg@digit
\global\let\mg@rel\mg@bin
\global\let\mg@ord\mg@bin
\global\let\mg@cop\mg@del
%    \end{macrocode}
% 
% 
% Symbols from the 128-character \fn{cmr} encoding.   
% Paren and square bracket delimiters from this encoding are covered
% by the definitions in the \fn{cmex} section, however.   
%    \begin{macrocode}
\DeclareFlexSymbol{!}     {Pun}{OT1}{21}
\DeclareFlexSymbol{+}     {Bin}{OT1}{2B}
\DeclareFlexSymbol{:}     {Rel}{OT1}{3A}
\DeclareFlexSymbol{\colon}{Pun}{OT1}{3A}
\DeclareFlexSymbol{;}     {Pun}{OT1}{3B}
\DeclareFlexSymbol{=}     {Rel}{OT1}{3D}
\DeclareFlexSymbol{?}     {Pun}{OT1}{3F}
%    \end{macrocode}
% \AmS\TeX, and therefore the \pkg{amsmath} package, make the
% uppercase Greek letters class 0 (nonvariable) instead of 7
% (variable), to eliminate the glaring inconsistency with lowercase
% Greek.    (In plain \TeX , \verb"{\bf\Delta}" works, while
% \verb"{\bf\delta}" doesn't.   ) Let us try to make them both
% variable (fonts permitting) instead of nonvariable.   
%    \begin{macrocode}
\DeclareFlexSymbol{\Gamma}  {Var}{Greek}{00}
\DeclareFlexSymbol{\Delta}  {Var}{Greek}{01}
\DeclareFlexSymbol{\Theta}  {Var}{Greek}{02}
\DeclareFlexSymbol{\Lambda} {Var}{Greek}{03}
\DeclareFlexSymbol{\Xi}     {Var}{Greek}{04}
\DeclareFlexSymbol{\Pi}     {Var}{Greek}{05}
\DeclareFlexSymbol{\Sigma}  {Var}{Greek}{06}
\DeclareFlexSymbol{\Upsilon}{Var}{Greek}{07}
\DeclareFlexSymbol{\Phi}    {Var}{Greek}{08}
\DeclareFlexSymbol{\Psi}    {Var}{Greek}{09}
\DeclareFlexSymbol{\Omega}  {Var}{Greek}{0A}
%    \end{macrocode}
% Decimal digits.   
%    \begin{macrocode}
\DeclareFlexSymbol{0}{Var}{digit}{30}
\DeclareFlexSymbol{1}{Var}{digit}{31}
\DeclareFlexSymbol{2}{Var}{digit}{32}
\DeclareFlexSymbol{3}{Var}{digit}{33}
\DeclareFlexSymbol{4}{Var}{digit}{34}
\DeclareFlexSymbol{5}{Var}{digit}{35}
\DeclareFlexSymbol{6}{Var}{digit}{36}
\DeclareFlexSymbol{7}{Var}{digit}{37}
\DeclareFlexSymbol{8}{Var}{digit}{38}
\DeclareFlexSymbol{9}{Var}{digit}{39}
%    \end{macrocode}
% Symbols from the 128-character \fn{cmmi} encoding.   
%    \begin{macrocode}
\DeclareFlexSymbol{,}{Pun}{OML}{3B}
\DeclareFlexSymbol{.}{Ord}{OML}{3A}
\DeclareFlexSymbol{/}{Ord}{OML}{3D}
\DeclareFlexSymbol{<}{Rel}{OML}{3C}
\DeclareFlexSymbol{>}{Rel}{OML}{3E}
%    \end{macrocode}
% To do: make the Var property of lc Greek work properly.   
%    \begin{macrocode}
\DeclareFlexSymbol{\alpha}{Var}{greek}{0B}
\DeclareFlexSymbol{\beta}{Var}{greek}{0C}
\DeclareFlexSymbol{\gamma}{Var}{greek}{0D}
\DeclareFlexSymbol{\delta}{Var}{greek}{0E}
\DeclareFlexSymbol{\epsilon}{Var}{greek}{0F}
\DeclareFlexSymbol{\zeta}{Var}{greek}{10}
\DeclareFlexSymbol{\eta}{Var}{greek}{11}
\DeclareFlexSymbol{\theta}{Var}{greek}{12}
\DeclareFlexSymbol{\iota}{Var}{greek}{13}
\DeclareFlexSymbol{\kappa}{Var}{greek}{14}
\DeclareFlexSymbol{\lambda}{Var}{greek}{15}
\DeclareFlexSymbol{\mu}{Var}{greek}{16}
\DeclareFlexSymbol{\nu}{Var}{greek}{17}
\DeclareFlexSymbol{\xi}{Var}{greek}{18}
\DeclareFlexSymbol{\pi}{Var}{greek}{19}
\DeclareFlexSymbol{\rho}{Var}{greek}{1A}
\DeclareFlexSymbol{\sigma}{Var}{greek}{1B}
\DeclareFlexSymbol{\tau}{Var}{greek}{1C}
\DeclareFlexSymbol{\upsilon}{Var}{greek}{1D}
\DeclareFlexSymbol{\phi}{Var}{greek}{1E}
\DeclareFlexSymbol{\chi}{Var}{greek}{1F}
\DeclareFlexSymbol{\psi}{Var}{greek}{20}
\DeclareFlexSymbol{\omega}{Var}{greek}{21}
\DeclareFlexSymbol{\varepsilon}{Var}{greek}{22}
\DeclareFlexSymbol{\vartheta}{Var}{greek}{23}
\DeclareFlexSymbol{\varpi}{Var}{greek}{24}
\DeclareFlexSymbol{\varrho}{Var}{greek}{25}
\DeclareFlexSymbol{\varsigma}{Var}{greek}{26}
\DeclareFlexSymbol{\varphi}{Var}{greek}{27}
%    \end{macrocode}
% Note that in plain \TeX\  \cs{imath} and \cs{jmath} are
% not variable-font.    But if a \verb"j" changes font to, let's
% say, sans serif or calligraphic, a dotless \verb"j" in the same
% context should change font in the same way.   
%    \begin{macrocode}
\DeclareFlexSymbol{\imath}{Var}{OML}{7B}
\DeclareFlexSymbol{\jmath}{Var}{OML}{7C}
\DeclareFlexSymbol{\ell}{Ord}{OML}{60}
\DeclareFlexSymbol{\wp}{Ord}{OML}{7D}
\DeclareFlexSymbol{\partial}{Ord}{OML}{40}
\DeclareFlexSymbol{\flat}{Ord}{OML}{5B}
\DeclareFlexSymbol{\natural}{Ord}{OML}{5C}
\DeclareFlexSymbol{\sharp}{Ord}{OML}{5D}
\DeclareFlexSymbol{\triangleleft}{Bin}{OML}{2F}
\DeclareFlexSymbol{\triangleright}{Bin}{OML}{2E}
\DeclareFlexSymbol{\star}{Bin}{OML}{3F}
\DeclareFlexSymbol{\smile}{Rel}{OML}{5E}
\DeclareFlexSymbol{\frown}{Rel}{OML}{5F}
\DeclareFlexSymbol{\leftharpoonup}{Rel}{OML}{28}
\DeclareFlexSymbol{\leftharpoondown}{Rel}{OML}{29}
\DeclareFlexSymbol{\rightharpoonup}{Rel}{OML}{2A}
\DeclareFlexSymbol{\rightharpoondown}{Rel}{OML}{2B}
\DeclareFlexSymbol{a}{Var}{latin}{61}
\DeclareFlexSymbol{b}{Var}{latin}{62}
\DeclareFlexSymbol{c}{Var}{latin}{63}
\DeclareFlexSymbol{d}{Var}{latin}{64}
\DeclareFlexSymbol{e}{Var}{latin}{65}
\DeclareFlexSymbol{f}{Var}{latin}{66}
\DeclareFlexSymbol{g}{Var}{latin}{67}
\DeclareFlexSymbol{h}{Var}{latin}{68}
\DeclareFlexSymbol{i}{Var}{latin}{69}
\DeclareFlexSymbol{j}{Var}{latin}{6A}
\DeclareFlexSymbol{k}{Var}{latin}{6B}
\DeclareFlexSymbol{l}{Var}{latin}{6C}
\DeclareFlexSymbol{m}{Var}{latin}{6D}
\DeclareFlexSymbol{n}{Var}{latin}{6E}
\DeclareFlexSymbol{o}{Var}{latin}{6F}
\DeclareFlexSymbol{p}{Var}{latin}{70}
\DeclareFlexSymbol{q}{Var}{latin}{71}
\DeclareFlexSymbol{r}{Var}{latin}{72}
\DeclareFlexSymbol{s}{Var}{latin}{73}
\DeclareFlexSymbol{t}{Var}{latin}{74}
\DeclareFlexSymbol{u}{Var}{latin}{75}
\DeclareFlexSymbol{v}{Var}{latin}{76}
\DeclareFlexSymbol{w}{Var}{latin}{77}
\DeclareFlexSymbol{x}{Var}{latin}{78}
\DeclareFlexSymbol{y}{Var}{latin}{79}
\DeclareFlexSymbol{z}{Var}{latin}{7A}
\DeclareFlexSymbol{A}{Var}{Latin}{41}
\DeclareFlexSymbol{B}{Var}{Latin}{42}
\DeclareFlexSymbol{C}{Var}{Latin}{43}
\DeclareFlexSymbol{D}{Var}{Latin}{44}
\DeclareFlexSymbol{E}{Var}{Latin}{45}
\DeclareFlexSymbol{F}{Var}{Latin}{46}
\DeclareFlexSymbol{G}{Var}{Latin}{47}
\DeclareFlexSymbol{H}{Var}{Latin}{48}
\DeclareFlexSymbol{I}{Var}{Latin}{49}
\DeclareFlexSymbol{J}{Var}{Latin}{4A}
\DeclareFlexSymbol{K}{Var}{Latin}{4B}
\DeclareFlexSymbol{L}{Var}{Latin}{4C}
\DeclareFlexSymbol{M}{Var}{Latin}{4D}
\DeclareFlexSymbol{N}{Var}{Latin}{4E}
\DeclareFlexSymbol{O}{Var}{Latin}{4F}
\DeclareFlexSymbol{P}{Var}{Latin}{50}
\DeclareFlexSymbol{Q}{Var}{Latin}{51}
\DeclareFlexSymbol{R}{Var}{Latin}{52}
\DeclareFlexSymbol{S}{Var}{Latin}{53}
\DeclareFlexSymbol{T}{Var}{Latin}{54}
\DeclareFlexSymbol{U}{Var}{Latin}{55}
\DeclareFlexSymbol{V}{Var}{Latin}{56}
\DeclareFlexSymbol{W}{Var}{Latin}{57}
\DeclareFlexSymbol{X}{Var}{Latin}{58}
\DeclareFlexSymbol{Y}{Var}{Latin}{59}
\DeclareFlexSymbol{Z}{Var}{Latin}{5A}
%    \end{macrocode}
% The \cs{ldotPun} glyph is used in constructing the
% \cs{ldots} symbol.    It is just a period with a different math
% symbol class.    \cs{lhookRel} and \cs{rhookRel} are used
% in a similar way for building hooked arrow symbols.   
%    \begin{macrocode}
\DeclareFlexSymbol{\ldotPun}{Pun}{OML}{3A}
\def\ldotp{\ldotPun}
\DeclareFlexSymbol{\lhookRel}{Rel}{OML}{2C}
\DeclareFlexSymbol{\rhookRel}{Rel}{OML}{2D}
%    \end{macrocode}
% Symbols from the 128-character \fn{cmsy} encoding.   
%    \begin{macrocode}
\DeclareFlexSymbol{*}{Bin}{bin}{03} % \ast
\DeclareFlexSymbol{-}{Bin}{bin}{00}
\DeclareFlexSymbol{|}{Ord}{OMS}{6A}
\DeclareFlexSymbol{\aleph}{Ord}{ord}{40}
\DeclareFlexSymbol{\Re}{Ord}{ord}{3C}
\DeclareFlexSymbol{\Im}{Ord}{ord}{3D}
\DeclareFlexSymbol{\infty}{Ord}{ord}{31}
\DeclareFlexSymbol{\prime}{Ord}{ord}{30}
\DeclareFlexSymbol{\emptyset}{Ord}{ord}{3B}
\DeclareFlexSymbol{\nabla}{Ord}{ord}{72}
\DeclareFlexSymbol{\top}{Ord}{ord}{3E}
\DeclareFlexSymbol{\bot}{Ord}{ord}{3F}
\DeclareFlexSymbol{\triangle}{Ord}{ord}{34}
\DeclareFlexSymbol{\forall}{Ord}{ord}{38}
\DeclareFlexSymbol{\exists}{Ord}{ord}{39}
\DeclareFlexSymbol{\neg}{Ord}{ord}{3A}
\DeclareFlexSymbol{\clubsuit}{Ord}{ord}{7C}
\DeclareFlexSymbol{\diamondsuit}{Ord}{ord}{7D}
\DeclareFlexSymbol{\heartsuit}{Ord}{ord}{7E}
\DeclareFlexSymbol{\spadesuit}{Ord}{ord}{7F}
\DeclareFlexSymbol{\smallint}{COs}{OMS}{73}
%    \end{macrocode}
% Binary operators.   
%    \begin{macrocode}
\DeclareFlexSymbol{\bigtriangleup}{Bin}{bin}{34}
\DeclareFlexSymbol{\bigtriangledown}{Bin}{bin}{35}
\DeclareFlexSymbol{\wedge}{Bin}{bin}{5E}
\DeclareFlexSymbol{\vee}{Bin}{bin}{5F}
\DeclareFlexSymbol{\cap}{Bin}{bin}{5C}
\DeclareFlexSymbol{\cup}{Bin}{bin}{5B}
\DeclareFlexSymbol{\ddagger}{Bin}{bin}{7A}
\DeclareFlexSymbol{\dagger}{Bin}{bin}{79}
\DeclareFlexSymbol{\sqcap}{Bin}{bin}{75}
\DeclareFlexSymbol{\sqcup}{Bin}{bin}{74}
\DeclareFlexSymbol{\uplus}{Bin}{bin}{5D}
\DeclareFlexSymbol{\amalg}{Bin}{bin}{71}
\DeclareFlexSymbol{\diamond}{Bin}{bin}{05}
\DeclareFlexSymbol{\bullet}{Bin}{bin}{0F}
\DeclareFlexSymbol{\wr}{Bin}{bin}{6F}
\DeclareFlexSymbol{\div}{Bin}{bin}{04}
\DeclareFlexSymbol{\odot}{Bin}{bin}{0C}
\DeclareFlexSymbol{\oslash}{Bin}{bin}{0B}
\DeclareFlexSymbol{\otimes}{Bin}{bin}{0A}
\DeclareFlexSymbol{\ominus}{Bin}{bin}{09}
\DeclareFlexSymbol{\oplus}{Bin}{bin}{08}
\DeclareFlexSymbol{\mp}{Bin}{bin}{07}
\DeclareFlexSymbol{\pm}{Bin}{bin}{06}
\DeclareFlexSymbol{\circ}{Bin}{bin}{0E}
\DeclareFlexSymbol{\bigcirc}{Bin}{bin}{0D}
\DeclareFlexSymbol{\setminus}{Bin}{bin}{6E}
\DeclareFlexSymbol{\cdot}{Bin}{bin}{01}
\DeclareFlexSymbol{\ast}{Bin}{bin}{03}
\DeclareFlexSymbol{\times}{Bin}{bin}{02}
%    \end{macrocode}
% Relation symbols.   
%    \begin{macrocode}
\DeclareFlexSymbol{\propto}{Rel}{rel}{2F}
\DeclareFlexSymbol{\sqsubseteq}{Rel}{rel}{76}
\DeclareFlexSymbol{\sqsupseteq}{Rel}{rel}{77}
\DeclareFlexSymbol{\parallel}{Rel}{rel}{6B}
\DeclareFlexSymbol{\mid}{Rel}{rel}{6A}
\DeclareFlexSymbol{\dashv}{Rel}{rel}{61}
\DeclareFlexSymbol{\vdash}{Rel}{rel}{60}
\DeclareFlexSymbol{\nearrow}{Rel}{rel}{25}
\DeclareFlexSymbol{\searrow}{Rel}{rel}{26}
\DeclareFlexSymbol{\nwarrow}{Rel}{rel}{2D}
\DeclareFlexSymbol{\swarrow}{Rel}{rel}{2E}
\DeclareFlexSymbol{\Leftrightarrow}{Rel}{rel}{2C}
\DeclareFlexSymbol{\Leftarrow}{Rel}{rel}{28}
\DeclareFlexSymbol{\Rightarrow}{Rel}{rel}{29}
\DeclareFlexSymbol{\leq}{Rel}{rel}{14}
\DeclareFlexSymbol{\geq}{Rel}{rel}{15}
\DeclareFlexSymbol{\succ}{Rel}{rel}{1F}
\DeclareFlexSymbol{\prec}{Rel}{rel}{1E}
\DeclareFlexSymbol{\approx}{Rel}{rel}{19}
\DeclareFlexSymbol{\succeq}{Rel}{rel}{17}
\DeclareFlexSymbol{\preceq}{Rel}{rel}{16}
\DeclareFlexSymbol{\supset}{Rel}{rel}{1B}
\DeclareFlexSymbol{\subset}{Rel}{rel}{1A}
\DeclareFlexSymbol{\supseteq}{Rel}{rel}{13}
\DeclareFlexSymbol{\subseteq}{Rel}{rel}{12}
\DeclareFlexSymbol{\in}{Rel}{rel}{32}
\DeclareFlexSymbol{\ni}{Rel}{rel}{33}
\DeclareFlexSymbol{\gg}{Rel}{rel}{1D}
\DeclareFlexSymbol{\ll}{Rel}{rel}{1C}
\DeclareFlexSymbol{\leftrightarrow}{Rel}{rel}{24}
\DeclareFlexSymbol{\leftarrow}{Rel}{rel}{20}
\DeclareFlexSymbol{\rightarrow}{Rel}{rel}{21}
\DeclareFlexSymbol{\sim}{Rel}{rel}{18}
\DeclareFlexSymbol{\simeq}{Rel}{rel}{27}
\DeclareFlexSymbol{\perp}{Rel}{rel}{3F}
\DeclareFlexSymbol{\equiv}{Rel}{rel}{11}
\DeclareFlexSymbol{\asymp}{Rel}{rel}{10}
%    \end{macrocode}
% The \cs{notRel} glyph is a special zero-width glyph intended only
% for use in constructing negated symbols.    \cs{mapstoRel} and
% \cs{cdotPun} have similar but more restricted applications.   
%    \begin{macrocode}
\DeclareFlexSymbol{\notRel}{Rel}{rel}{36}
\DeclareFlexSymbol{\mapstoOrd}{Ord}{OMS}{37}
\DeclareFlexSymbol{\cdotOrd}{Ord}{OMS}{01}
\def\cdotp{\mathpunct{\cdotOrd}}
%    \end{macrocode}
% Symbols from the 128-character \fn{cmex} encoding.   
% \verb"COs" stands for `cumulative operator
% (sum-like)'.   
% \verb"COi" stands for `cumulative operator
% (integral-like)'.    These typically differ only in the
% default placement of limits.    \verb"cop" stands for
% `cumulative operator math group'.   
%    \begin{macrocode}
\DeclareFlexSymbol{\coprod}{COs}{cop}{60}
\DeclareFlexSymbol{\bigvee}{COs}{cop}{57}
\DeclareFlexSymbol{\bigwedge}{COs}{cop}{56}
\DeclareFlexSymbol{\biguplus}{COs}{cop}{55}
\DeclareFlexSymbol{\bigcap}{COs}{cop}{54}
\DeclareFlexSymbol{\bigcup}{COs}{cop}{53}
\DeclareFlexSymbol{\int}{COi}{cop}{52}
\DeclareFlexSymbol{\prod}{COs}{cop}{51}
\DeclareFlexSymbol{\sum}{COs}{cop}{50}
\DeclareFlexSymbol{\bigotimes}{COs}{cop}{4E}
\DeclareFlexSymbol{\bigoplus}{COs}{cop}{4C}
\DeclareFlexSymbol{\bigodot}{COs}{cop}{4A}
\DeclareFlexSymbol{\oint}{COi}{cop}{48}
\DeclareFlexSymbol{\bigsqcup}{COs}{cop}{46}
%    \end{macrocode}
% Delimiter symbols.   
% \verb"DeL" stands for `delimiter (left)'.   
% \verb"DeR" stands for `delimiter (right)'.   
% \verb"DeB" stands for `delimiter (bidirectional)'.   
% The principal encoding point for an extensible delimiter is the
% first link in the list of linked sizes as specified in the font metric
% information.   
% For a math encoding such as OT1/OML/OMS/OMX where not all sizes of a
% given delimiter reside in a given font, the extra encoding point for the
% smallest delimiter must be supplied by defining
% \begin{verbatim}
% \sd@GXX
% \end{verbatim}
% where G is the mathgroup and XX is the hexadecimal glyph position.   
%    \begin{macrocode}
\DeclareFlexSymbol{\rangle}{DeR}{del}{0B}
\DeclareFlexSymbol{\langle}{DeL}{del}{0A}
\DeclareFlexSymbol{\rbrace}{DeR}{del}{09}
\DeclareFlexSymbol{\lbrace}{DeL}{del}{08}
\DeclareFlexSymbol{\rceil}{DeR}{del}{07}
\DeclareFlexSymbol{\lceil}{DeL}{del}{06}
\DeclareFlexSymbol{\rfloor}{DeR}{del}{05}
\DeclareFlexSymbol{\lfloor}{DeL}{del}{04}
\DeclareFlexSymbol{(}{DeL}{del}{00}
\DeclareFlexSymbol{)}{DeR}{del}{01}
\DeclareFlexSymbol{[}{DeL}{del}{02}
\DeclareFlexSymbol{]}{DeR}{del}{03}
\DeclareFlexSymbol{\lVert}{DeL}{del}{0D}
\DeclareFlexSymbol{\rVert}{DeR}{del}{0D}
\DeclareFlexSymbol{\lvert}{DeL}{del}{0C}
\DeclareFlexSymbol{\rvert}{DeR}{del}{0C}
\DeclareFlexSymbol{\Vert}{DeB}{del}{0D}
\DeclareFlexSymbol{\vert}{DeB}{del}{0C}
%    \end{macrocode}
% Maybe make the vert bars mathord instead of delimiter, to discourage
% poor usage.   
%    \begin{macrocode}
\DeclareFlexSymbol{|}{DeB}{del}{0C}
\DeclareFlexSymbol{/}{DeB}{del}{0E}
%    \end{macrocode}
% 
% 
% These wacky delimiters need to be supported I guess for
% compabitility reasons.   
% The DeA delimiter type is a special case used only for these
% arrows.   
%    \begin{macrocode}
\DeclareFlexSymbol{\lmoustache}{DeL}{del}{40}
\DeclareFlexSymbol{\rmoustache}{DeR}{del}{41}
\DeclareFlexSymbol{\lgroup}{DeL}{del}{3A}
\DeclareFlexSymbol{\rgroup}{DeR}{del}{3B}
\DeclareFlexSymbol{\bracevert}{DeB}{del}{3E}
\DeclareFlexSymbol{\arrowvert}{DeB}{del}{3C}
\DeclareFlexSymbol{\Arrowvert}{DeB}{del}{3D}
\DeclareFlexSymbol{\uparrow}{DeA}{del}{78}
\DeclareFlexSymbol{\downarrow}{DeA}{del}{79}
\DeclareFlexSymbol{\updownarrow}{DeA}{del}{3F}
\DeclareFlexSymbol{\Uparrow}{DeA}{del}{7E}
\DeclareFlexSymbol{\Downarrow}{DeA}{del}{7F}
\DeclareFlexSymbol{\Updownarrow}{DeA}{del}{77}
\DeclareFlexSymbol{\backslash}{DeB}{del}{0F}
%    \end{macrocode}
% 
% 
% 
% 
% \section{Some compound symbols}
% The following symbols are not robust in standard \LaTeX\ 
% because they use \verb"#" or \cs{mathpalette} (which is not
% robust and contains a \verb"#" in its expansion): \cs{angle},
% \cs{cong}, \cs{notin}, \cs{rightleftharpoons}.   
% 
% In this definition of \cs{hbar}, the symbol is cobbled together
% from a math italic h and the cmr overbar accent glyph.   
%    \begin{macrocode}
\DeclareFlexSymbol{\hbarOrd}{Ord}{OT1}{16}
\DeclareFlexCompoundSymbol{\hbar}{Ord}{\hbarOrd\mkern-9mu h}
%    \end{macrocode}
% For \cs{surd}, the interior symbol gets math class 1
% (cumulative operator) to make the glyph vertically centered on the
% math axis, but the desired horizontal spacing is the spacing for a
% mathord.    (Couldn't it just be class mathopen, though?   )
%    \begin{macrocode}
\DeclareFlexSymbol{\surdOrd}{Ord}{OMS}{70}
\DeclareFlexCompoundSymbol{\surd}{Ord}{\mathop{\surdOrd}}
%    \end{macrocode}
% As shown in this definition of \cs{angle}, rule dimens are not
% allowed to use math-units, unfortunately.   
%    \begin{macrocode}
\DeclareFlexCompoundSymbol{\angle}{Ord}{%
  \vbox{\ialign{%
      $\m@th\scriptstyle##$\crcr
      \notRel\mathrel{\mkern14mu}\crcr
      \noalign{\nointerlineskip}%
      \mkern2.5mu\leaders\hrule \@height.34pt\hfill\mkern2.5mu\crcr
  }}%
}
%    \end{macrocode}
% The \cs{not} function, which is defined in the \pkg{flexisym}
% package, requires a suitably defined \cs{notRel} symbol.   
%    \begin{macrocode}
\DeclareFlexCompoundSymbol{\neq}{Rel}{\not{=}}
%    \end{macrocode}
% .   
%    \begin{macrocode}
\DeclareFlexCompoundSymbol{\mapsto}{Rel}{\mapstoOrd\rightarrow}
%    \end{macrocode}
% The \cs{@vereq} function ends by centering the whole
% construction on the math axis, unlike \cs{buildrel} where the base
% symbol remains at its normal altitude.    Furthermore,
% \cs{@vereq} leaves the math style of the top symbol as given
% instead of downsizing to scriptstyle.   
%    \begin{macrocode}
\DeclareFlexCompoundSymbol{\cong}{Rel}{\mathpalette\@vereq\sim}
%    \end{macrocode}
% The \cs{m@th} in the \fn{fontmath.ltx} definition of
% \cs{notin} is superfluous unless \cs{c@ncel} doesn't include
% it (which was perhaps true in an older version of
% \fn{plain.tex}?).   
%    \begin{macrocode}
\providecommand*\joinord{}
%<cmbase|mathptmx>\renewcommand*\joinord{\mkern-3mu }
%<mathpazo>\renewcommand*\joinord{\mkern-3.45mu }
\DeclareFlexCompoundSymbol{\notin}{Rel}{\mathpalette\c@ncel\in}
\DeclareFlexCompoundSymbol{\rightleftharpoons}{Rel}{\mathpalette\rlh@{}}
\DeclareFlexCompoundSymbol{\doteq}{Rel}{\buildrel\textstyle.\over=}
\DeclareFlexCompoundSymbol{\hookrightarrow}{Rel}{\lhookRel\joinord\rightarrow}
\DeclareFlexCompoundSymbol{\hookleftarrow}{Rel}{\leftarrow\joinord\rhookRel}
\DeclareFlexCompoundSymbol{\bowtie}{Rel}{\triangleright\joinord\triangleleft}
\DeclareFlexCompoundSymbol{\models}{Rel}{\vert\joinord=}
\DeclareFlexCompoundSymbol{\Longrightarrow}{Rel}{\Relbar\joinord\Rightarrow}
\DeclareFlexCompoundSymbol{\longrightarrow}{Rel}{\relbar\joinord\rightarrow}
\DeclareFlexCompoundSymbol{\Longleftarrow}{Rel}{\Leftarrow\joinord\Relbar}
\DeclareFlexCompoundSymbol{\longleftarrow}{Rel}{\leftarrow\joinord\relbar}
\DeclareFlexCompoundSymbol{\longmapsto}{Rel}{\mapstochar\longrightarrow}
\DeclareFlexCompoundSymbol{\longleftrightarrow}{Rel}{\leftarrow\joinord\rightarrow}
\DeclareFlexCompoundSymbol{\Longleftrightarrow}{Rel}{\Leftarrow\joinord\Rightarrow}
%    \end{macrocode}
% Here is what you get from the old definition of \cs{iff}.   
% \begin{verbatim}
% \glue 2.77771 plus 2.77771
% \glue(\thickmuskip) 2.77771 plus 2.77771
% \OMS/cmsy/m/n/10 (
% \hbox(0.0+0.0)x-1.66663
% .\kern -1.66663
% \OMS/cmsy/m/n/10 )
% \penalty 500
% \glue 2.77771 plus 2.77771
% \glue(\thickmuskip) 2.77771 plus 2.77771
% \end{verbatim}
% Looks like it could be simplified slightly.    But it's not so
% easy as it looks to do it without screwing up the line breaking
% possibilities.   
%    \begin{macrocode}
\renewcommand*\iff{%
  \mskip\thickmuskip\Longleftrightarrow\mskip\thickmuskip
}
%    \end{macrocode}
% Some dotly symbols.   
%    \begin{macrocode}
\DeclareFlexCompoundSymbol{\cdots}{Inn}{\cdotp\cdotp\cdotp}%
\DeclareFlexCompoundSymbol{\vdots}{Ord}{%
  \vbox{\baselineskip4\p@ \lineskiplimit\z@
    \kern6\p@\hbox{.}\hbox{.}\hbox{.}}}
\DeclareFlexCompoundSymbol{\ddots}{Inn}{%
  \mkern1mu\raise7\p@
  \vbox{\kern7\p@\hbox{.}}\mkern2mu%
  \raise4\p@\hbox{.}\mkern2mu\raise\p@\hbox{.}\mkern1mu%
}
%    \end{macrocode}
% .   
%    \begin{macrocode}
\def\relbar{\begingroup \def\smash@{tb}% in case amsmath is loaded
    \mathpalette\mathsm@sh{\mathchar"200 }\endgroup}
%    \end{macrocode}
% For \cs{Relbar} we take an equal sign of class $0$ (Ord) from the
% operator family. For \fn{cmr} and \pkg{mathptmx} we know this is
% family $0$.
%    \begin{macrocode}
%<cmbase|mathptmx>\def\Relbar{\mathchar"3D }
%    \end{macrocode}
% For the \pkg{mathpazo} setup we need to use the equal sign from
% \fn{cmr} and so must insert class $0$ and use the symbol from the
% upright symbols.
%    \begin{macrocode}
%<mathpazo>\edef\Relbar{\mathchar\string"\hexnumber@\symupright3D }
%    \end{macrocode}
% Done.
%    \begin{macrocode}
%</cmbase|mathpazo|mathptmx>
%    \end{macrocode}
% Various synonyms such as \cs{le} for \cs{leq} and
% \cs{to} for \cs{rightarrow} are defined in
% \pkg{flexisym} with \cs{def} instead of \cs{let}, for
% slower execution speed but smaller chance of synchronization
% problems.   
%
%
%
%    \begin{macrocode}
%<*msabm>
\ProvidesSymbols{msabm}[2001/09/08 v0.91]
%    \end{macrocode}
%    \begin{macrocode}
\RequirePackage{amsfonts}\relax
%    \end{macrocode}
%    \begin{macrocode}
\@xp\xdef\csname mg@MSA\endcsname{\hexnumber@\symAMSa}%
\@xp\xdef\csname mg@MSB\endcsname{\hexnumber@\symAMSb}%
%    \end{macrocode}
%    \begin{macrocode}
\DeclareFlexSymbol{\boxdot}       {Bin}{MSA}{00}
\DeclareFlexSymbol{\boxplus}      {Bin}{MSA}{01}
\DeclareFlexSymbol{\boxtimes}     {Bin}{MSA}{02}
\DeclareFlexSymbol{\square}       {Ord}{MSA}{03}
\DeclareFlexSymbol{\blacksquare}  {Ord}{MSA}{04}
\DeclareFlexSymbol{\centerdot}    {Bin}{MSA}{05}
\DeclareFlexSymbol{\lozenge}      {Ord}{MSA}{06}
\DeclareFlexSymbol{\blacklozenge} {Ord}{MSA}{07}
\DeclareFlexSymbol{\circlearrowright}   {Rel}{MSA}{08}
\DeclareFlexSymbol{\circlearrowleft}    {Rel}{MSA}{09}
%    \end{macrocode}
% In amsfonts.sty:
%    \begin{macrocode}
%%\DeclareFlexSymbol{\rightleftharpoons}{Rel}{MSA}{0A}
\DeclareFlexSymbol{\leftrightharpoons}  {Rel}{MSA}{0B}
\DeclareFlexSymbol{\boxminus}     {Bin}{MSA}{0C}
\DeclareFlexSymbol{\Vdash}        {Rel}{MSA}{0D}
\DeclareFlexSymbol{\Vvdash}       {Rel}{MSA}{0E}
\DeclareFlexSymbol{\vDash}        {Rel}{MSA}{0F}
\DeclareFlexSymbol{\twoheadrightarrow}  {Rel}{MSA}{10}
\DeclareFlexSymbol{\twoheadleftarrow}   {Rel}{MSA}{11}
\DeclareFlexSymbol{\leftleftarrows}     {Rel}{MSA}{12}
\DeclareFlexSymbol{\rightrightarrows}   {Rel}{MSA}{13}
\DeclareFlexSymbol{\upuparrows}         {Rel}{MSA}{14}
\DeclareFlexSymbol{\downdownarrows} {Rel}{MSA}{15}
\DeclareFlexSymbol{\upharpoonright} {Rel}{MSA}{16}
 \let\restriction\upharpoonright
\DeclareFlexSymbol{\downharpoonright}   {Rel}{MSA}{17}
\DeclareFlexSymbol{\upharpoonleft}  {Rel}{MSA}{18}
\DeclareFlexSymbol{\downharpoonleft}{Rel}{MSA}{19}
\DeclareFlexSymbol{\rightarrowtail} {Rel}{MSA}{1A}
\DeclareFlexSymbol{\leftarrowtail}  {Rel}{MSA}{1B}
\DeclareFlexSymbol{\leftrightarrows}{Rel}{MSA}{1C}
\DeclareFlexSymbol{\rightleftarrows}{Rel}{MSA}{1D}
\DeclareFlexSymbol{\Lsh}            {Rel}{MSA}{1E}
\DeclareFlexSymbol{\Rsh}            {Rel}{MSA}{1F}
\DeclareFlexSymbol{\rightsquigarrow}  {Rel}{MSA}{20}
\DeclareFlexSymbol{\leftrightsquigarrow}{Rel}{MSA}{21}
\DeclareFlexSymbol{\looparrowleft}  {Rel}{MSA}{22}
\DeclareFlexSymbol{\looparrowright} {Rel}{MSA}{23}
\DeclareFlexSymbol{\circeq}       {Rel}{MSA}{24}
\DeclareFlexSymbol{\succsim}      {Rel}{MSA}{25}
\DeclareFlexSymbol{\gtrsim}       {Rel}{MSA}{26}
\DeclareFlexSymbol{\gtrapprox}    {Rel}{MSA}{27}
\DeclareFlexSymbol{\multimap}     {Rel}{MSA}{28}
\DeclareFlexSymbol{\therefore}    {Rel}{MSA}{29}
\DeclareFlexSymbol{\because}      {Rel}{MSA}{2A}
\DeclareFlexSymbol{\doteqdot}     {Rel}{MSA}{2B}
 \let\Doteq\doteqdot
\DeclareFlexSymbol{\triangleq}    {Rel}{MSA}{2C}
\DeclareFlexSymbol{\precsim}      {Rel}{MSA}{2D}
\DeclareFlexSymbol{\lesssim}      {Rel}{MSA}{2E}
\DeclareFlexSymbol{\lessapprox}   {Rel}{MSA}{2F}
\DeclareFlexSymbol{\eqslantless}  {Rel}{MSA}{30}
\DeclareFlexSymbol{\eqslantgtr}   {Rel}{MSA}{31}
\DeclareFlexSymbol{\curlyeqprec}  {Rel}{MSA}{32}
\DeclareFlexSymbol{\curlyeqsucc}  {Rel}{MSA}{33}
\DeclareFlexSymbol{\preccurlyeq}  {Rel}{MSA}{34}
\DeclareFlexSymbol{\leqq}         {Rel}{MSA}{35}
\DeclareFlexSymbol{\leqslant}     {Rel}{MSA}{36}
\DeclareFlexSymbol{\lessgtr}      {Rel}{MSA}{37}
\DeclareFlexSymbol{\backprime}    {Ord}{MSA}{38}
\DeclareFlexSymbol{\risingdotseq} {Rel}{MSA}{3A}
\DeclareFlexSymbol{\fallingdotseq}{Rel}{MSA}{3B}
\DeclareFlexSymbol{\succcurlyeq}  {Rel}{MSA}{3C}
\DeclareFlexSymbol{\geqq}         {Rel}{MSA}{3D}
\DeclareFlexSymbol{\geqslant}     {Rel}{MSA}{3E}
\DeclareFlexSymbol{\gtrless}      {Rel}{MSA}{3F}
%    \end{macrocode}
% in amsfonts.sty
%    \begin{macrocode}
%% \DeclareFlexSymbol{\sqsubset}    {Rel}{MSA}{40}
%% \DeclareFlexSymbol{\sqsupset}    {Rel}{MSA}{41}
\DeclareFlexSymbol{\vartriangleright}{Rel}{MSA}{42}
\DeclareFlexSymbol{\vartriangleleft} {Rel}{MSA}{43}
\DeclareFlexSymbol{\trianglerighteq} {Rel}{MSA}{44}
\DeclareFlexSymbol{\trianglelefteq}  {Rel}{MSA}{45}
\DeclareFlexSymbol{\bigstar}    {Ord}{MSA}{46}
\DeclareFlexSymbol{\between}    {Rel}{MSA}{47}
\DeclareFlexSymbol{\blacktriangledown}  {Ord}{MSA}{48}
\DeclareFlexSymbol{\blacktriangleright} {Rel}{MSA}{49}
\DeclareFlexSymbol{\blacktriangleleft}  {Rel}{MSA}{4A}
\DeclareFlexSymbol{\vartriangle}        {Rel}{MSA}{4D}
\DeclareFlexSymbol{\blacktriangle}      {Ord}{MSA}{4E}
\DeclareFlexSymbol{\triangledown}       {Ord}{MSA}{4F}
\DeclareFlexSymbol{\eqcirc}       {Rel}{MSA}{50}
\DeclareFlexSymbol{\lesseqgtr}    {Rel}{MSA}{51}
\DeclareFlexSymbol{\gtreqless}    {Rel}{MSA}{52}
\DeclareFlexSymbol{\lesseqqgtr}   {Rel}{MSA}{53}
\DeclareFlexSymbol{\gtreqqless}   {Rel}{MSA}{54}
\DeclareFlexSymbol{\Rrightarrow}  {Rel}{MSA}{56}
\DeclareFlexSymbol{\Lleftarrow}   {Rel}{MSA}{57}
\DeclareFlexSymbol{\veebar}       {Bin}{MSA}{59}
\DeclareFlexSymbol{\barwedge}     {Bin}{MSA}{5A}
\DeclareFlexSymbol{\doublebarwedge} {Bin}{MSA}{5B}
%    \end{macrocode}
% In amsfonts.sty
%    \begin{macrocode}
%%\DeclareFlexSymbol{\angle}        {Ord}{MSA}{5C}
\DeclareFlexSymbol{\measuredangle}  {Ord}{MSA}{5D}
\DeclareFlexSymbol{\sphericalangle} {Ord}{MSA}{5E}
\DeclareFlexSymbol{\varpropto}    {Rel}{MSA}{5F}
\DeclareFlexSymbol{\smallsmile}   {Rel}{MSA}{60}
\DeclareFlexSymbol{\smallfrown}   {Rel}{MSA}{61}
\DeclareFlexSymbol{\Subset}       {Rel}{MSA}{62}
\DeclareFlexSymbol{\Supset}       {Rel}{MSA}{63}
\DeclareFlexSymbol{\Cup}          {Bin}{MSA}{64}
 \let\doublecup\Cup
\DeclareFlexSymbol{\Cap}          {Bin}{MSA}{65}
 \let\doublecap\Cap
\DeclareFlexSymbol{\curlywedge}   {Bin}{MSA}{66}
\DeclareFlexSymbol{\curlyvee}     {Bin}{MSA}{67}
\DeclareFlexSymbol{\leftthreetimes} {Bin}{MSA}{68}
\DeclareFlexSymbol{\rightthreetimes}{Bin}{MSA}{69}
\DeclareFlexSymbol{\subseteqq}    {Rel}{MSA}{6A}
\DeclareFlexSymbol{\supseteqq}    {Rel}{MSA}{6B}
\DeclareFlexSymbol{\bumpeq}       {Rel}{MSA}{6C}
\DeclareFlexSymbol{\Bumpeq}       {Rel}{MSA}{6D}
\DeclareFlexSymbol{\lll}          {Rel}{MSA}{6E}
 \let\llless\lll
\DeclareFlexSymbol{\ggg}          {Rel}{MSA}{6F}
 \let\gggtr\ggg
\DeclareFlexSymbol{\circledS}     {Ord}{MSA}{73}
\DeclareFlexSymbol{\pitchfork}    {Rel}{MSA}{74}
\DeclareFlexSymbol{\dotplus}      {Bin}{MSA}{75}
\DeclareFlexSymbol{\backsim}      {Rel}{MSA}{76}
\DeclareFlexSymbol{\backsimeq}    {Rel}{MSA}{77}
\DeclareFlexSymbol{\complement}   {Ord}{MSA}{7B}
\DeclareFlexSymbol{\intercal}     {Bin}{MSA}{7C}
\DeclareFlexSymbol{\circledcirc}  {Bin}{MSA}{7D}
\DeclareFlexSymbol{\circledast}   {Bin}{MSA}{7E}
\DeclareFlexSymbol{\circleddash}  {Bin}{MSA}{7F}
%    \end{macrocode}
%   Begin AMSb declarations
%    \begin{macrocode}
\DeclareFlexSymbol{\lvertneqq}    {Rel}{MSB}{00}
\DeclareFlexSymbol{\gvertneqq}    {Rel}{MSB}{01}
\DeclareFlexSymbol{\nleq}         {Rel}{MSB}{02}
\DeclareFlexSymbol{\ngeq}         {Rel}{MSB}{03}
\DeclareFlexSymbol{\nless}        {Rel}{MSB}{04}
\DeclareFlexSymbol{\ngtr}         {Rel}{MSB}{05}
\DeclareFlexSymbol{\nprec}        {Rel}{MSB}{06}
\DeclareFlexSymbol{\nsucc}        {Rel}{MSB}{07}
\DeclareFlexSymbol{\lneqq}        {Rel}{MSB}{08}
\DeclareFlexSymbol{\gneqq}        {Rel}{MSB}{09}
\DeclareFlexSymbol{\nleqslant}    {Rel}{MSB}{0A}
\DeclareFlexSymbol{\ngeqslant}    {Rel}{MSB}{0B}
\DeclareFlexSymbol{\lneq}         {Rel}{MSB}{0C}
\DeclareFlexSymbol{\gneq}         {Rel}{MSB}{0D}
\DeclareFlexSymbol{\npreceq}      {Rel}{MSB}{0E}
\DeclareFlexSymbol{\nsucceq}      {Rel}{MSB}{0F}
\DeclareFlexSymbol{\precnsim}     {Rel}{MSB}{10}
\DeclareFlexSymbol{\succnsim}     {Rel}{MSB}{11}
\DeclareFlexSymbol{\lnsim}        {Rel}{MSB}{12}
\DeclareFlexSymbol{\gnsim}        {Rel}{MSB}{13}
\DeclareFlexSymbol{\nleqq}        {Rel}{MSB}{14}
\DeclareFlexSymbol{\ngeqq}        {Rel}{MSB}{15}
\DeclareFlexSymbol{\precneqq}     {Rel}{MSB}{16}
\DeclareFlexSymbol{\succneqq}     {Rel}{MSB}{17}
\DeclareFlexSymbol{\precnapprox}  {Rel}{MSB}{18}
\DeclareFlexSymbol{\succnapprox}  {Rel}{MSB}{19}
\DeclareFlexSymbol{\lnapprox}     {Rel}{MSB}{1A}
\DeclareFlexSymbol{\gnapprox}     {Rel}{MSB}{1B}
\DeclareFlexSymbol{\nsim}         {Rel}{MSB}{1C}
\DeclareFlexSymbol{\ncong}        {Rel}{MSB}{1D}
\DeclareFlexSymbol{\diagup}       {Ord}{MSB}{1E}
\DeclareFlexSymbol{\diagdown}     {Ord}{MSB}{1F}
\DeclareFlexSymbol{\varsubsetneq}   {Rel}{MSB}{20}
\DeclareFlexSymbol{\varsupsetneq}   {Rel}{MSB}{21}
\DeclareFlexSymbol{\nsubseteqq}     {Rel}{MSB}{22}
\DeclareFlexSymbol{\nsupseteqq}     {Rel}{MSB}{23}
\DeclareFlexSymbol{\subsetneqq}     {Rel}{MSB}{24}
\DeclareFlexSymbol{\supsetneqq}     {Rel}{MSB}{25}
\DeclareFlexSymbol{\varsubsetneqq}  {Rel}{MSB}{26}
\DeclareFlexSymbol{\varsupsetneqq}  {Rel}{MSB}{27}
\DeclareFlexSymbol{\subsetneq}      {Rel}{MSB}{28}
\DeclareFlexSymbol{\supsetneq}      {Rel}{MSB}{29}
\DeclareFlexSymbol{\nsubseteq}      {Rel}{MSB}{2A}
\DeclareFlexSymbol{\nsupseteq}      {Rel}{MSB}{2B}
\DeclareFlexSymbol{\nparallel}      {Rel}{MSB}{2C}
\DeclareFlexSymbol{\nmid}           {Rel}{MSB}{2D}
\DeclareFlexSymbol{\nshortmid}      {Rel}{MSB}{2E}
\DeclareFlexSymbol{\nshortparallel} {Rel}{MSB}{2F}
\DeclareFlexSymbol{\nvdash}         {Rel}{MSB}{30}
\DeclareFlexSymbol{\nVdash}         {Rel}{MSB}{31}
\DeclareFlexSymbol{\nvDash}         {Rel}{MSB}{32}
\DeclareFlexSymbol{\nVDash}         {Rel}{MSB}{33}
\DeclareFlexSymbol{\ntrianglerighteq}{Rel}{MSB}{34}
\DeclareFlexSymbol{\ntrianglelefteq}{Rel}{MSB}{35}
\DeclareFlexSymbol{\ntriangleleft}  {Rel}{MSB}{36}
\DeclareFlexSymbol{\ntriangleright} {Rel}{MSB}{37}
\DeclareFlexSymbol{\nleftarrow}     {Rel}{MSB}{38}
\DeclareFlexSymbol{\nrightarrow}    {Rel}{MSB}{39}
\DeclareFlexSymbol{\nLeftarrow}     {Rel}{MSB}{3A}
\DeclareFlexSymbol{\nRightarrow}    {Rel}{MSB}{3B}
\DeclareFlexSymbol{\nLeftrightarrow}{Rel}{MSB}{3C}
\DeclareFlexSymbol{\nleftrightarrow}{Rel}{MSB}{3D}
\DeclareFlexSymbol{\divideontimes}  {Bin}{MSB}{3E}
\DeclareFlexSymbol{\varnothing}     {Ord}{MSB}{3F}
\DeclareFlexSymbol{\nexists}        {Ord}{MSB}{40}
\DeclareFlexSymbol{\Finv}           {Ord}{MSB}{60}
\DeclareFlexSymbol{\Game}           {Ord}{MSB}{61}
%    \end{macrocode}
% In amsfonts.sty:
%    \begin{macrocode}
%%\DeclareFlexSymbol{\mho}          {Ord}{MSB}{66}
\DeclareFlexSymbol{\eth}            {Ord}{MSB}{67}
\DeclareFlexSymbol{\eqsim}          {Rel}{MSB}{68}
\DeclareFlexSymbol{\beth}           {Ord}{MSB}{69}
\DeclareFlexSymbol{\gimel}          {Ord}{MSB}{6A}
\DeclareFlexSymbol{\daleth}         {Ord}{MSB}{6B}
\DeclareFlexSymbol{\lessdot}        {Bin}{MSB}{6C}
\DeclareFlexSymbol{\gtrdot}         {Bin}{MSB}{6D}
\DeclareFlexSymbol{\ltimes}         {Bin}{MSB}{6E}
\DeclareFlexSymbol{\rtimes}         {Bin}{MSB}{6F}
\DeclareFlexSymbol{\shortmid}       {Rel}{MSB}{70}
\DeclareFlexSymbol{\shortparallel}  {Rel}{MSB}{71}
\DeclareFlexSymbol{\smallsetminus}  {Bin}{MSB}{72}
\DeclareFlexSymbol{\thicksim}       {Rel}{MSB}{73}
\DeclareFlexSymbol{\thickapprox}    {Rel}{MSB}{74}
\DeclareFlexSymbol{\approxeq}       {Rel}{MSB}{75}
\DeclareFlexSymbol{\succapprox}     {Rel}{MSB}{76}
\DeclareFlexSymbol{\precapprox}     {Rel}{MSB}{77}
\DeclareFlexSymbol{\curvearrowleft} {Rel}{MSB}{78}
\DeclareFlexSymbol{\curvearrowright}{Rel}{MSB}{79}
\DeclareFlexSymbol{\digamma}        {Ord}{MSB}{7A}
\DeclareFlexSymbol{\varkappa}       {Ord}{MSB}{7B}
\DeclareFlexSymbol{\Bbbk}           {Ord}{MSB}{7C}
\DeclareFlexSymbol{\hslash}         {Ord}{MSB}{7D}
%    \end{macrocode}
% In amsfonts.sty:
%    \begin{macrocode}
%%\DeclareFlexSymbol{\hbar}         {Ord}{MSB}{7E}
\DeclareFlexSymbol{\backepsilon}    {Rel}{MSB}{7F}
%</msabm>
%    \end{macrocode}
%
% \PrintIndex
%
% \Finale
