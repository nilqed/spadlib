\documentclass{article}

\usepackage[T1]{fontenc}
\usepackage[utf8]{inputenc}
\usepackage[english]{babel}
\usepackage{amssymb}
\usepackage{xcolor}

\newcommand{\grobner}{Gr\"obner}
\newcommand{\faugere}{Faug\`ere}
\newcommand{\code}[1]{\texttt{#1}}

\begin{document}
\title{F5: Computing Gröbner Bases Using F5}

\author{Alexander Demin, Hamid Rahkooy\\
  MPI Informatics, Saarland Informatics Campus, Germany\\
  \texttt{\{ademin,hrahkooy\}@mpi-inf.mpg.de}
  \and
  Thomas Sturm\\
  CNRS, Inria, and the University of Lorraine, France\\
  \texttt{thomas.sturm@cnrs.fr}
}

\date{July 2022}

\maketitle

\section{Introduction}

F5 is a package for the computation of \grobner{} Bases using \faugere{}'s F5 algorithm \cite{f5}.
Admissible coefficient field is rational functions. It is compatible with existing REDUCE term
orderings, as used with the GROEBNER package and the CGB package. Some package usage examples are
presented in Section~\ref{SE:usage} and in the package test file
\code{f5.tst}.

The efficiency of this algorithm comes from two criteria, namely, the \emph{$F_5$ Criterion} and the
\emph{Rewritten Criterion}, for which it assigns to each polynomial a unique signature. The general
idea of signatures is to keep track of the history of the computation with a minimal overhead and to
exploit this information to detect redundant polynomial reductions. It has been shown that in the
special case of "regular" input F5 computes a \grobner{} basis without reduction to zero \cite{f5}.
For a description of the F5 algorithm as well as a general survey of the area of signature
based-methods see \cite{survey}.

\section{Package Usage}\label{SE:usage}

The following command loads the package into REDUCE:
%
\begin{center}
    \code{load\_package f5;}
\end{center}

The package provides the command \code{f5} and the command \code{torder} to manipulate the current
term order. It comes wth some switches, which are described in Section~\ref{SE:switches}.

The principal function in the interface is
%
\begin{center}
    \code{f5(\{expr\_1, \ldots, expr\_m\}, \{var\_n, \ldots, var\_n\}, ord)}
\end{center}
%
It computes a \grobner{} basis of the polynomials \code {expr\_1}, \dots,~\code{expr\_m} in the variables \code{var\_1}, \dots,~\code{var\_n} using a
term order constructed from the given variables and the sort mode \code{ord}. As usual, all REDUCE
kernels can play the role of variables of the polynomial ring under consideration. Among the sort
modes that are available through the
\code{torder} command, F5 currently supports the basic orders \code{lex},
\code{gradlex}, and \code{revgradlex}, as well as block orders,
weighted orders, and graded orders, but not matrix orders. Recall that the list of variables
specifies their relative order from largest to smallest, like $\code{var\_1} \succ \dots \succ
\code{var\_n}$.

All variables occurring in \code{expr\_1}, \dots,~\code{expr\_m} but not in \code{var\_1},
\dots,~\code{var\_n} are considered to belong to the coefficients. Depending on the switch
\code{f5para\-met\-ric} (compare Section~\ref{SE:switches}) those coefficients are treated either as
formal rational functions or as parametric expressions. In the latter case, certain assumptions on
the non-vanishing of such expressions are made and recorded in the course of the \grobner{} basis
computation. The assumptions can be dumped using the command
%
\begin{center}
\code{f5dumpAssumptions()}.
\end{center}
An example can be found in Figure~\ref{fig:para}.

The arguments \code{var\_1}, \dots,~\code{var\_n} and \code{ord} are optional. The default term
order is $\code{lex}$ with respect to all kernels occurring in numerators of \code{expr\_1},
\dots,~\code{expr\_m}, sorted by the REDUCE kernel order. Note that in this default case, variables
are assigned to the polynomial ring without being explicitly listed in \code{var\_1},
\dots,~\code{var\_n}.

Variables and sort mode can be globally specified using the command
%
\begin{center}
    \code{torder(vars, ord)}.
\end{center}
%
Using the optional arguments of \code{f5} shadows the current global
setting. Note that the observation that graded term orders like
\code{gradlex} and \code{revgradlex} tend to be fast for \grobner{}
basis computations \cite[\S5]{tolstaya} holds also for F5.

The coefficient
field is the field of rational functions in all variables occurring in
\code{expr\_1}, \dots,~\code{expr\_m} but not in \code{var\_1}, \dots,~
\code{var\_n}. The field of rational numbers is a special case.
%
\paragraph{Example}
On input of
\begin{verbatim}
1: load_package f5;

2: torder({x, y, z}, lex)$

3: f5({x*y + 1, y*z + 1});
\end{verbatim}
we obtain the \grobner{} basis
\code{\{x - z, y*z + 1\}} using lexicographic order with $\code{x} \succ
\code{y} \succ \code{z}$. A subsequent call
\begin{verbatim}
4: f5({x*y + 1, y*z + 1}, {z, y, x}, lex);
\end{verbatim}
with optional arguments shadows our \code{torder} setting above in favor of $
\code{z}
\succ \code{y} \succ \code{x}$. We then obtain the \grobner{} basis
\code{\{z - x, y*x + 1\}}.
%
\section{Switches}
\label{SE:switches}

\begin{description}
\item[\code{f5fractionfree:}] Default is \code{off}. If set \code{on}, keep
coefficients fraction-free during computation.
%
\item[\code{f5interreduce:}] Default is \code{off}.
If set \code{on}, inter-reduce the resulting basis, which makes it unique
wrt.~the chosen term order. Note that even with
\code{off f5interreduce} the output basis has minimal number of elements.
%
\item[\code{f5parametric:}] Default is \code{off}. If set \code{on}, parametric coefficients are not
considered as rational functions but as parametric expressions. Necessary assumptions on the
non-vanishing of such parametric expressions are made during computation. They can be queried using
the function \code{f5dumpAs\-sump\-tions}. Figure~\ref{fig:para} shows an example.
%
\item[\code{f5parametricNormalize:}] This switch is considered only with \code{on f5para\-met\-ric}.
Its default is \code{off}. \emph{Normalize} refers to the conversion to primitive polynomials with
\code{on f5fractionfree} and to monic polynomials otherwise. If set \code{off}, normalization
factors are computed over the integers. If set \code{on}, normalization factors are computed over
the polynomial ring in the parameters, and necessary assumptions on their non-vanishing are made.
%
\item[\code{f5statistics:}] Default is
\code{off}. If set \code{on}, collect and print statistics about \grobner{}
basis computation with each call to \code{f5}. Figure~\ref{fig:stat} shows an example.
%
\item[\code{f5sugar:}]  Default is \code{on}. If set \code{on}, use sugar selection strategy;
otherwise use the normal selection strategy.
%
\item[\code{f5usef5c:}] Default is \code{off}. If set \code{on}, use the F5C algorithm described in
\cite{f5c}.
\end{description}
%
\begin{figure}[p]
\footnotesize
\begin{verbatim}
1: load_package f5;

2: on f5parametric;

3: f5({a*x*y + b*x, (1-b)*x*y^2 - 1}, {x, y}, lex);

      3    2       2        3  3    3  2        2  4    2  3
{ - (b  - b )*x - a ,   - (a *b  - a *b )*y - (a *b  - a *b ) }

4: f5dumpAssumptions();

             2
    {a <> 0,b  - b <> 0}
\end{verbatim}
\caption{Assumptions made with switch \code{f5parametric}\label{fig:para}}
\end{figure}
%
\begin{figure}[p]
\footnotesize
\begin{verbatim}
1: load_package f5;

2: system := {z1 + z2 + z3 + z4, 
              z1*z2 + z1*z4 + z2*z3 + z3*z4,
              z1*z2*z3 + z1*z2*z4 + z1*z3*z4 + z2*z3*z4, 
              z1*z2*z3*z4 - 1};

3: on f5statistics;

4: f5(system, {z1,z2,z3,z4}, revgradlex);

      2   4        5        5                     2
{ - z3 *z4  - z1*z4  - z3*z4  - z2*z3 - z3*z4 + z4 ,

   3   2           3           3           3     2   3
 z3 *z4  + z1*z2*z4  + z1*z3*z4  + z2*z3*z4  + z3 *z4  - z3,

      4     5                    2     2   2        3          3
 z2*z4  + z4  - z2 - z4, z2*z3*z4  + z3 *z4  + z1*z4  + 2*z3*z4  - 1,

      2                           2
 z2*z3  - z1*z2*z4 - z2*z3*z4 + z3 *z4,

      2
  - z2  + z1*z4 - z2*z4 + z3*z4,  z1 + z2 + z3 + z4 }

-------------------------------------------------------------------------------
| # i | # pairs | # reductions | # zero reductions | # normal forms | degrees |
-------------------------------------------------------------------------------
|  1  |    0    |       0      |         0         |        0       |    -    |
|  2  |    1    |       1      |         1         |        2       |  4 - 4  |
|  3  |    3    |       3      |         0         |        4       |  5 - 6  |
-------------------------------------------------------------------------------
 Total     4            4                1                  6
\end{verbatim}
\caption{Statistical information with switch \code{f5statistics}
\label{fig:stat}}
\end{figure}
%
\section{Symbolic Mode Interface}

In symbolic mode the term order must be specified explicitly before \grobner{} basis computation.
%
\begin{description}\sloppy
\item[\code{torder(\{\{'list, var\_1, \ldots, var\_n\}, ord\}):}] with variables \code{var\_1},
\dots, \code{var\_n} and sort mode \code{ord} as described in Section~\ref{SE:usage}. The return
value is the old setting.
\end{description}

There are three alternative symbolic mode entry points for \grobner{} basis computation, depending
on the syntax of input and output polynomials:
\begin{description}
\item[\code{f5\_groebnera(\{expr\_1, \ldots, expr\_m\}):}] \code{expr\_1}, \dots, \code{expr\_m} are
Lisp Prefix Forms. The return value is a list of polynomials as Lisp Prefix Forms, which are ordered
with respect to the chosen term order.
%
\item[\code{f5\_groebnerf(\{expr\_1, \ldots, expr\_m\}):}] \code{expr\_1}, \dots, \code{expr\_m} are
polynomials as Standard Forms. The return value is a list of polynomials as Standard Forms, which
are ordered with respect to the current REDUCE kernel order. Note that the information on the term
order gets lost here.
%
\item[\code{f5\_groebnerPoly(\{expr\_1, \ldots, expr\_m\}):}] \code{expr\_1}, \dots, \code{expr\_m}
are polynomials as Polynomial data structures of the F5 package. The return value is a list of
polynomials as such Polynomial data structures.
%
\item[\code{f5\_groebnerq(\{expr\_1, \ldots, expr\_m\}):}] \code{expr\_1}, \dots, \code{expr\_m} are
polynomials as Standard Quotients, possibly with parametric denominators. The return value is a list
of polynomials as Standard Quotients, which are ordered with respect to the current REDUCE kernel
order. Note that the information on the term order gets lost here.
\end{description}

Finally, assumptions can be accessed in symbolic mode as follows:
\begin{description}
\item[\code{param\_dumpAssumptions():}] the return value is a list of polynomial inequations in the
form \code{(neq expr nil)}, where \code{expr} is a Lisp Prefix polynomial. This format is suitable
to be processed with REDLOG.
\end{description}

\begin{thebibliography}{BWK93}

\bibitem[BWK93]{tolstaya}
Thomas Becker, Volker Weispfenning, and Hainz Kredel.
\newblock {\em Gr{\"o}bner Bases}, volume 141 of {\em Graduate Texts in
  Mathematics}.
\newblock Springer, 1993.

\bibitem[EF17]{survey}
Christian Eder and Jean-Charles Faugère.
\newblock A survey on signature-based algorithms for computing {G}röbner
  bases.
\newblock {\em J. Symb. Comput.}, 80:719--784, 2017.

\bibitem[EP10]{f5c}
Christian Eder and John Perry.
\newblock {F5C}: A variant of {F}aug{\`{e}}re's {F5} algorithm with reduced
  {G}r{\"o}bner bases.
\newblock {\em J. Symb. Comput.}, 45(12):1442--1458, 2010.

\bibitem[Fau02]{f5}
Jean~Charles Faugère.
\newblock A new efficient algorithm for computing {G}röbner bases without
  reduction to zero ($\textrm{{F}}_5$).
\newblock In {\em Proc. ISSAC 2002}, pages 75--83. ACM, 2002.

\end{thebibliography}

% \bibliography{f5}
% \bibliographystyle{alpha}

\end{document}
