\documentclass{article}

\usepackage[dvipdfmx]{graphicx}
\usepackage[dvipdfmx]{color}
\usepackage[dvipdfmx]{hyperref}

\setlength{\parindent}{0cm}

\title{SOFIA LAPLACE AND INVERSE LAPLACE TRANSFORM PACKAGE}
\author{C. Kazasov\and M. Spiridonova \and V. Tomov}
\date{}


\begin{document}
\maketitle

\begin{center}
\begin{tabular}{lp{10cm}}
Reference: & {\bf Christomir Kazasov}, Laplace Transformations in REDUCE 3, Proc.
Eurocal '87, Lecture Notes in Comp. Sci., Springer-Verlag
(1987) 132-133.
\end{tabular} 
\end{center}
\ \\
\ \\
Some hints on how to use to use this package: \\
\ \\
Syntax: \\
\ \\
{\tt LAPLACE($<exp>,<var-s>,<var-t>$ }) \\
\ \\
{\tt INVLAP($<exp>,<var-s>,<var-t>$)} \\
\ \\
where $<exp>$ is the expression to be transformed, $<var-s>$ is the source
variable (in most cases $<exp>$ depends explicitly of this variable) and
$<var-t>$ is the target variable. If $<var-t>$ is omitted, the package uses
an internal variable lp!\& or il!\&, respectively. \\
\ \\
The following switches can be used to control the transformations: \\
\begin{center}
\begin{tabular}{lp{10cm}}
{\tt lmon}: & If on, sin, cos, sinh and cosh are converted by {\tt LAPLACE} into
exponentials, \\
{\tt lhyp}: & If on, expressions $e^{\tilde{}x}$ are converted by {\tt INVLAP} into 
hyperbolic functions sinh and cosh, \\
{\tt ltrig}: & If on, expressions $e^{\tilde{}x}$ are converted by {\tt INVLAP} into
trigonometric functions sin and cos. \\
\end{tabular} 
\end{center}
\ \\
The system can be extended by adding Laplace transformation rules for
single functions by rules or rule sets.~ In such a rule the source
variable MUST be free, the target variable MUST be il!\& for {\tt LAPLACE} and
lp!\& for {\tt INVLAP} and the third parameter should be omitted.~ Also rules for
transforming derivatives are entered in such a form. \\

\pagebreak
{\bf Examples:}
\begin{verbatim}

    let {laplace(log(~x),x) => -log(gam * il!&)/il!&,

    invlap(log(gam * ~x)/x,x) => -log(lp!&)};


    operator f;

    let{

    laplace(df(f(~x),x),x) => il!&*laplace(f(x),x) - sub(x=0,f(x)),

    laplace(df(f(~x),x,~n),x) => il!&**n*laplace(f(x),x) -

    for i:=n-1 step -1 until 0 sum

    sub(x=0, df(f(x),x,n-1-i)) * il!&**i

    when fixp n,

    laplace(f(~x),x) = f(il!&)

    };

\end{verbatim}


Remarks about some functions: \\
\ \\
The DELTA and GAMMA functions are known. \\
ONE is the name of the unit step function. \\
INTL is a parametrized integral function 
\begin{center}
{\tt intl($<expr>,<var>,0,<obj.var>$)}
\end{center}
which means \char`\"{}Integral of $<expr>$ wrt.~ $<var>$ taken from 0 to $<obj.var>$\char`\"{},
e.g. {\tt intl($2{*}y^2,y,0,x$)} which is formally a function in $x$.
\ \\
\ \\
We recommend reading the file LAPLACE.TST for a further introduction.

\end{document}
