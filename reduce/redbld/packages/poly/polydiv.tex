\documentclass[11pt]{article}

\title{POLYDIV: Enhanced Polynomial Division}

\author{Francis J. Wright \\
School of Mathematical Sciences \\
Queen Mary and Westfield College \\
University of London \\
Mile End Road, London E1 4NS, UK. \\
Email: {\tt F.J.Wright@QMW.ac.uk}}

\date{6 November 1995}

\begin{document}
\maketitle

\begin{abstract}
  This package provides better access to the standard internal
  polynomial division facilities of REDUCE and implements polynomial
  pseudo-division.  It provides optional local control over the main
  variable used for division.
\end{abstract}


\section{Introduction}

The \texttt{polydiv} package provides several enhancements to the
standard REDUCE algebraic-mode facilities for Euclidean division of
polynomials.  The source file (\texttt{polydiv.red}) should be
compiled (using \texttt{faslout}) and loaded when required by
\begin{verbatim}
load_package polydiv;
\end{verbatim}
The numerical coefficient domain is always that specified globally.
Further examples are provided in the test and demonstration file
\texttt{polydiv.tst}.


\section{Polynomial Division}

The \texttt{polydiv} package provides the infix operators \texttt{div}
and \texttt{mod} (as used in Pascal) for the Euclidean quotient and
remainder, e.g.
\begin{verbatim}
(x^2 + y^2) div (x - y);

        x + y

(x^2 + y^2) mod (x - y);

           2
        2*y
\end{verbatim}
(They can also be used as prefix operators.)

It provides a Euclidean division operator \texttt{divide} that returns
both the quotient and the remainder together as the first and second
elements of a list, e.g.
\begin{verbatim}
divide(x^2 + y^2, x - y);

                  2
        {x + y,2*y }
\end{verbatim}
(It can also be used as an infix operator.)

All Euclidean division operators (when used in prefix form, and
including the standard \texttt{remainder} operator) accept an optional
third argument, which specifies the main variable to be used during
the division.  The default is the leading kernel in the current global
ordering.  Specifying the main variable does not change the ordering
of any other variables involved, nor does it change the global
environment.  For example
\begin{verbatim}
div(x^2 + y^2, x - y, y);

         - (x + y)

remainder(x^2 + y^2, x - y, y);

           2
        2*x

divide(x^2 + y^2, x - y, y);

                       2
        { - (x + y),2*x }
\end{verbatim}

Specifying $x$ as main variable gives the same behaviour as the
default shown earlier, i.e.
\begin{verbatim}
divide(x^2 + y^2, x - y, x);

                  2
        {x + y,2*y }

remainder(x^2 + y^2, x - y, x);

           2
        2*y
\end{verbatim}


\section{Polynomial Pseudo-Division}

The polynomial division discussed above is normally most useful for a
univariate polynomial over a field, otherwise the division is likely
to fail giving trivially a zero quotient and a remainder equal to the
dividend.  (A ring of univariate polynomials is a Euclidean domain
only if the coefficient ring is a field.)  For example, over the
integers:
\begin{verbatim}
divide(x^2 + y^2, 2(x - y));

            2    2
        {0,x  + y }
\end{verbatim}

The division of a polynomial $u(x)$ of degree $m$ by a polynomial
$v(x)$ of degree $n \le m$ can be performed over any commutative ring
with identity (such as the integers, or any polynomial ring) if the
polynomial $u(x)$ is first multiplied by $\mathrm{lc}(v,x)^{m-n+1}$
(where lc denotes the leading coefficient).  This is called
\emph{pseudo-division}.  The \texttt{polydiv} package implements the
polynomial pseudo-division operators \texttt{pseudo\_divide},
\texttt{pseudo\_quotient} (or \texttt{pseudo\_div}) and
\texttt{pseudo\_remainder} as prefix operators (only).  When
multivariate polynomials are pseudo-divided it is important which
variable is taken as the main variable, because the leading
coefficient of the divisor is computed with respect to this variable.
Therefore, if this is allowed to default and there is any ambiguity,
i.e.\ the polynomials are multivariate or contain more than one
kernel, the pseudo-division operators output a warning message to
indicate which kernel has been selected as the main variable -- it is
the first kernel found in the internal forms of the dividend and
divisor.  (As usual, the warning can be turned off by making the
switch setting ``\texttt{off msg;}''.)  For example
\begin{verbatim}
pseudo_divide(x^2 + y^2, x - y);

        *** Main division variable selected is x 

                  2
        {x + y,2*y }

pseudo_divide(x^2 + y^2, x - y, x);

                  2
        {x + y,2*y }

pseudo_divide(x^2 + y^2, x - y, y);

                       2
        { - (x + y),2*x }
\end{verbatim}

If the leading coefficient of the divisor is a unit (invertible
element) of the coefficient ring then division and pseudo-division
should be identical, otherwise they are not, e.g.
\begin{verbatim}
divide(x^2 + y^2, 2(x - y));

            2    2
        {0,x  + y }

pseudo_divide(x^2 + y^2, 2(x - y));

        *** Main division variable selected is x

                      2
        {2*(x + y),8*y }
\end{verbatim}

The pseudo-division gives essentially the same result as would
division over the field of fractions of the coefficient ring (apart
from the overall factors [contents] of the quotient and remainder),
e.g.
\begin{verbatim}
on rational;

divide(x^2 + y^2, 2(x - y));

          1             2
        {---*(x + y),2*y }
          2

pseudo_divide(x^2 + y^2, 2(x - y));

        *** Main division variable selected is x

                      2
        {2*(x + y),8*y }
\end{verbatim}

Polynomial division and pseudo-division can only be applied to what
REDUCE regards as polynomials, i.e.\ rational expressions with
denominator 1, e.g.
\begin{verbatim}
off rational;

pseudo_divide((x^2 + y^2)/2, x - y);

                2    2
               x  + y
        ***** --------- invalid as polynomial
                  2
\end{verbatim}

Pseudo-division is implemented in the \texttt{polydiv} package using
an algorithm (D. E. Knuth 1981, \textit{Seminumerical Algorithms},
Algorithm R, page 407) that does not perform any actual division at
all (which proves that it applies over a ring).  It is more efficient
than the naive algorithm, and it also has the advantage that it works
over coefficient domains in which REDUCE may not be able to perform in
practice divisions that are possible mathematically.  An example of
this is coefficient domains involving algebraic numbers, such as the
integers extended by $\sqrt{2}$, as illustrated in the file
\texttt{polydiv.tst}.

The implementation attempts to be reasonably efficient, except that it
always computes the quotient internally even when only the remainder
is required (as does the standard remainder operator).

\end{document}